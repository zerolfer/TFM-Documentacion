\makeglossaries

%
%   TEMPLATE:
%
%\newglossaryentry{ }
%{
%    name={ },
%    description=
%    {   
%        
%    }
%}
%
%

\newacronym{acr:ATC}{ATC}{Air Traffic Control}

%\newacronym{ATC}{ATC}{Air Traffic Control (Control de Tránsito Aéreo)}

\newglossaryentry{ABACO}
{
    name={ABACO},
    description=
    {
        Nombre del proyecto
    }
}

\newglossaryentry{ACC (Centro de Control)}
{
    name={ACC (Centro de Control)},
    description=
    {
        <<Centro de control de tránsito aéreo responsable de la circulación aérea segura a lo largo de las rutas ATS 
        (servicio de tránsito aéreo). Un ACC se divide en varios sectores, cada uno de los cuales tiene claramente 
        definidas sus responsabilidades. Los procedimientos para transferir una aeronave de un sector a otro entre 
        estados limítrofes están perfectamente definidos por acuerdos internacionales, así como 
        bilaterales>>.~\cite{ENAIRE-web}
    }
}

\newglossaryentry{__ATC}
{
    name={ATC (Control de Tránsito Aéreo)},
    description=
    {   
        <<Término común que designa todos los servicios proporcionados para asegurar y acelerar el flujo de tráfico 
        aéreo a través del espacio aéreo controlado>>.~\cite{ENAIRE-web}
    }
}

\newglossaryentry{CRIDA}
{
    name={CRIDA},
    description=
    {   
        \textit{acrón.} Centro de Referencia de Investigación, Desarrollo e Innovación 
        en \hyperref[ATC]{Gestión del Tráfico Aéreo}. Agrupación de interés económico (A.I.E.) sin ánimo de lucro 
        establecida por ENAIRE, la Universidad Politécnica de Madrid (UPM) e Ingeniería y Economía del Transporte, S.A. 
        (INECO).~\cite{CRIDA-web}
    }
}

\newglossaryentry{Nucleo}
{
    name={Núcleo},
    description=
    {   
        Conjunto de Sectores. Un sector puede pertenecer a más de un núcleo (relación N a N). Esta 
        agrupación se lleva a cabo para poder gestionar los posibles sectores que cada controlador puede controlar, de 
        esta 
        forma, un controlador tiene acreditación para un único núcleo.
    }
}

\newglossaryentry{Matriz-afinidad}
{
    name={Matriz de Afinidad},
    description=
    {   
        Tabla o matriz booleana cuyas filas y columnas son los diferentes sectores de 
        una Unidad de Control dada, de forma que la intersección de dos sectores tendrá el valor de Cierto si y solo si 
        los sectores son afines entre sí (relación bidireccional)
    }
}

\newglossaryentry{Proyecto-Airport}
{
    name={Proyecto Airport 2050+},
    description=
    {   
        Proyecto europeo con colaboración española por parte de UPM, CRIDA e INECO
    }
}

\newglossaryentry{TMA}
{
    name={TMA},
    description=
    {   
        Área de control terminal. <<Espacio aéreo controlado en torno a uno o varios aeropuertos 
        donde se realizan las maniobras de aproximación (aterrizajes y despegues)>>~\cite{ENAIRE-web}.
    }
}



\newglossaryentry{ATC}
{
    type=\acronymtype,
    name={ATC}, 
    description=
    {
        Centro de Referencia de Investigación, Desarrollo e Innovación 
        en \hyperref[ATC]{Gestión del Tráfico Aéreo}. Agrupación de interés económico (A.I.E.) sin ánimo de lucro 
        establecida por ENAIRE, la Universidad Politécnica de Madrid (UPM) e Ingeniería y Economía del Transporte, S.A. 
        (INECO).~\cite{CRIDA-web}
     }, 
%    first=
%    {
%        Control del espacio aéreo (también conocido como \glsadd{ATC}, \acrlong{acr:ATC})
%    }
%    see=[Glossary:]{apig}
    long={Air Traffic Management}
}
