\makeglossaries

%
%   TEMPLATE (versión larga):
%
%\newglossaryentry{ } % etiqueta de referencia
%{
%    type=\acronymtype, % indicar que es acrónimo en caso de serlo. combinar con "long"
%    name={ }, % forma de aparición en el glosario
%    text={ }, % forma de aparición en el texto
%	 long={ }, % forma larga (muéstrese mediante \acrlong{label}) 
%    first={ }, % cado especial para la primera aparición en el texto
%    plural={ }, % forma en plural (muéstrese mediante \glspl{label})
%    firstplural={ },
%    symbol={ }, % símbolo del término (úsese mediante ¿?)
%    description= 
%    {   
%        
%    }
%}
%

%
%   TEMPLATE (versión mínima):
%
%\newglossaryentry{ } % etiqueta de referencia
%{
%    name={ }, % forma de aparición en el glosario
%    description= 
%    {   
%        
%    }
%}
%


%\newacronym{acr:ATC}{ATC}{Air Traffic Control}

\newglossaryentry{ATC}
{
%    type=\acronymtype, % indicar que es acrónimo
    long={Air Traffic Management}, % y su forma larga. Usar mediante \acrlong{label}
    name={ATC}, 
    description=
    {
       <<Término común que designa todos los servicios proporcionados para asegurar y acelerar el flujo de tráfico 
       aéreo a través del espacio aéreo controlado>>~\cite{ENAIRE-web}
    }
%    first=
%    {
%        ATC, \textit{\acrlong{ATC}}
%    }
%    see=[Glossary:]{apig}
}

\newglossaryentry{ABACO}
{
    name={ABACO},
    description=
    {   
        Nombre del proyecto previo a este, cuyo objetivo pretendia automatizar el proceso de planificación de controladores aéreos de forma ``estratégica", es decir, con antelación previa, empleando información de predicción de la sectorización en lugar de la real. Es por ello que el tiempo de ejecución no era crucial en éste sistema.
        No es otro que el nombre oficial del sistema que en el presente documento hemos decidido denominar \legacy{}.
    }
}

\newglossaryentry{Centro de Control}
{
    name={Centro de Control (ACC)},
    text={ACC},
%	type=\acronymtype, % indicar que es acrónimo
	long={Centro de Control}, % y su forma larga. Usar mediante \acrlong{label}
	firstplural= {Centros de Control de Tráfico Aéreo},
	plural={Centros de Control (ACC)},
%	first={ACC (Centro de Control)},
    description=
    {
        <<Centro de control de tránsito aéreo responsable de la circulación aérea segura a lo largo de las rutas ATS 
        (servicio de tránsito aéreo). Un ACC se divide en varios sectores, cada uno de los cuales tiene claramente 
        definidas sus responsabilidades. Los procedimientos para transferir una aeronave de un sector a otro entre 
        estados limítrofes están perfectamente definidos por acuerdos internacionales, así como 
        bilaterales>> (también como Unidad de Control).~\cite{ENAIRE-web}
    }
}

\newglossaryentry{CRIDA}
{
    name={CRIDA},
    description=
    {   
        \textit{acrón.} Centro de Referencia de Investigación, Desarrollo e Innovación 
        en Gestión del Tráfico Aéreo (\gls{ATC}). Agrupación de interés económico (A.I.E.) sin ánimo de lucro 
        establecida por ENAIRE, la Universidad Politécnica de Madrid (UPM) e Ingeniería y Economía del Transporte, S.A. 
        (INECO).~\cite{CRIDA-web}
    }
}

\newglossaryentry{Nucleo}
{
    name={Núcleo},
    description=
    {   
        Conjunto de Sectores. Un sector puede pertenecer a más de un núcleo (relación N a N). Esta 
        agrupación se lleva a cabo para poder gestionar los posibles sectores que cada controlador puede controlar, de 
        esta 
        forma, un controlador tiene acreditación para un único núcleo.
    }
}

\newglossaryentry{Matriz-afinidad}
{
    name={Matriz de Afinidad},
    description=
    {   
        Tabla o matriz booleana cuyas filas y columnas son los diferentes sectores de 
        una Unidad de Control dada, de forma que la intersección de dos sectores tendrá el valor de Cierto si y solo si 
        los sectores son afines entre sí (relación bidireccional)
    }
}

\newglossaryentry{Proyecto-Airport}
{
    name={Proyecto Airport 2050+},
    description=
    {   
        Proyecto europeo con colaboración española por parte de UPM, CRIDA e INECO que busca nuevas propuestas y conceptos para apoyar el desarrollo de los aeropuertos para el año 2050.~\cite{airports-web}
        %TODO: "http://www.upm.es/Estudiantes?fmt=detail&prefmt=articulo&id=c89071caf5738310VgnVCM10000009c7648a____"
    }
}

\newglossaryentry{TMA}
{
    name={TMA},
    long={Área de control terminal},
    description=
    {   
        Área de control terminal. <<Espacio aéreo controlado en torno a uno o varios aeropuertos 
        donde se realizan las maniobras de aproximación (aterrizajes y despegues)>>~\cite{ENAIRE-web}.
    }
}

\newglossaryentry{FIR}
{
    name={FIR},
    long={Flight Information Region},
    firstplural={FIRs (Flight Information Region)},
    description=
    {   
        Unidad de subdivisión del espacio aéreo a nivel mundial empleado por todos los países europeos (entre otros) 
        con el objetivo de tener un marco común de gestión aérea sobre el que organizarse. El territorio español abarca 
        dos sectores, conocidos como MADRID y BARCELONA pese a que comprenden regiones nacionales mucho mayores 
        (véaser~\autoref{fig:2:fireuropa})
    }
}



\newglossaryentry{SA} % etiqueta de referencia
{
    type=\acronymtype, % indicar que es acrónimo en caso de serlo. combinar con "long"
    name={Simulated Annealing (Recocido Simulado)}, % forma de aparición en el glosario
    text={SA}, % forma de aparición en el texto
	 long={Simulated Annealing}, % forma larga (muéstrese mediante \acrlong{label}) 
    first={\textit{Simulated Annealing} (SA)}, % cado especial para la primera aparición en el texto
    description= 
    {   
        Metaheurística trayectorial de búsqueda local basada en la aceptación de soluciones peores según el valor de su parámetro que varía con el tiempo una cantidad constante de forma que al final de la ejecución ya no se aceptan soluciones peores en ningún caso. %TODO: mejorar definición
    }
}

\newglossaryentry{VNS} % etiqueta de referencia
{
	type=\acronymtype, % indicar que es acrónimo en caso de serlo. combinar con "long"
	name={Variable Neighborhood Search (Búsqueda de Entorno Variable)}, % forma de aparición en el glosario
	text={VNS}, % forma de aparición en el texto
	long={Variable Neighborhood Search}, % forma larga (muéstrese mediante \acrlong{label}) 
	first={\textit{Variable Neighborhood Search} (VNS)}, % cado especial para la primera aparición en el texto
	description= 
	{   
		Metaheurística trayectorial basada en una búsqueda local continuada a lo largo de un conjunto dado de entornos o vecindades. Cuando la búsqueda local llega a un óptimo local, se da paso a otro de los entornos y se repite nuevamente la búsqueda local.
	}
}