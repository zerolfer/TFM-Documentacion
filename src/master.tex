% !TeX spellcheck = es_ES

\documentclass[spanish,12pt, a4paper,twoside]{article}

\let\oldsection\section
\def\section{\cleardoublepage\oldsection}

\usepackage{afterpage}
\usepackage{ulem}
\usepackage[dvipsnames]{xcolor}

\usepackage{xfrac}

\newcommand\blankpage{
    \null
    \thispagestyle{empty}
    \addtocounter{page}{-1}
    \newpage
}


% Para que \paragraph tenga la misma forma que un hipotético \subsubsubsection
\makeatletter
\renewcommand\paragraph{\@startsection{paragraph}{4}{\z@}%
	{-2.5ex\@plus -1ex \@minus -.25ex}%
	{1.25ex \@plus .25ex}%
	{\normalfont\normalsize\bfseries}}

\renewcommand\subparagraph{\@startsection{subparagraph}{5}{\z@}%
	{-2.5ex\@plus -1ex \@minus -.25ex}%
	{1.25ex \@plus .25ex}%
	{\normalfont\normalsize\bfseries\sffamily}}

%\newcommand\footnoteref[1]{\protected@xdef\@thefnmark{\ref{#1}}\@footnotemark}
\makeatother
%\titlespacing*{\subparagraph}{0pt}{3.5ex plus 1ex minus .2ex}{2.3ex plus .2ex}
%\renewcommand*{\subparagraphformat}{\numberinmargin{\thesubparagraph}}


\setcounter{tocdepth}{4}
\setcounter{secnumdepth}{4}

%\renewcommand{\listoftables}{Índice de tablas}
\newcommand{\refcruzada}[2]{\hyperref[#2]{#1~\ref{#2}}}

\newcommand{\NOTE}[1]{\textcolor{ForestGreen}{#1}}
\newcommand{\NEW}[1]{  \colorbox{RubineRed!30!}{ \parbox{\textwidth}{#1} }  }

\newcommand*{\legacy}{\texttt{Legacy}}
\newcommand*{\faseuno}{\hyperref[sec:3:inicializacion-soluciones]{Fase 1}}
\newcommand*{\fasedos}{\hyperref[sec:3:metaheurística]{Fase 2}}

\newcommand*{\sa}{\textit{Simulated Annealing}}
\newcommand*{\vns}{\textit{Variable Neighborhood Search}}


%\newcommand{\sublist}[1]{
%	\begin{enumerate}[label*={\textbf{.\arabic*}}]
%		#1
%	\end{enumerate}
%}

\renewcommand{\labelitemi}{$\square$}
\renewcommand{\labelitemii}{$-$}
\renewcommand{\labelitemiii}{$\bullet$}
\renewcommand{\labelitemiv}{$\ast$}

\makeatletter
\def\namedlabel#1{\begingroup
\def\@currentlabel{#1}%
\@currentlabel
\phantomsection\label{item:\@currentlabel}\endgroup
}
\makeatother


\usepackage[textwidth=15cm, textheight=22.5cm, top=3.5cm, bottom=3.5cm,left= 4cm,right=2cm]{geometry}


\usepackage[spanish]{babel}

\decimalpoint
\usepackage[utf8]{inputenc}

\usepackage{graphicx}
\usepackage{graphics}
\usepackage{subcaption}
\usepackage{amsmath}
\usepackage{float}
\usepackage{changepage}

\usepackage{enumitem}


\usepackage[nottoc, notlot, notlof, numbib]{tocbibind}
\usepackage[spanish, onelanguage, ruled, vlined, algosection]{algorithm2e}
\newcommand{\Comment}[1]{ \tcp*[r]{#1} }
\newcommand{\algovspace}{ \medskip }
\newcommand{\algorefpaso}[1]{\texttt{\autoref{#1}}}
%\newcommand{\listofalgorithmes}{\tocfile{\listalgorithmcfname}{loa}}


\usepackage{setspace}
\usepackage{multirow}
\usepackage{cancel}
\renewcommand\CancelColor{\color{red}}


\usepackage[breaklinks=true]{hyperref}
\usepackage[T1]{fontenc}
\usepackage{textcomp}
\usepackage{ amssymb }
\usepackage{parskip}
%\usepackage{natbib} 
\usepackage{cleveref}
\crefformat{footnote}{#2\footnotemark[#1]#3}
%\usepackage{numprint}
%\npthousandsep{\,}
%\usepackage{siunitx}
\usepackage[numberedsection, nopostdot, nogroupskip]{glossaries} % must be under hyperref
%\usepackage{glossary-long}

%
%   GLOSSARY
%
\setglossarysection{subsection} % En caso de incluirlo, el glosario es subsección en vez de sección
\loadglsentries{glossary}
\setglossarystyle{altlist}

%\renewcommand{\cftsectnumwidth}{6em}

\usepackage[page,title,titletoc]{appendix}
\usepackage{pdfpages}
\usepackage{pdflscape}

\usepackage{placeins}



\pdfminorversion=7
%
%   DESCOMENTAR LO SIGUIENTE PARA CAMBIAR EL ESTILO DEL GLOSARIO DE "list" (default) A "super":
%

% _________________________________________________________________________________________________
%\setglossarystyle{longborder} % TODO: remove "border", just "long" ; para mas estilos: https://bit.ly/2o1Qa6j 
% %\setlength{\glsdescwidth}{5.05in}
%\setlength{\glsdescwidth}{0.8\textwidth} % NOTE: ver https://bit.ly/2n9cLOn y/o https://bit.ly/2n0kRJh
%
%
%\renewenvironment{theglossary}
%{
%    \begin{longtable}[l]{lp{\glsdescwidth}}}
%    {\end{longtable}
%}
% _________________________________________________________________________________________________
	
	
\renewcommand{\glsnamefont}[1]{\textbf{#1}}
\renewcommand{\glsgroupskip}{}
%\let \oldautoref \autoref
%\renewcommand\autoref{1}{\uline{\oldautoref{#1}}}

\hypersetup{
    pdftitle={Resolución Táctica del Problema de Asignación de Controladores Aéreos a Puestos de Control},
    pdfauthor={Sergio Flórez Vallina}
    backref,
    unicode=true,
    bookmarksnumbered=true,
    bookmarksopen=true,
    bookmarksopenlevel=1,
    %	pdfborder={1 1 1},
%    colorlinks=true,
%    	hidelinks=true,
    %pdfpagemode=UseOutlines,    % this is the option you were lookin for
    %pdfpagelayout=TwoPageRight
    %	colorlinks=false,% hyperlinks will be black
   	linkbordercolor=MidnightBlue,
%    linkcolor=MidnightBlue
    pdfborderstyle={/S/U/W 1}
    bordercolor
%    	pdfborderstyle={/S/U/W 1}% border style will be underline of width 1pt
}

\addto\extrasspanish{\def\sectionautorefname{Capítulo}}
\addto\extrasspanish{\def\subsectionautorefname{Sección}}
\addto\extrasspanish{\def\subsubsectionautorefname{Sección}}
\addto\extrasspanish{\def\paragraphautorefname{Sección}}

\addto\extrasspanish{\def\max{\text{max\,}}}
\addto\extrasspanish{\def\min{\text{min\,}}}

\addto\extrasspanish{\def\appendixautorefname{Anexo}}

\addto\extrasspanish{\def\algorithmautorefname{Algoritmo}}
\renewcommand{\algorithmcflinename}{Paso}

\usepackage[fixlanguage]{babelbib}
\selectbiblanguage{spanish}
\declarebtxcommands{spanish}{% Esto es para el estilo de bibliografía "bababbrv"
	\def\btxmastthesis#1{\protect\foreignlanguage{spanish}{Trabajo de Fin de Máster}}%
	\def\btxphdthesis#1{\protect\foreignlanguage{spanish}{Tesis Doctoral}}%
}


\captionsetup{labelfont=bf}
\captionsetup[subfigure]{labelfont=bf}

% las dos líneas siguientes hacen que las figuras estén subnumeradas según el numero de la sección "num-section.num-fig"
\usepackage{chngcntr}
\counterwithin{figure}{section}

%\includeonly{capitulos/Capitulo1-Introduccion/Capitulo1, capitulos/Capitulo2-Definicion-del-problema/Capitulo2}
\begin{document}
	\pagenumbering{gobble}% Remove page numbers (and reset to 1)
    % renombrar strings de babel:
    \renewcommand{\listtablename}{Índice de tablas}
    \renewcommand{\tablename}{Tabla}
%    \renewcommand{\figureshortname}{Figura}
%    \renewcommand{\tableshortname}{Tabla}

	\renewcommand\appendixname{Anexo}
	\renewcommand\appendixpagename{Anexos (aún sin completar)}
   	\renewcommand\appendixtocname{Anexos}
   	
    %\maketitle
    %\thispagestyle{empty}
    \begin{titlepage}
    
        % Defines a new command for the horizontal lines, change thickness here
        \newcommand{\HRule}{\rule{\linewidth}{0.5mm}} 

        \center % Center everything on the page

        %	HEADING SECTIONS
        \includegraphics[width=2.25cm]{recursos/logoFi.png}
        \hspace{8cm}
        \includegraphics[width=2cm]{recursos/logoupm.png}
        \\[1cm]

        \textsc{\Large Escuela Técnica Superior de Ingenieros Informáticos}\\[0.5cm]
        \textsc{\large Universidad Politécnica de Madrid}
        \\[3cm]


        %	TITLE SECTION
        \HRule \\[0.4cm]
        { \huge \bfseries Resolución Táctica del Problema de Asignación de Controladores Aéreos a Puestos
        de Control}\\%[0.4cm] % Title of your document
        \HRule \\[1.5cm]

        \textsc{\LARGE Trabajo de Fin de Máster}\\[0.5cm]
        \textsc{\large Máster Universitario en Inteligencia Artificial (MUIA) }\\[2.5cm]

        %	AUTHOR SECTION
        \begin{flushright}
            \large
            AUTOR: Sergio Flórez Vallina\\
            TUTORES: Alfonso Mateos Caballero y \linebreak
            Antonio Jiménez Martín
        \end{flushright}

        \vspace{1.3cm}

        %	DATE SECTION
        { 
        	{Curso 2020/21}%\\[0.4cm]
    	}\\[0.8cm]
    
        {
            {v0.7}
        }

                    
        %	LOGO SECTION

        \vfill
        % Fill the rest of the page with whitespace

    \end{titlepage}

    \afterpage{\blankpage}
    \pagenumbering{roman}


    \section*{AGRADECIMIENTOS}
    Quiero dar las gracias a todos los amigos/as, compañeras/os y familiares que se han interesado por cómo iba mi proyecto y se han ilusionado conmigo en cada avance. 
    
    Agradezco especialmente a Tino por toda su paciencia y dedicación para ayudarme
    a dar vida a este proyecto, sin su ayuda aquellos primeros días en su despacho todo habría sido más difícil. 
    
    Doy las gracias al apoyo de mis tutores, Alfonso y Antonio, por sus consejos y ayudas siempre que lo necesitaba.
    
    A todo el grupo de investigación, por facilitarme un lugar de trabajo disponible; así como ofrecerme esta beca con la que pude costearme la habitación en aquel piso de Madrid. También por darme la posibilidad de volver a \textit{la mio tierrina} y terminar el TFM a distancia.
    
    Y por supuesto gracias a \textit{ti}, por leerme.

    %	RESUMEN
    \section*{RESUMEN}
    Extensión máxima de una página %TODO: hacer resumen/abstract


    %	SUMMARY
    \section*{SUMMARY}
    Extensión máxima de una página


    %	ÍNDICE
    \tableofcontents % indice de contenidos



    %	INDICE DE FIGURAS Y TABLAS
    \listoffigures

    \listoftables
    
    \listofalgorithms
	\afterpage{\blankpage}
	
    %	CAPTULOS DEL TRABAJO FIN DE MÁSTER
    \newpage
    \pagenumbering{arabic}
%	\setlength{\parindent}{0em}
%	\setlength{\parskip}{\baselineskip}%
	
    %
    % EN ESTE DOCUMENTO ESTÁ EL RESTO DE LA PLANTILLA
    % \section{INFORMACIÓN SOBRE EL TFM}

\subsection{Asignación de Trabajo Fin de Máster}
\noindent El proceso de asignacin de Trabajo Fin de Máster, aprobado por la CAMIA en su novena reunión ordinaria de 15/12/2011, es el siguiente:
\begin{enumerate}
	\item Los alumnos pueden contactar con los profesores del MUIA y acordar el tema de su Trabajo Fin de Mster.
	\item A travs de una aplicacin informtica desarrollada por el DIA (manual de usuario), los alumnos pueden introducir sus preferencias sobre las propuestas de TFM que anualmente realizan por los profesores del MUIA (entre Diciembre y Enero). Identifican, si as lo desean, en orden hasta un mximo de 5 propuestas que ms le atraigan.
	\item En el caso de que no les atraiga ninguna oferta, o no se le haya asignado ninguna de las seleccionadas (varios alumnos pueden seleccionar la misma propuesta), el alumno deber realizar una propuesta, encuadrndola en una de las materias del MUIA e indicando hasta tres profesores de la misma que puedan ejercer de directores.
\end{enumerate}

En el siguiente enlace (http://www.dia.fi.upm.es/grupos-investigacion) se dispone de un listado de los grupos de investigacin, con una descripcin breve de los mismos y enlaces a sus correspondientes pginas web.

Los alumnos pueden identificar a partir de la informacin proporcionada por los grupos de investigacn la lnea en que basar el desarrollo de su TFM e incluso de una posterior Tesis Doctoral.

Se permitide un TFM por dos profesores, previa solicitud y justificac de la misma a la CAMIA, siendo obligatorio que al menos uno de los dos profesores forme parte del profesorado del ter.

La  asigne Trabajos Fin de Master se encuentra disponible en la web.



\subsection{Tribunal evaluador}
\noindent Se constituiun tribunal para cada defensa de TFM. El director del TFM formarte del tribunal y eleg miembros restantes, debiendo ser:
\begin{itemize}
	\item Uno de ellos, un profesor del MUIA de la materia del TFM.
	\item  El otro, un profesor del MUIA de la materia del TFM o de una mater.
\end{itemize}

En caso de codireccionesl visto bueno.


\subsection{Proceso administrativo de defensa del TFM}
\noindent La Figura \label{fig:proceso} muestra el proceso completo desde la asignacer (TFM) hasta su defensa.

\begin{figure}[h]
	\centering
	\includegraphics[width=0.65\textwidth]{recursos/Proceso}
	\caption{Proceso desde la asignahasta la defensa del TFM}
	\label{fig:proceso}
\end{figure}


El TFM {\bf puede matricularse en cualquier momento a lo largo del curmico} en la Secreta alumnos (ETSIInf), donde se generarcarta de pago.

El tiempo que puede transcurrir entre la matriculla defensa del TFM no  limitado (salvo los 7  naturales de antelaccircunscrito al mismo curso .

Es necesario tener en cuenta que al hacer el pago, el importe de la  que llegue a la Universidad tiene que coincidir exactamente con el de la carta de pago. Si no hay coincidencia no se  defender hasta que esa cantidad coincidiera, por lo que las comisiones bancarias o cargos correspondientes transferencias desde el extranjero, cambios de divisas, etc. los tiene que asumir el alumno. Una vez realizado el pago se debe entregar en el Centro de Postgrado (o bien enviarlos mediante un mail a centro.postgrado@fi.upm.es) lo siguiente:

\begin{itemize}
	\item el resguardo de la transferencia.
	\item y los datos siguientes:     
	\begin{itemize}
		\item Nombre y apellidos de la persona matriculada.
		\item Nombre del Master.
		\item Fecha de pago.
		\item Cantidad transferida.
		\item Cuenta desde la que se transfiere la cantidad.
	\end{itemize}
\end{itemize}

{\bf Nota:} El alumno debe tener en cuenta que si no  matriculado de ninguna asignatura en el MUIA pierde su  oficial con la UPM y no puede optar a becas oficiales y no oficiales,  Por ello, recomendamos a los alumnos que  tengan pendiente el TFM la matriculen al principio del semestre correspondiente para mantener la  con la universidad.

Las {\bf defensas} de los TFM se  realizar a lo largo de todo el curso

Una vez matriculada el TFM, el alumno  con una {\bf  de 7  naturales} la fecha y tribunal de la defensa, mediante la instancia correspondiente, en la {\bf  que se debe entregar es la siguiente:
	
	\begin{itemize}
		\item Instancia con tribunal y fecha de la defensa del TFM y  del director/es.
		\item Copia de la carta de pago de  del TFM.
		\item Instancia de  de cara a que el TFM pueda ser publicada en el archivo digital de la UPM.
	\end{itemize}
	a de la misma un ejemplar del TFM en el formato prescrito en formato  (pdf).
	
	El secretario del tribunal l encargado de {\bf reservar hemiciclo} para la  del acto de defensa del TFM y de {\bf recoger y entregar las actas} de la defensa en la  de Postgrado de la ETSIInf.
	
	\subsection{Acto de defensa del TFM}
	\noindent La {\bf lengua} tanto de la memoria del TFM, como de la defensa del mismo ante el tribunal,  ser el castellano o el .
	
	El secretario del tribunal s de Postgrado de la ETSIInf.
	
	La {\bf defensa del TFM}  oral sobre el misma por parte del alumno durante un {\bf tiempo  de 20 minutos y  de 20 minutos.
		
		El tribunal  los siguientes aspectos a la hora de evaluar el TFM:
		\begin{itemize}
			\item El alumno {\bf conoce}  de Inteligencia Artificial que le permiten abordar y solucionar problemas de .
			\item El alumno {\bf aplica}  existentes de la Inteligencia Artificial para la  de un problema.
			\item El alumno {\bf crea} alguna  de  de la Inteligencia Artificial.
			\item El alumno {\bf crea y difunde}  aceptados) los resultados de la TFM en una revista o congreso (nacional o internacional) con  por pares.
		\end{itemize}
		
		
		
		\subsection{Confidencialidad}
		\noindent En el caso de que el alumno desee la confidencialidad sobre su TFM,  solicitarlo mediante la correspondiente instancia disponible en la web que se  con una copia impresa del TFM y se  por Registro en  de Alumnos.
		
		
		\subsection{Concesión de Matriculas de Honor}
		\noindent Para proponer la  de Matrícula de Honor, se  en cuenta los criterios ya aprobados en la CAMIA de 15/12/2012: El alumno {\bf crea y difunde}  o  (nacional o internacional) con  por pares.
		
		formada por 3 profesores del Master Universitario en Inteligencia Artificial (MUIA) para la  aquellos profesores que hayan tutorizado alguna de los TFM propuesta para MH.
		
		Una vez finalizada la defensa de todos los trabajos de fin de  lugar en el mes de Julio. 
		
		La  solamente  en cuenta los TFM que hayan sido propuestas para MH por los respectivos tribunales.
		
		El tribunal otra convocatoria posterior.
		
		Si hubiese un  de alumnos matriculados (de conformidad con lo dispuesto en el Real Decreto 1125/2003, de 5 de septiembre), se  en cuenta las siguientes recomendaciones:
		
		\begin{itemize}
			\item Se  de honor obtenidas por el alumno en asignaturas del master.
		\end{itemize}
		
		
		\section{TABLAS, FIGURAS, EXPRESIONES MATEMÁTICAS Y ALGORITMOS}
		
		\subsection{Figuras}
		
		Las Figuras \ref{fig:Bernoulli1} y \ref{fig:violin_besa_escenario4} muestran ejemplos de  insertar figuras en el TFM.
		\begin{figure*}[htb]
			\centering
			\begin{subfigure}{0.5\textwidth}
				\includegraphics[width=\textwidth]{recursos/Figure1a}
				\caption{Mean cumulative regret along trials}
				\label{fig:Bernoulli1_semilog}
			\end{subfigure}
			\begin{subfigure}{0.5\textwidth}
				\includegraphics[width=\textwidth]{recursos/Figure1b}
				\caption{Multiple violinplot}
				\label{fig:Bernoulli1_boxplot}
			\end{subfigure}
			\caption{Comparative of the policies for scenario 1}
			\label{fig:Bernoulli1}
		\end{figure*}
		
		\begin{figure*}
			\centering
			\includegraphics[width=0.5\textwidth]{recursos/Figure2}
			\caption{Violinplot fot BESA in scenario 4}
			\label{fig:violin_besa_escenario4}
		\end{figure*}
		
		
		
		\subsection{Expresiones matemáticas}
		A continuación, se muestran algunos ejemplos de expresiones matemáticas:
		\begin{equation}
		\mu^*\times 25000-\frac{1}{1000}\sum_{r=1}^{1000}\sum_{i=1}^{K}\sum_{j=1}^{25000}\mu_i\times X_{i,j}^r.
		\end{equation}
		
		\begin{equation}
		\mu_{\widetilde{A}}(x)=\left\{ \begin{array}{cc}
		\frac{x-a_{1}}{a_{2}-a_{1}} & if\; a_{1}\leq x\leq a_{2}\\
		1 & if\; a_{2}\leq x\leq a_{3}\\
		\frac{x-a_{4}}{a_{3}-a_{4}} & if\; a_{3}\leq x\leq a_{4}\\
		0 & otherwise
		\end{array}\right. .
		\end{equation}
		
		
		\begin{equation}
		\begin{tabular}{ll}
		$\widetilde{DD}(A_{1},A_{4})$ & $=\widetilde{DD}(A_{1},A_{4}|P_{1})\oplus 
		\widetilde{DD}(A_{1},A_{4}|P_{2})$ \\ 
		$=[\widetilde{dd}(A_{1},A_{2})\otimes \widetilde{dd}(A_{2},A_{4})]\oplus \lbrack \widetilde{dd}(A_{1},A_{3})\otimes \widetilde{%
			dd}(A_{3},A_{4})].$%
		\end{tabular}%
		\end{equation}
		
		
		
		\begin{itemize}
			\item Si$\;{max} \{(a_{4}-a_{1}),(b_{4}-b_{1})\}\neq 0$, entonces
			\begin{eqnarray*}
				{\small S(}\widetilde{A}{\small ,}\widetilde{B}{\small )} &{\small =}&%
				\left. {\small 1-(1-\alpha -\beta })\times \left ( {\small 1-}\frac{\int_{0}^{1}%
					{\small \mu }_{\widetilde{A}\cap \widetilde{B}}{\small (x)dx}}{\int_{0}^{1}%
					{\small \mu }_{\widetilde{A}\cup \widetilde{B}}{\small (x)dx}}\right)
				\right.  \\
				&&\left. -{\small \alpha } \frac{\sum {\small \mid a}_{i}{\small -b}_{i}%
					{\small \mid }}{{\small 4}}-{\small \beta }\frac{{\small d[(X}_{\widetilde{A}%
					}{\small ,Y}_{\widetilde{A}}{\small ),(X}_{\widetilde{B}}{\small ,Y}_{%
						\widetilde{B}}{\small )]}}{{\small M}}\right., 
			\end{eqnarray*}
			
			\item En caso contrario,%
			\begin{eqnarray*}
				{\small S(}\widetilde{A}{\small ,}\widetilde{B}{\small )} &{\small =}&%
				\left. {\small 1-}%
				\left( \frac{{\small 1-\alpha -\beta }}{{\small 2}}{\small +\alpha } \right) \times
				\frac{\sum {\small \mid a}_{i}{\small -b}_{i}{\small \mid }}{{\small 4}}%
				{\small -}\right.  \\
				&&\left. {\small -}\left( \frac{{\small 1-\alpha -\beta }}{{\small 2}}%
				{\small +\beta }\right)\times \frac{{\small d[(X}_{\widetilde{A}}{\small ,Y}_{%
						\widetilde{A}}{\small ),(X}_{\widetilde{B}}{\small ,Y}_{\widetilde{B}}%
					{\small )]}}{{\small M}}\right., 
			\end{eqnarray*}
		\end{itemize}
		donde $\alpha +\beta <1$, $\mu _{\widetilde{\chi }}$ es la funcion de pertenencia de $\widetilde{\chi}$, 
		\begin{equation}
		M=\underset{[0,1]\times[0,\frac{1}{2}]}{max}\{d((x,y),(x^{\prime },y^{\prime }))\}\text{,} 
		\end{equation}%
		\begin{equation*}
		\mu _{\widetilde{A}\cap \widetilde{B}}(x)=\underset{0\leq x\leq 1}{min}%
		\{\mu _{\widetilde{A}}(x),\mu _{\widetilde{B}}(x)\} ,
		\;\;\; \mu _{\widetilde{A}\cup \widetilde{B}}(x)=\underset{0\leq x\leq 1}{max}%
		\{\mu _{\widetilde{A}}(x),\mu _{\widetilde{B}}(x)\}.
		\end{equation*}%
		
		\subsection{Algoritmos}
		
		El Algoritmo \ref{getDelay} ilustra la forma que debe adoptarse. 
		\begin{algorithm}[h]
			%\begin{algorithmic}
			{\bf  Data:} ($t_0$ = instante en el que se genera el retardo)
			\medskip
			
			\hspace{0.5em} {\bf if} $(update\_architecture==1)$ {\bf then} 
			
			\hspace{1.5em} {\bf if} $(delay\_scenario==1)$ {\bf then} delay$=C$
			
			\hspace{1.5em} {\bf else} 
			
			\hspace{2.5em} {\bf if} $(reward\_scenario==1)$ {\bf then} 
			
			\hspace{3.5em} delay $\leftarrow [0,300]$-trunc\_Exp($\lambda=1/80$)
			
			\hspace{2.5em} {\bf else} 
			
			\hspace{3.5em} delay $\leftarrow [0,480]$-trunc\_Exp($\lambda=1/150$)
			
			\hspace{2.5em} {\bf end if}
			
			\hspace{1.5em} {\bf end if}
			
			\hspace{0.5em} {\bf else} (arquitectura en modo batch)
			
			\hspace{1.5em} delay= difference(24:00, $t_0$)
			
			\hspace{0.5em} {\bf end if}
			
			\hspace{0.5em}  {\bf return} delay
			
			{\bf end} 
			\caption{$getDelay(t_0)$}
			\label{getDelay}
		\end{algorithm}
		
		
		\subsection{Tablas}
		Las Tablas \ref{table:results45} y \ref{table:risk} muestran el formato de tabla a utilizar.
		
		\begin{table}[htb]
			\centering
			\caption{Mean cumulative regrets and standard deviations}
			\label{table:results45}
			\begin{tabular}{llllll}
				\hline
				& \multicolumn{2}{c}{\small Truncated Poisson} &  & \multicolumn{2}{c}{\small Truncated Exponential} \\ 
				\cline{2-3}\cline{5-6}\cline{5-6}
				& {\small Mean} & ${\small \sigma}$ &  &  {\small Mean} & ${\small \sigma}$\\ \hline
				{\small UCB}      & {\small 2632.65} & {\small 246.03}  &  & {\small 1295.79} & {\small 514.03}   \\
				{\small DMED+}            & {\small 978.56} & {\small 225.24}  &  & {\bf \small645.70} & {\small 493.8}   \\
				{\small KL-UCB}   & {\small 1817.4} & {\small 236.57}  &  & {\small 1219.98} & {\small 510.69}   \\ 
				{\small KL-UCB poisson}    & {\bf \small314.99*} & {\small 201.79}  &  & {\small -} & {\small -}   \\
				{\small KL-UCB exp}    & {\small -} & {\small -}  &  & {\small 786.30} & {\small 498.16}   \\
				{\small KL-UCB+}    & {\small 1190.64} & {\small 225.82}  &  & {\small 813.45} & {\small 494.59}   \\
				{\small BESA}      & {\small 2015.73} & {\small 3561.5}  &  & {\small 755.87} & {\small 2323.22}   \\
				{\small PR-1}            & {\small 1314.9} & {\small 234.25}  &  & {\small 660.64} & {\small 492.37}   \\ 
				{\small PR-2 (TS)}  & ${\bf 917.67}$ & {\small222.79}  &  & {\bf \small630.38} & {\small487.01} \\
				{\small PR-3}  & ${\bf 736.6}$ & {\small210.96}  &  & {\bf \small565.79*} & {\small480.99} \\
				\hline
			\end{tabular}
		\end{table}
		
		\begin{table}[htb]
			\centering
			\caption{Risks to $A_5$ after the implementation of the selected safeguards }
			\label{table:risk}
			\begin{tabular}{cccc}
				\hline
				\noalign{\smallskip} 
				{\scriptsize{Threat}}& {\scriptsize{Confidentiality}} & {\scriptsize{Integrity}} & {\scriptsize{Authenticity}}\tabularnewline
				\hline  
				{\scriptsize{$T_{1}^{1}$}} & \scriptsize{(16.9, 161.72, 936.2, 3681.5)} & \scriptsize{(32.70, 239.7, 1295.6, 5197.4)} & \scriptsize{(25.1, 198.6, 1576.7, 5777.1)}\\
				{\scriptsize{$T_{1}^{2}$}} & \scriptsize{(0, 49.6, 458.1, 1791.2)} & \scriptsize{(0, 29.7, 289.7, 1397.1)} & \scriptsize{(0, 24.6, 352.6, 1552.9)}\\
				{\scriptsize{$T_{2}^{2}$}} & \scriptsize{(0, 49.6, 458.1, 1791.2)} & \scriptsize{(0, 29.7, 289.7, 1397.1)} & \scriptsize{(76, 379.3, 2074.3, 5588.4)}\\
				{\scriptsize{$T_{1}^{3}$}} & \scriptsize{(12.2, 110.5, 647.2, 2465.6)} & \scriptsize{(21.9, 147.3, 744.3, 2958.7)} & \scriptsize{(6.8, 58.5, 487.1, 1923.2)}\\ 
				{\scriptsize{$T_{1}^{4}$}} & \scriptsize{(34.8, 245.5, 1176.8, 3793.2)} & \scriptsize{(62.7, 327.4, 1353.3, 4551.9)} & \scriptsize{(19.5, 129.9, 885.7, 2958.7)}\\
				\hline 
			\end{tabular}
		\end{table}
		
		
		\section{CONTENIDOS DEL TFM}
		Durante la resultante de la tesis satisface los deseos o necesidades del cliente (real, potencial o ficticio).
		
		{\bf Conclusiones:}
		Establecer las conclusiones del trabajo  actuales aplicadas al problema, planteando leneas de I+D+i realistas y capaces de superarlos.
		
		\section{CONCLUSIONES Y LENEAS FUTURAS DE TRABAJO}
		Establecer las conclusiones del trabajo apoyándose fundamentalmente en los datos y observaciones obtenidas durante su desarrollo. Discutir que medios, cauces, etapas y tecnologías hartan falta (si procede) para llevar a cabo una implantación real de los resultados.
		Discutir los limites de las tecnologías actuales aplicadas al problema, planteando leneas de I+D+i realistas y capaces de superarlos.
		
		\section{SOBRE LAS REFERENCIAS}
		
		La bibliográfica o referencias deben aparecer siempre al final de la tesis, incluso en aquellos casos donde se hayan utilizado notas finales. La bibliográfica debe incluir los materiales utilizados, incluida la edición, para que la cita pueda ser fácilmente verificada. 
		
		\bigskip
		{\bf Citar dentro del texto:}
		
		Las fuentes consultadas se describen brevemente dentro del texto y estas citas cortas se amplían en una lista de referencias final, en la que se ofrece la información bibliográfica completa. 
		
		La cita dentro del texto es una referencia corta que permite identificar la publicación de donde se ha extraído una frase o parafraseado una idea, e indica la localización precisa dentro de la publicación fuente. Esta cita informa del apellido del autor, la fecha de publicación y la pagina (o paginas) y se redacta de la forma que puede verse a través de los siguientes ejemplos:
		
		Cuando se citan las palabras exactas del autor deben presentarse entre comillas e indicarse, tras el apellido del autor y, entre paréntesis, la fecha de publicación de la obra citada, seguida de la/s pagina/s.
		
		Si lo que se reproduce es la idea de un autor (no sus palabras exactas) no se ponrse; debe indicarse siempre con puntos suspensivos entre corchetes [...]
		
		Ejemplos de como citar una referencia en el texto son los siguientes \cite{Ashtiani2014} o \cite{Ashtiani2014,Mateos2009,Vicente2016}.
		
		
		\bigskip
		{\bf Como ordenar las referencias:}
		\begin{enumerate}
			\item Las referencias bibliográficas deben presentarse ordenadas alfabéticamente por el apellido del autor, o del primer autor en caso de que sean varios.
			\item Si un autor tiene varias obras se orde
			\item Si son trabajos de un autor en colabora de publicación. Las publicaciones individuales se colocan antes de las obras en colaboración.
		\end{enumerate}
		
		\bigskip
		{\bf Como citar un articulo de revista}
		
		Un articulo de revista, siguiendo las normas de la APA, se cita de acuerdo con el siguiente esquema general:
		Apellido(s), Iniciales del nombre o nombres. (Aulo.
		
		\bigskip
		{\bf Cmo citar una monografista/libro}
		
		Las monografistas, siguiendo las normas de la APA, se citan de acuerdo con el siguiente esquema general:
		Apellido(s), n cursiva.
		
		\bigskip
		{\bf Como citar un capitulo de un libro}
		
		Los c Editorial.
		
		\bigskip
		{\bf Cmo citar un acta de un congreso}
		
		Apellido(s), Iniciales del nombre o nombres. (A). Ttulo del trabajo. En A. A. Apellido(s) Editor A, B. B. Apellido(s) Editor B, y C. Apellido(s) Editor C (Eds. o Comps. et.), Nombre de los proceedings en cursiva (pp. xxx-xxx). Lugar de publicaci: Editorial.
		
		\bigskip
		{\bf Como citar tesis doctorales, trabajos fin de míster o proyectos fin de carrera}
		
		Apellido(s), Nombre. (Aro). Titulo de la obra en cursiva. (Tesis doctoral). Institución a académica en la que se presenta. Lugar.
		
		\bigskip
		{\bf Como citar un recurso de Internet}
		
		Los recursos disponibles en Internet pueden presentar una tipografía muy variada: revistas, , portales, bases de datos... Por ello, es muy difícil dar una pauta general que sirva para cualquier tipo de recurso.
		Como mínimo una referencia de Internet debe tener los siguientes datos:
		\begin{enumerate}
			\item Titulo y autores del documento.
			\item Fecha en que se )
		\end{enumerate}
		
		Veamos, a .
		
		Monografistas:
		Se emplea la misma forma de cita que para las monografistas en versión impresa. Debe agregar la URL y la fecha en que se consulta el documento
		
		de revistas:
		Se emplea la misma forma de cita que para los artículos de revista en  impresa. Debe agregar la URL y la fecha en que se  el documento.
		
		de revistas  que se encuentran en una base de datos:
		Se emplea la misma forma de cita que para los  de revista en  impresa, pero debe  el nombre de la base datos, la fecha en que se  el documento.
    %
    %

    \graphicspath{{capitulos/Capitulo1-Introduccion/recursos/}}

\section{Introducción y objetivos}

En éste capítulo describiremos el contexto principal y las hipótesis iniciales de las que parte el proyecto, así como una ligera introducción al problema bajo estudio en el presente proyecto de fin de máster.

\begin{figure}
	\centering
	\includegraphics[width=0.7\linewidth]{Figure1a}
	\caption{ figura de test}
	\label{fig:figure1a}
\end{figure}

    \newpage

    \graphicspath{{capitulos/Capitulo2-Definicion-del-problema/recursos/}}

\section{Definición del problema} \label{apartado:2}

Tal y como se ha introducido antes, el proyecto ABACO pretende automatizar el proceso de creación de un horario de trabajo para
los distintos controladores del espacio aéreo de forma que dada una sectorización del espacio aéreo, todos los sectores puedan ser
controlados.

\subsection{Sectores y sectorización}
En primera lugar, explicaremos brevemente cómo se divide el espacio aéreo del territorio español, cuyo organismo encargado de su gestión es AENA. Si bien la realidad es muy compleja, aquí unicamente describiremos una simplificación de la misma, omitiéndose detalles técnicos de aviación que no son necesarios para la implementación del sistema.
\\

El espacio aéreo mundial se encuentra dividido en \textit{FIR}s (\textit{Flight Information Region}), áreas del territorio sobre las que se mueven los diferentes aviones de cada compañía aérea de cada país, en la \refcruzada{Figura}{fig:fireuropa} puede verse graficamente los límites de cada región. En el caso de España, podemos ver que tiene control sobre 3 \textit{FIR}s: el de Barcelona, el de Madrid y el de Canarias, sin embargo, a nivel nacional, existen algunas subdivisiones denominadas \textit{Dependencias} (ya que dependen del \textit{FIR} en el que se encuentren), que permiten una mejor gestión del territorio:
\begin{itemize}
	\item Barcelona RutaE
	\item Barcelona RutaW
	\item Barcelona TMA ESTE
	\item Barcelona TMA NORTE
	\item Barcelona TMA OESTE
	\item Canarias ACC App
	\item Canarias ACC Ruta
	\item Madrid Ruta 1
	\item Madrid Ruta 2
	\item Madrid TMA NORTE
	\item Madrid TMA SUR
	\item Malaga App
	\item Palma TACC
	\item Sevilla TACC
	\item  Valencia TACC TMA
\end{itemize}

Algunos de ellos aparecerán en los casos de prueba del sistema del \refcruzada{Apartado}{apartado:5}
\begin{figure}
	\centering
	\includegraphics[width=1\linewidth]{capitulos/Capitulo2-Definicion-del-problema/recursos/FIR_europa}
	\caption{FIRs del la zona europea. Fuente: EUROCONTROL}
	\label{fig:fireuropa}
\end{figure}



    \newpage

   	%! Suppress = LineBreak
%! Suppress = LabelConvention
\graphicspath{{capitulos/Capitulo3-Metodologia-propuesta/recursos/}}

\section{Metodología propuesta} \label{capitulo:3}
En este capítulo se describe en detalle la metodología propuesta para resolver el problema planteado en la \hyperref[capitulo:2]{sección anterior}, entendiendo por metodología el conjunto de decisiones previas a la implementación tomadas con el fin de plantear una forma de resolver dicho problema.

La hipótesis de partida (\hyperref[H2]{H2}) plantea como punto de comienzo del proyecto el uso de una metaheurística con el fin de optimizar todos los parámetros del sistema, pero hemos de definir dichos parámetros antes de poder definir la metaheurística.

Para comenzar con el planteamiento de la metodología, podemos comenzar desde el punto de vista de la ingeniería: verlo como un sistema de caja negra que recibe una entrada y una salida.
El sistema debe poder corregir la planificación de controladores de ese día, por lo tanto, es claro que la entrada será esa planificación. Recordemos que el sistema \legacy{} será empleado por el personal del aeropuerto para realizar la planificación de forma automatizada, por lo que la entrada del sistema deberá tener un formato común con la salida del sistema \legacy{}.
Respecto a la salida, deberá ser de un formato comprensible por el personal gerente de los puestos de control.

Por último, resta detallar la parte más importante: la \textit{caja negra} propiamente dicha. Pues bien, como hemos dicho antes, en primer lugar el sistema recibirá una solución inicial, que deberá ser tratada de acuerdo con las contingencias habidas. Por ejemplo, ocurre un cambio de sectorización en mitad del turno que no estaba planificado. 
A partir del momento en el que se cierra debemos eliminar todas las apariciones de los sectores que se cierran y añadir los que se abren, pero no antes de dicho momento (más detalles en la \autoref{sec:3:inicializacion-soluciones}). 
Llamaremos al momento en el que sucede la incidencia como momento actual para simplificar, aunque lo habitual es que el sistema sea ejecutado varios minutos antes de que suceda la incidencia.

Una vez tratada la solución inicial, la metaheurística ya podrá dar comienzo sobre ella, buscando diferentes 
planificaciones alternativas (soluciones) y dando como resultado la mejor. 

%Con todo, el sistema tiene cuatro módulos:
%\begin{itemize}
%	\item Módulo de lectura de datos: lleva a cabo las tareas de lectura e inicialización de estructuras de datos.
%	\item Módulo de inicialización: inicializa la solución inicial de acuerdo con la(s) contingencias recibidas del módulo anterior
%	\item Módulo de búsqueda: lleva a cabo la búsqueda de una solución factible al problema.
%	\item Módulo de entrega de datos: lleva a cabo las tareas de escritura y trazabilidad de las soluciones.
%\end{itemize}

Sin entrar en detalles de implementación (para ello véase el \autoref{capitulo:4}), el sistema tiene, claramente, dos 
\textit{Fases}:
\begin{enumerate}[label={}]
	\item \label{Fase 1} Fase 1: Tratamiento de la solución
	\item \label{Fase 2} Fase 2: Resolución del problema
\end{enumerate}

En adelante, aludiremos a la fase del sistema que comprende la inicialización de la solución de entrada en función de las necesidades del caso concreto de incidencia que se produzca como \faseuno{} o \textit{Fase de Inicialización}. 
Mientras que la \fasedos{} o simplemente \textit{Metaheurística}, será la fase del sistema en la que se resolverá el problema propiamente dicho de acuerdo con las pautas establecidas en forma de restricciones y puntuaciones sobre la metaheurística.
En las próximas secciones definiremos cada una de estas Fases en detalle.

\subsection{Representación de las soluciones}
\label{sec:3:representacion-soluciones}

Antes de entrar en detalle, es importante explicar cómo se han representado las soluciones. Recordemos que estamos representado planificaciones, es decir una relación matricial de sectores con trabajadores a lo largo del tiempo que dura un turno. De forma que, dada una lista de controladores, en cada instante de tiempo tendremos un sector asignado, así como el tipo de puesto (planificador o ejecutivo).

Las filas de la matriz serán por tanto los controladores que tengamos a nuestra disposición así como los que añadamos dinámicamente para facilitar la inicialización y que serán eliminados en la \fasedos{}, mientras que las columnas de la matriz representarán el tiempo (véase \autoref{fig:3:ejemplo-distribucion-inicial}) desde el inicio del turno hasta el final del turno. Por lo que el tamaño de cada una dependerá de la instancia concreta del problema que estemos 
resolviendo.

El tiempo es una variable continua, que nos permitiría conocer el sector que controla un trabajador para un momento exacto del tiempo como por ejemplo las 8:29:17 (horas, minutos, segundos), una precisión del todo innecesaria en este problema, pero también difícil de representarlo en este tipo de problemas de \textit{timetabling}. Para poder 
representar el tiempo, debemos convertir la variable continua en discreta mediante el proceso denominado 
discretización: renunciamos a la precisión del tiempo fragmentándolo en intervalos de tiempo uniformes que llamaremos \textit{slots}, por ejemplo de 5 minutos cada uno.

\begin{figure}[htbp]
	\begin{subfigure}{\linewidth}
		\centering
		\includegraphics[width=\linewidth]{tiempo-continuo}
		\caption{Tiempo continuo}
		\label{fig:timepo-continuo}
	\end{subfigure}
	
	\begin{subfigure}{\linewidth}
		\centering
		\includegraphics[width=\linewidth]{tiempo-disccreto}
		\caption{Tiempo discreto con \textit{slots} de 5 minutos}
		\label{fig:timepo-disccreto}
	\end{subfigure}
	
	\caption{Ilustración de la discretización del tiempo}
\end{figure}

Esta discretización nos hace perder precisión, por lo que el tamaño del slot deberá ser el adecuado para no perder capacidad de representación en nuestras soluciones y por ende limitar espacio de búsqueda en exceso, lo cual podría desembocar en que una buena solución no pueda ser representada y por lo tanto no será contemplada por el sistema de búsqueda así que nunca se dará como solución.
En nuestro caso, hemos elegido un tamaño de slot de 5 minutos debido a que se trata del máximo común divisor de todas las restricciones numéricas del dominio del problema (véase la \autoref{sec:4:RD}). Los expertos de \gls{CRIDA} están satisfechos con este nivel de detalle.

En las representaciones realizadas (véase como ejemplo la \autoref{fig:3:ejemplo-distribucion-inicial}) se utilizan identificadores de tres letras en lugar del nombre completo del sector para abreviar y mantener el número de caracteres constante.
Se han añadido también colores para una mejor visualización.
Las letras en mayúscula (AAA-ZZZ) representan un trabajo en puesto de ejecutivo, mientras que las letras minúsculas (aaa-zzz) indican un trabajo en puesto planificador. Los descansos se representan mediante unos (111).
Hemos agrupado slots contiguos tanto de trabajo como de descanso idénticos, de manera que visualmente sea más cómodo de entender. Para que las soluciones aquí presentadas tengan validez final, deberíamos añadir indicadores de las horas de los cambios de puesto, sin embargo, para este documento esto no es realmente importante por lo que podemos omitirlo.


%\NOTE{ AÑADIR ESTRUCTURA PIRAMIDAL DE LAS SOLUCIONES (de cara a referenciarlo en el fitness \autoref{apartado:adaptacion-fitness})} % PUESTO EN EL APARTADO DE FITNESS

\begin{figure}[htbp]
	\centering
	\includegraphics[width=\linewidth]{Ejemplo-distribucion-inicial}
	\caption[Ejemplo de una solución inicial]{Ejemplo de una posible solución inicial. Constituida mediante el uso 
		de plantillas.}
	\label{fig:3:ejemplo-distribucion-inicial}
\end{figure}

\subsection{Fase 1: Inicialización de Soluciones} \label{sec:3:inicializacion-soluciones}

La fase de inicialización toma como entrada la planificación inicial: aquella planificada para el día y que ya no tiene validez debido a la incidencia; junto a los datos relativos a la incidencia, que son:

\begin{itemize}
	\item Hora a la que se produce la incidencia.
	\item Tipo de incidencia.
	\item Si la incidencia es por un cambio imprevisto de sectorización, la nueva sectorización.
	\item Si la incidencia es por una baja de un trabajador, hora de la baja y los datos del trabajador y, opcionalmente, hora del alta y datos del trabajador (puede ser el mismo u otro que no forme parte del turno inicial).
\end{itemize}

\begin{figure}[htbp]
	\centering
	\includegraphics[width=\linewidth]{Esquema-Fase-1-extendido}
	\caption{Diagrama de flujo del funcionamiento de la Fase 1}
	\label{fig:3:esquema-fase-1}
\end{figure}

Con esos datos, la \faseuno{} deberá convertir la planificación inválida en una \textit{solución inicial}, que 
emplearemos como punto de partida para el sistema de búsqueda inteligente que es la \fasedos{}. Para ello distinguimos dos tipos de tareas, las relativas a la incidencia por cambio de sector (pasos 1 y 2) y las relativas a las bajas y altas de los trabajadores (pasos 3 y 4). Los pasos pueden verse esquemáticamente en la \autoref{fig:3:esquema-fase-1}.
Adicionalmente, un quinto paso fue planteado para poder facilitar la tarea de la \fasedos{}, que consistía en reducir el número de controladores añadidos artificialmente en los pasos anteriores moviendo, heurísticamente, carga de trabajo a otros controladores que la soporten. Finalmente no fue implementada y fue añadida como trabajo futuro.

En la figura \autoref{fig:3:ejemplo-distribucion-inicial} se muestra cómo sería una posible planificación inicial. En este caso ha sido creada en base a \textit{plantillas} o \textit{estadillos}, que es el método empleado habitualmente por el personal para crear la planificación. Consiste en la repetición de un patrón conformado por tres 
controladores para un sector, en el que se suceden trabajo en puesto planificador, trabajo en puesto ejecutivo y descanso con un desfase en incremento para cada controlador, de forma que en cada instante de tiempo (imagínese una línea transversal) habrá un controlador en puesto ejecutivo, otro en planificador y otro descansado (véase  la \autoref{fig:3:plantilla-3x1}). \gls{CRIDA} sabe que el uso de estas plantillas si bien no es lo más óptimo es lo más cómodo tanto para la creación manual de la planificación como de cara a no incumplir las restricciones de cada trabajador (ver requisito). %TODO: referencias!!).

\begin{figure}
	\centering
	\includegraphics[width=0.9\linewidth]{capitulos/Capitulo3-Metodologia-propuesta/recursos/Plantilla-3x1}
	\caption[Aspecto de una plantilla 3x1]{Aspecto de una plantilla 3x1. Las letras mayúsculas representan trabajo en 
		puesto ejecutivo y las minúsculas en planificador.}
	\label{fig:3:plantilla-3x1}
\end{figure}

En las representaciones realizadas, se utilizan identificadores de tres letras en lugar del nombre completo del sector para abreviar y mantener el número de caracteres contante. Se han añadido también colores para una mejor visualización.
\textbf{Las letras en mayúscula (AAA-ZZZ) representan un trabajo en puesto de ejecutivo, mientras que las letras minúsculas (aaa-zzz) indican un trabajo en puesto planificador. Los descansos se representan mediante unos (111)}.
Hemos agrupado slots contiguos tanto de trabajo como de descanso idénticos, de manera que visualmente sea más cómodo de entender. Para que las soluciones aquí presentadas tengan validez final, deberíamos añadir indicadores de las horas de los cambios de puesto, sin embargo para este documento esto no es realmente importante por lo que podemos omitirlo.

El tipo de plantilla descrito se le denomina $3\times1$ (3 controladores para 1 sector) pero existen otros tipos como $8\times3$ o $4\times1$, no obstante, la más importante para el sistema es la $3\times1$, que será empleada durante esta Fase.

\subsubsection{Paso 1: Eliminar sectores que se cierran}
El primer paso es el encargado de eliminar los sectores que se cierran. Pongamos por ejemplo que tenemos una 
sectorización 5A que pasa a ser una 6C en un momento dado, tal y como se ilustra en la 
\autoref{fig:3:ejemplo-cambio-sectorizacion}. Como ya hemos dicho previamente, nosotros partimos de una planificación inicial como la representada en la \autoref{fig:3:ejemplo-distribucion-inicial}, que con la nueva sectorización queda totalmente inutilizada, pues podemos ver sectores que ya no se encuentran abiertos a partir del punto de cambio.

\begin{figure}[htbp]
	\centering
	\includegraphics[width=\linewidth]{Ejemplo-cambio-sectorizacion}
	\caption[Ejemplo de cambio de sectorización]{Ejemplo de un posible cambio de sectorización en la Unidad de Control 
	de Barcelona. En color aquellos sectores comunes a ambas sectorizaciones}
	\label{fig:3:ejemplo-cambio-sectorizacion}
\end{figure}


Identificamos pues, el momento de la incidencia a las 10:30, sin embargo el \textit{momento actual} viene dado a como parte de la entrada. En este caso, han decidido que sea media hora antes de la incidencia, a las 10:00 horas, que equivale al slot número 30:
\[ 
	10 \,h-7 \,h \,30 \,min = 2 \,h \,30 \,min = \left(2 \, \cancel{h} \times \frac{60 \,min}{1 \,\cancel{h}}\right) 
	\,min + 30\,min = 150 
	\,min 
\]

\[
	150 \,\cancel{min} \times \frac{1\,slot}{5\,\cancel{min}} = 30\,slots
\]

Antes de dicha hora, la planificación no debe ser alterada en ningún caso, pues representa el pasado. En la 
\autoref{fig:3:ejemplo-distribucion-inicial} la hemos representado con una línea roja. 

En el resto de la planificación, debemos eliminar todos los sectores que no aparecen. Para ello eliminamos aquellos que se cierran: los que dentro de la sectorización antigua, no pertenezcan a la nueva (una resta de conjuntos) (es decir, los que siguen en color negro en el cuadro azul de la \autoref{fig:3:ejemplo-cambio-sectorizacion}). 
Adicionalmente, para obtener una mejor solución inicial y favorecer así a la búsqueda, en el momento de eliminar un sector de la sectorización inicial, tratamos de sustituirlo por uno de los sectores nuevos que se abren (los de la nueva sectorización, los del cuadro naranja en color negro) \textbf{de forma que sean afines de entre sí}, pues el controlador seguirá pudiendo controlarlo sin problemas de acreditación. %FIXME es esto cierto?

Para hacer más eficiente el recorrido del algoritmo, en lugar de ir slot a slot, podemos agruparlos mientras la sectorización sea la misma. Buscamos un sector afín a cada sector que se cierra y lo sustituimos en todas las apariciones dentro de ese conjunto de slots. Así sucesivamente para cada tramo de slots con la misma sectorización. 
La búsqueda del sector afín es un algoritmo voraz que obtiene el primer sector de entre los que se abren que sea afín al que se cierra, evitando repeticiones.

\begin{algorithm}[H]
%	\SetAlgoLined
	\DontPrintSemicolon
	\KwData{
		
		$Sectorizacion_{inicial} = $ conjunto de sectores de la sectorización inicial para cada slot.
			
		$Sectorizacion_{modificada} = $ conjunto de sectores de la sectorización modificada para cada slot.
	}
	\medskip
	
	\ForEach{conjunto de slots con la misma sectorización}{
		$cerrados \leftarrow { Sectorizacion_{modificada}[slot] \setminus Sectorizacion_{inicial}[slot] }$\;
		$abiertos \leftarrow { Sectorizacion_{inicial}[slot] \setminus Sectorizacion_{modificada}[slot] }$\;
		
		\ForEach{$sector_c \in cerrados$}{
			$afin \leftarrow$ buscarPrimerAfin($Sectorizacion_{modificada}[slot]$)\;
			
			\If{$\nexists{afin}$}{
				$\forall$ aparición de $sector_c$, sustituir por descansos $(111)$\;
			} \Else{
				$\forall$ aparición de $sector_c$, sustituir por $afin$\;
			}
		}
			
	}
	
	\caption{Heurística de inicialización: AFINIDADES}
\end{algorithm}


En nuestro ejemplo, solo tenemos un único sector que se cierra, LECBGOI, que sustituiremos, mediante el algoritmo, por el sector LECBG12, que es el primero afín de entre los nuevos abiertos. Mediante esta heurística, evitaremos tener que añadir plantillas (ver \autoref{apartado:3:paso-2}) de todos los sectores nuevos reutilizando los controladores ya existentes.

\subsubsection{Paso 2: Introducir plantillas de los nuevos sectores} \label{apartado:3:paso-2}

Partimos de la lista de sectores nuevos que se abren y que no han sido ya empleados en el paso anterior como sustituto de alguno de los que se cierran. Lo que haremos será añadir a la planificación una plantilla $3\times1$ como la de la 
\autoref{fig:3:plantilla-3x1} donde alternamos trabajo y descanso a tamaños iguales: el doble de trabajo (uno en cada puesto) por cada uno de descanso. Las plantillas pueden emplearse con cualquier escala de tiempo, manteniendo las proporciones, por ejemplo 2 horas de trabajo y una de descanso.
Para este proyecto se ha utilizado un tamaño de 6 slots de descanso (12 de trabajo) debido a que los resultados eran mejores empleando esta proporción en proyectos previos a la presente tesis, por lo que se ha mantenido dicha proporción.

Para cada uno de los sectores mencionados añadiremos una de estas plantillas, con 3 controladores adicionales inexistentes que emplearemos auxiliarmente y que trataremos de eliminar en la \fasedos{}. En caso de no ser posible eliminar todos los auxiliares, diremos que para resolver el problema actual necesitamos obligatoriamente de un controlador extra.

De esta forma, obtendremos una planificación en la que se han tenido en cuenta las contingencias relativas a los cambios de sectorización. Si la instancia concreta del problema incluye únicamente esta incidencia, la planificación de la \autoref{fig:3:ejemplo-distribucion-pasos-1-y-2} sería una \textit{solución inicial} preparada para emplear como entrada a la \fasedos{}.

\begin{figure} 
	\centering
	\includegraphics[width=\linewidth]{Ejemplo-distribucion-pasos-1-y-2}
	\caption[Planificación tras los pasos 1 y 2 de la Fase 1]{Planificación tras los pasos 1 y 2 de la \faseuno{} siguiendo el ejemplo de la \autoref{fig:3:ejemplo-distribucion-inicial}. Los controladores con identificador C0 son imaginarios, es decir no existen pero son necesarios en la inicialización y deberán ser eliminador en la \fasedos{}} 
	\label{fig:3:ejemplo-distribucion-pasos-1-y-2}
\end{figure}

\subsubsection{Paso 3 y 4: Dar de baja/alta a los controladores}
Estos pasos son ejecutados únicamente en caso de que haya una incidencia relacionada con el personal. Podría suceder que tras ponerse de baja repentinamente, otro controlador cubra ese puesto a lo largo de la jornada, por lo que tendremos que hacer dos modificaciones a la planificación:

El controlador que se da de baja dejará de trabajar ese día, sin embargo no podemos eliminarlo de la planificación puesto que en momentos previos al \textit{momento actual} si ha trabajado, y hemos de contabilizarlo como carga de trabajo. Emplearemos el carácter especial ``000'' para indicar que se trata de un slot en el que el controlador no está trabajando pero tampoco descansado, al que se deberá tratar de forma especial durante la ejecución de la \fasedos{}, pues no se podrá mover a otro controlador ni se le podrá asignar nuevo trabajo. 

Si ningún trabajador se reincorpora a su puesto de trabajo, emplearemos un controlador imaginario al que se le asignará toda la carga de trabajo que tenía el controlador de baja a partir del momento de la incidencia.

La \autoref{fig:3:ejemplo-distribucion-pasos-3-y-4} muestra un ejemplo donde el controlador 23 se ha puesto de baja a las 9:30 (slot 30) y no hay reincorporación (en caso de haberla el controlador $C_0$ tendría su identificador correspondiente). Nótese que se emplean los caracteres de fuera de turno (``000'') en todos los slots a partir de la incidencia en el caso del controlador de baja mientras que para el controlador imaginario (o el reincorporado según se aplique) sucede lo contrario: todos los slots previos a la incidencia tienen el mencionado carácter. 
%TODO Se tratan de slots inalterables en todos los casos, pero a la hora de contabilizarse el trabajo (ver Fitness) %TODO referencia no se han de computar como tiempo de descanso

\begin{figure}
	\centering
	\includegraphics[width=\linewidth]{Ejemplo-distribucion-pasos-3-y-4}
	\caption[Ejemplo de planificación tras los pasos 3 y 4 de la \faseuno{}]{Ejemplo de planificación tras los pasos 3 y 4 de la \faseuno{}. En este caso no ha habido reincorporaciones}
	\label{fig:3:ejemplo-distribucion-pasos-3-y-4}
\end{figure}
%
%
%


\subsection{Fase 2: Metaheurística de optimización multiobjetivo} \label{sec:3:metaheurística}
Se trata del núcleo principal del presente proyecto, pues trata de dar solución al problema en sí mediante un enfoque de metaheurísticas.
\\

Como ya se ha introducido previamente, estamos ante un problema de \textit{timetabling/scheduling} que son generalmente problemas complejos debido a su naturaleza combinatoria. 
Mateméticamente se dice que pertenecen al conjunto de los problemas llamados \textit{NP-Duros}, pues los algoritmos clásicos empleados para resolverlos tienen una complejidad al menos de tipo exponencial. Clásicamente se han empleado por ejemplo los algoritmos 





se han tratado de resolver empleando diversas técnicas: inicialmente se emplearon heurísticas

\subsubsection{Búsqueda en Entornos Variables (VNS)}
test
\subsubsection{Adaptación del VNS al problema}
test
\paragraph{Función Fitness} 
test

\paragraph{Definiciones de entornos}
test

\paragraph{Búsqueda diversificada/intensificada}
test

\paragraph{Condiciones de Parada}
test












   	\newpage

    
\section{Implementación} \label{capitulo:4}

Como ingeniero informático, este proyecto se ha desarrollado de forma orientada al proceso de ingeniería de software que consta de las siguientes etapas o procesos: 
\begin{enumerate}
    \item Planificación
    \item Análisis
    \item Diseño
    \item Implementación propiamente dicha
    \item Pruebas
    \item Despliegue y explotación    
\end{enumerate}
%planificación, análisis, diseño, implementación propiamente dicha, pruebas y despliegue/explotación. 
Adicionalmente existe la etapa de monitorización y seguimiento que se desarrollan posteriormente de la puesta en marcha del sistema en un entorno real, y esto no forma parte del presente TFM debido a su naturaleza de investigación (al igual que el propio máster), así que no se han llevado a cabo dichos procesos.

En las sucesivas secciones de este capítulo iremos describiendo los procesos llevados a cabo en cada una de estas etapas y las conclusiones alcanzadas en las mismas.

\subsection{Planificación}
En esta primera etapa presente en todo proyecto tratamos de fijar principalmente el concepto de alcance, que podríamos definir como el trabajo realizado para entregar un producto, servicio o resultado con
las funciones y características especificadas~\cite{PMBOK}.

Los proyectos de naturaleza de investigación e innovación como éste requieren de una cierta flexibilidad a la hora de definir el alcance, puesto que planteamos hipótesis iniciales que en función de ser válidas o no, plantearemos otras nuevas. Como éste TFM forma parte de un proyecto mayor, es más sencillo definir el alcance, aunque este sufrió alguna modificación.

El proyecto parte con la definición de las hipótesis iniciales recogidas en la \autoref{sec:Hipotesis}. 
Pretendía abarcar en primera instancia la comprensión del dominio del problema y la lógica de negocio sobre la que poder comenzar a trabajar, y tras ello, podría dar comienzo el desarrollo de la metaheurística VNS descrita en el \autoref{capitulo:3} y analizar los resultados comparándolos con el SA (experimentación, recopilado en el \autoref{capitulo:5}). Al comienzo del proyecto se implementarían los 5 tipos de VNS descritos previamente:
\begin{itemize}
	\item Variable Neighborhood Descent (VND)
	\item Reduced Variable Neighborhood Search (RVNS)
	\item Basic Variable Neighborhood Search (BVNS)
	\item General Variable Neighborhood Search (GVNS)
	\item Skewed Variable Neighborhood Search (SVNS)
\end{itemize}

Y posteriormente fue ampliado con la variación de la naturaleza de los entornos probabilista.

Por otro lado, el sistema organizativo acordado con los tutores para este TFM consistió en un desarrollo iterativo en ciclos marcados por metas en una jornada de trabajo en un despacho ofrecido por el grupo de investigación de 8 horas diarias de Febrero de 2019 a Junio de 2019. Finalmente las fechas fueron ampliadas. Se trata de una variación de los sprints de SCRUM combinada con un tablero KANBAN creado en la aplicación web Tello. 

A continuación describiremos cada uno de esos ciclos.

El primer ciclo incluía el proceso de planificación y de análisis junto al estudio del sistema y los documentos previos facilitados por los tutores (\textit{literature review}). Tuvo por meta la comprensión del dominio del problema (primera parte) y de la lógica de negocio utilizada en los antecedentes del TFM (véase la \autoref{capitulo:2:detalles-sistema})

Los demás ciclos únicamente requerirán de las fases de diseño, implementación y pruebas. La meta para el segundo era la implementación de la \faseuno{} (diseño, implementación y pruebas). El tercer ciclo como meta fue la implementación de una primera versión funcional del sistema implementando unicamente el VND, definiendo previamente los diferentes entornos a emplear. Para el cuarto ciclo se debían implementar el resto de variaciones del VNS así como realizar un despliegue del sistema empaquetandolo y enviandolo en forma de entrgable a los tutores (véase la \autoref{sec:4:despliegue}). El cuarto ciclo fue añadido posteriormente para subsanar errores e implementar los entornos probabilísticos que ya fueron descritos en la \autoref{capitulo:3:busqueda-divers-intens}. Finalmente se encuentra el ciclo de experimentación, donde comparamos los resultados y analizamos la eficiencia del sistema.

En la \autoref{sec:4:implementacion} se describen las tecnologías software utilizadas. %TODO: hacerlo efectivamente

% TODO: ¿cronograma?

\subsection{Análisis}
Lorem ipsum

\subsection{Planificación}
Lorem ipsum

\subsection{Diseño}
Lorem ipsum

\subsection{Implementación}
\label{sec:4:implementacion}

Las herramientas software a utilizar serán:
\begin{itemize}
	\item Tello (planificación)
	\item IntelliJ IDEA (implementación)
	\item Jupyther Notebook (experimentación)
	\item Microsoft Office (diagramas)
	\item \TeX{} Studio (memoria)
\end{itemize}


\subsection{Pruebas}
Lorem ipsum

\subsection{Despliegue y explotación} 
\label{sec:4:despliegue}

Lorem ipsum



















\NOTE{CONTAR AQUI LAS ETAPAS DE SW. Y DESPUES DESCRIBIR UNA A UNA EL INPUT Y EL OUPUT. PARA LA FASE DE REQUISITOS AQUÍ HAY ALGUNO:}



Con todo, el sistema tiene cuatro módulos:
\begin{itemize}
	\item Módulo de lectura de datos: lleva a cabo las tareas de lectura e inicialización de estructuras de datos.
	\item Módulo de inicialización: inicializa la solución inicial de acuerdo a la(s) contingencias recibidas del módulo anterior
	\item Módulo de búsqueda: lleva a cabo la búsqueda de una solución factible al problema.
	\item Módulo de entrega de datos: lleva a cabo las tareas de escritura y trazabilidad de las soluciones.
\end{itemize}



\subsection{Requisitos del sistema}
Una vez definidos los conceptos básicos, procedemos a recopilar las características y restricciones del sistema.

\subsubsection{Requisitos de entrada/salida}

\begin{enumerate}[label={\textbf{RIO\arabic*}}]
	\item  Una entrada al sistema se compondrá de dos partes: la información de la dependencia y la información del caso,
	de esta forma, la información común a varios casos será independiente de cada caso concreto.
	\item La información de la dependencia será un subdirectorio con el nombre de la dependencia, contendrá 4 ficheros:
	\begin{enumerate}[label*={\textbf{.\arabic*}}]
		\item  Lista de todos los sectores pertenecientes la unidad de control y los sectores elementales\footnote{
			Sector que comprende una zona del espacio aéreo que no es subdivisible empleando otros sectores. Recuérdese el sector LECMBDP (azul) de la \autoref{fig:2:sectorizacion-3d} se podía sustituir por otros más pequeños, por lo tanto no es elemental
		} por los que están formados cada uno de los sectores.
		
		\item  Matriz de Afinidad de los sectores de la dependencia (definida en la 	\autoref{section:2:sectores-y-sectorizacion})
		\item Lista de los sectores pertenecientes a la unidad de control, en la que nos indica el tipo de sector (véase~\ref{RD-tipos-sector}) y los núcleos a los que pertenece (ver~\ref{RD-sector-nucleo}).
	\end{enumerate}
	
\end{enumerate}


\subsubsection{Restricciones de dominio}
Las restricciones de dominio son aquellos requisitos del sistema que son impuestos únicamente por el dominio del problema, no por la propia naturaleza del sistema ni de forma externa.

\begin{enumerate}[label={\textbf{RD\arabic*}}]
	\item \label{RD:tipos-sector}  Cada sector tendrá un tipo de sector, que podrá ser Aproximación o Ruta
	\item  \label{RD:sector-nucleo} Cada sector tendrá uno o varios núcleos asociados, así como cada núcleo tendrá un conjunto de sectores (relación N a N)
	
\end{enumerate}

\subsubsection{Detalles de implementación del sistema}
\label{sec:detalles-impl-sistema}

    \newpage
    
    \graphicspath{{capitulos/Capitulo5-Resultados-experimentales/recursos/}}

\section{Resultados experimentales} \label{capitulo:5}
%En este capítulo se detallan los casos de prueba empleados para la parte de experimentación realizada para este TFM. La metaheurística de la \fasedos{} del sistema (definida en la \autoref{sec:3:metaheurística}) consta de un conjunto de parámetros, enumerados en esta sección, que afectan al rendimiento de esta. Se ha analizado para cada caso, los valores de cada parámetro que mejores resultados ofrecen.

En este capítulo se detallan los procesos realizados para la parte de experimentación realizada en este TFM. En primer lugar se definen los casos de prueba empleados, posteriormente se detalla el proceso de ajuste de los parámetros, presente en toda metaheurística y que nos permite fijar los valores del VNS empleado en la \fasedos{} del sistema (definida en la \autoref{sec:3:metaheurística}) a aquellos que ofrecen mejores resultados. Por último, se ha hecho una comparación de rendimiento de la metaheurística implementada frente a la ya mencionada metaheurística \textit{Simulated Annealing}.

Las ejecuciones presentadas a lo largo de este capítulo han sido realizadas en un ordenador con las características recopiladas en la \autoref{table:5:caracteristicas-pc}.

\begin{table}[h]
	\centering
	\caption{Características del ordenador empleado para la experimentación}
	\begin{tabular}{lcc}
		\hline
		Procesador   & Intel Core i7 &  \\
		Memoria      &     16GB      &  \\
		Version Java &    JDK 8.1    &  \\ \hline
		             &               &
	\end{tabular}
\label{table:5:caracteristicas-pc}
\end{table}

\subsection{Definición de los casos de prueba}
\label{sec:5:def-casos}

Para este capítulo se han utilizado un conjunto de casos (instancias) de prueba reales que fueron proporcionados por CRIDA. Inicialmente CRIDA proporcionó información en los formatos de ficheros propuesto (véase \autoref{sec:4:req-io}) que fue adaptada para conformar los 8 casos de prueba distintos que se han empleado y definiremos a continuación. Se ha incluido en el \autoref{Anexo:C} una tabla de los casos que permite ver de forma más detallada y clara las características de cada uno de ellos.

\subsubsection{Caso 1}

\textbf{Unidad de Control}: Barcelona

\textbf{Incidencia}: Modificación de sectorizaciones. La \autoref{fig:5:caso1} muestra cómo es este cambio.
Como podemos ver...



%\begin{table}[h]
%	\begin{tabular}{|c|c|c|}
%		\hline
%		          \textbf{Núcleo}            & \textbf{Sectorización} & \textbf{Intervalo} \\ \hline
%		\multirow{2}{*}{Barcelona Ruta Este} &           3D           & 7:30:00--15:00:00  \\ \cline{2-3}
%		                                     &           5A           & 7:30:00--10:30:00  \\ \hline
%		        Barcelona Ruta Oeste         &           6C           & 10:30:00--15:00:00 \\ \hline
%	\end{tabular}
%\end{table}
\begin{table}[h]
		\centering
	\caption{Sectorización modificada del Caso 1}
	\begin{tabular}{ccc}
		\hline
		\textbf{Núcleo}                                           & \textbf{Configuración} & \textbf{Intervalo}   \\ \hline
		\multicolumn{1}{l}{}                                      & \multicolumn{1}{l}{}   & \multicolumn{1}{l}{} \\
		\multicolumn{1}{c|}{\multirow{2}{*}{Barcelona Ruta Este}} & 3D                     & 7:30:00--15:00:00    \\
		\multicolumn{1}{c|}{}                                     & 5A                     & 7:30:00--10:30:00    \\
		\multicolumn{1}{l}{}                                      & \multicolumn{1}{l}{}   & \multicolumn{1}{l}{} \\
		Barcelona Ruta Oeste                                      & 6C                     & 10:30:00--15:00:00   \\ \hline
	\end{tabular}
	\label{table:5:caso1-modif}
\end{table}


\subsubsection{Caso 3}

\textbf{Unidad de Control}: Barcelona

\textbf{Situación inicial}:
\begin{itemize}[label={}]
	
	\item \textbf{Turno}: MC, 7:30-15:00
	
	\item \textbf{Recursos}: \\
	7 PTD Barcelona Ruta Este \\
	17 PTD Barcelona Ruta Oeste
	
	
	\item \textbf{Sectorización}: véase la \autoref{table:5:caso3-inicial}
	\begin{table}[h]
		\centering
		\caption{Sectorización inicial del Caso 1}
		\begin{tabular}{ccc}
			\hline
			\textbf{Núcleo}      & \textbf{Configuración} & \textbf{Intervalo}   \\ \hline
			\multicolumn{1}{l}{} & \multicolumn{1}{l}{}   & \multicolumn{1}{l}{} \\
			Barcelona Ruta Este  & 3D                     & 7:30:00--15:00:00    \\
			\multicolumn{1}{l}{} & \multicolumn{1}{l}{}   & \multicolumn{1}{l}{} \\
			Barcelona Ruta Oeste & 5A                     & 10:30:00--15:00:00   \\ \hline
		\end{tabular}
		\label{table:5:caso3-inicial}
	\end{table}
	
	
\end{itemize}

\textbf{Momento Actual}: 10:00:00

\textbf{Tipo incidencia}: Baja de un controlador

\textbf{Descripción}:Se produce una baja del controlador $c_{23}$ a las 9:30. No se producen altas. Se necesita 1 controlador imaginarios para inicializar este caso.

\subsubsection{Caso 4}

\textbf{Unidad de Control}: Madrid

\textbf{Situación inicial}:
\begin{itemize}[label={}]
	
	\item \textbf{Turno}: MC, 7:30-15:00
	
	\item \textbf{Recursos}: \\
	7 PTD Barcelona Ruta Este \\
	17 PTD Barcelona Ruta Oeste
	
	
	\item \textbf{Sectorización}: véase la \autoref{table:5:caso1-inicial}
	\begin{table}[h]
		\centering
		\caption{Sectorización inicial del Caso 1}
		\begin{tabular}{ccc}
			\hline
			\textbf{Núcleo}      & \textbf{Configuración} & \textbf{Intervalo}   \\ \hline
			\multicolumn{1}{l}{} & \multicolumn{1}{l}{}   & \multicolumn{1}{l}{} \\
			Barcelona Ruta Este  & 3D                     & 7:30:00--15:00:00    \\
			\multicolumn{1}{l}{} & \multicolumn{1}{l}{}   & \multicolumn{1}{l}{} \\
			Barcelona Ruta Oeste & 5A                     & 10:30:00--15:00:00   \\ \hline
		\end{tabular}
		\label{table:5:caso1-inicial}
	\end{table}
	
	
\end{itemize}

\textbf{Momento Actual}: 10:00:00

\textbf{Tipo incidencia}: Modificación de sectorizaciones

\textbf{Descripción}:Pasamos de una 3D a una 5A, lo que implica el cierre de un sector y la apertura de otros dos. Además se abre un sector adicional en el núcleo \textit{Barcelona Ruta Oeste}. La nueva sectorización es la de la \autoref{table:5:caso1-modif}. Se necesitan 3 controladores imaginarios para inicializar éste caso.

%\begin{table}[h]
%	\begin{tabular}{|c|c|c|}
%		\hline
%		          \textbf{Núcleo}            & \textbf{Sectorización} & \textbf{Intervalo} \\ \hline
%		\multirow{2}{*}{Barcelona Ruta Este} &           3D           & 7:30:00--15:00:00  \\ \cline{2-3}
%		                                     &           5A           & 7:30:00--10:30:00  \\ \hline
%		        Barcelona Ruta Oeste         &           6C           & 10:30:00--15:00:00 \\ \hline
%	\end{tabular}
%\end{table}
\begin{table}[h]
	\centering
	\caption{Sectorización modificada del Caso 1}
	\begin{tabular}{ccc}
		\hline
		\textbf{Núcleo}                                           & \textbf{Configuración} & \textbf{Intervalo}   \\ \hline
		\multicolumn{1}{l}{}                                      & \multicolumn{1}{l}{}   & \multicolumn{1}{l}{} \\
		\multicolumn{1}{c|}{\multirow{2}{*}{Barcelona Ruta Este}} & 3D                     & 7:30:00--15:00:00    \\
		\multicolumn{1}{c|}{}                                     & 5A                     & 7:30:00--10:30:00    \\
		\multicolumn{1}{l}{}                                      & \multicolumn{1}{l}{}   & \multicolumn{1}{l}{} \\
		Barcelona Ruta Oeste                                      & 6C                     & 10:30:00--15:00:00   \\ \hline
	\end{tabular}
	\label{table:5:caso1-modif}
\end{table}

\subsection{Ajuste paramétrico}
En toda metaheurística y demás sistemas de optimización, tienen un conjunto de parámetros que pueden tomar un conjunto de posibles valores y que afectan activamente al rendimiento del sistema. Para poder fijar estos valores, se lleva a cambo un proceso denominado Ajuste Paramétrico, o \textit{Parameter Tunning}, que se puede realizar de diferentes formas.

El enfoque de inicialización \textit{off-line} consiste en fijar los valores previamente a la ejecución de la metaheurística de forma empírica para cada instancia del problema dado. Este proceso suele realizar de forma secuencial, es decir, de uno en uno. Sin embargo esta estrategia no considera las interacciones entre los parámetros y no garantiza hallar la configuración optima de los parámetros. Existen otras estrategias como \textit{Latin Hypercube}, emplear \textit{Racing Algorithms} o incluso puede ser planteado como otro problema de optimización a resolver mediante otra metaheurística.

Por otro lado se considera el enfoque de inicialización de parámetros \textit{on-line}, que permite una evolución dinámica, en tiempo de ejecución, de los valores de cada parámetro en función del rendimiento del sistema u otro criterio determinista o estocástico.

En este TFM, por ser lo más común y sencillo, se decidió emplear un enfoque \textit{off-line} de forma secuencial. Para ello, se han de ordenar los parámetros en función de su robustez, es decir, lo mucho que afecte un pequeño cambio en el parámetro al desempeño del algoritmo y, secuencialmente en ese orden, fijar el valor de todos los demás y realizar ejecuciones de la metaheurística con diferentes valores para el parámetro en cuestión para finalmente seleccionar aquel valor que mejores resultados ofrece. A continuación se repite el proceso para el siguiente parámetro, sucesivamente.
Una vez ajustados todos los parámetros, se repite el proceso desde el principio hasta que ningún parámetro cambie de valor.

% ALFONSO SAID:
% Empieza por aquel parámetro que creas que es menos robusto, es decir, aquel que al realizar un pequeño cambio en el parámetro pueda hacer que cambie significativamente el desempeño del algoritmo. Una vez que determines su valor más adecuado, lo fijas y analizas el segundo parámetro menos rubusto. Una vez que determines su valor más adecuado, lo fijas y analizas el tercer parámetro menos robusto. Y así sucesivamente, hasta el último y vuelves a empezar para realizar otro ciclo hasta que los parámetros no cambien o cambien muy poco. Siempre cambias un parámetro, nunca realices cambios en más de un parámetro a la vez.

\subsubsection{Parámetros del sistema} \label{sec:5:parametros-sistema}
Los parámetros del sistema, ordenados por su robustez son:
\begin{enumerate}
	\item Tipo de VNS
	\begin{enumerate}[label={},left=-1pt]
		\item Para skewed:
	\end{enumerate}
	\begin{enumerate}[label*={\arabic*}]
		\item Alpha
		\item Función de distancia 
	\end{enumerate}
	\item Estructuras de vecindad y orden
	\item Naturaleza del orden de los entornos (determinísticos o probabilísticos)
	\begin{enumerate}[label={},left=-1pt]
		\item Para Probabilístico:
	\end{enumerate}
	\begin{enumerate}[label*={\arabic*}]
		\item Probabilidad de diversificación
		\item Variación de la probabilidad de diversificación 
		\item Numero de iteraciones sin variar la probabilidad de diversificación 
	\end{enumerate}
	\item Número de iteraciones para comprobar el porcentaje de mejoría (ciclos)
	\item Porcentaje mínimo de mejoría
	\item Número máximo de iteraciones sin mejora para la búsqueda local
	\item Porcentaje mínimo de mejoría para la búsqueda local
\end{enumerate}

\subsubsection{Resultados del ajuste para cada caso}

Los resultados tras el ajuste paramétrico se encuentran recogidos en la \autoref{table:5:tunning-results}

\begin{table}[h]
	\centering
	\caption{Sectorización modificada del Caso 1}
	\begin{tabular}{ccc}
		\hline
		\textbf{Núcleo}                                           & \textbf{Configuración} & \textbf{Intervalo}   \\ \hline
		\multicolumn{1}{l}{}                                      & \multicolumn{1}{l}{}   & \multicolumn{1}{l}{} \\
		\multicolumn{1}{c|}{\multirow{2}{*}{Barcelona Ruta Este}} & 3D                     & 7:30:00--15:00:00    \\
		\multicolumn{1}{c|}{}                                     & 5A                     & 7:30:00--10:30:00    \\
		\multicolumn{1}{l}{}                                      & \multicolumn{1}{l}{}   & \multicolumn{1}{l}{} \\
		Barcelona Ruta Oeste                                      & 6C                     & 10:30:00--15:00:00   \\ \hline
	\end{tabular}
	\label{table:5:tunning-results}
\end{table}

La calidad de las soluciones alcanzadas con este ajuste son las de la \autoref{table:5:soluciones}. Como puede verse.... \NOTE{TODO: Continuar}% TODO: continuar

\subsection{Comparación de metaheurísticas}
Lorem ipsum

    \newpage
    
    \graphicspath{{capitulos/Capitulo6-Conclusiones/recursos/}}


\section{Conclusiones} \label{capitulo:6}

Se ha descrito el problema en profundidad, y hemos propuesto una metodología concreta para su resolución empleando la metaheurística \vns{} (VNS), de la que esperábamos mejores resultados que con el \sa{} (SA). 

Se ha detallado dicha metodología, desglosándola en dos fases, una de inicialización, para poder adaptar la planificación de entrada a las nuevas exigencias originadas por las contingencias a tratar de resolver; y otra de resolución, donde empleamos la metaheurística implementada y adaptada al problema para obtener soluciones cuya calidad es medible mediante la función de evaluación, o fitness, que permite tanto a la metaheurística en sí como a nosotros mismos realizar comparaciones entre soluciones y diferentes ejecuciones. 

Empleando este indicador, entre otros, se ha procedido a realizar un ajuste de los parámetros previamente descritos del sistema, realizando un pequeño estudio del comportamiento de los mismos en función de los diferentes casos y valores. Por lo tanto, podemos decir que la hipótesis inicial \ref{H1} ha sido correcta: el problema ha sido resuelto exitosamente en el plazo acordado.

Además, de todos los diferentes tipos de VNS implementados, hemos concluido que aceptar soluciones peores no da soluciones buenas, funcionando mejor la versión \textit{Basic} o la \textit{Descendant} en todos los casos. En cuanto a la naturaleza de los entornos, hemos comprobado que para esta metaheurística con los entornos definidos aplicados a este problema, la toma de entornos según probabilidades no aporta lo suficiente en casi ningún caso de prueba.

También se ha realizado un estudio comparativo del sistema alterando únicamente la metaheurística empleada para poder evaluar la hipótesis inicial \ref{H2}. En vista de los resultados, podemos concluir que también es correcta, pues el VNS muestra resultados significativamente mejores que los obtenidos mediante el SA y en algunas de las instancias de prueba empleadas el comportamiento del VNS es considerablemente mejor, en especial con relación a la primera de las funciones objetivo, \ref{O1}, catalogada como la más importante.

Por último, la hipótesis \ref{H3} hacía referencia a la eficiencia del sistema, y para evaluarla se ha realizado un estudio del rendimiento de este, del que se concluye que la eficiencia ha mejorado para la parte más ardua: el cómputo de la función de evaluación. Dicha mejoría se ha realizado mediante un uso mayor de la memoria dinámica del programa, almacenando temporalmente los resultados de la función fitness de cada solución en una estructura de datos de lectura rápida (de baja complejidad computacional). A su vez, también se han cambiado otras estructuras de datos de tipo listas por otras de menor complejidad computacional de lectura. Sin embargo, por falta de tiempo e importancia de esta hipótesis frente a todas las demás, no se ha realizado la mejoría que posiblemente sea más significativa: cambiar la representación de los turnos de trabajo por otra mejor que la actual, que emplea cadenas de texto, pues su acceso y modificación es computacionalmente más costoso que otras estructuras propuestas, y dichas acciones se repiten de forma continuada a lo largo de toda la ejecución.

\subsection{Líneas futuras de trabajo}
\label{sec:6:trabajo-futuro}

El proyecto entre manos es de un tamaño mayor que este TFM en sí, pues forma parte de una línea de investigación de \gls{CRIDA} para automatizar el proceso de asignación de trabajo a sus empleados. En este TFM se continúa la línea de trabajo mediante la inclusión de nuevas funcionalidades que se esperan en el sistema automatizado final: el manejo de ciertas contingencias descritas.

Una vez concluido este TFM, la línea de trabajo continuará para poder mejorar los resultados alcanzados para que puedan ser realmente empleados por el personal del Centro de Control. Dicha mejoría puede realizarse en diferentes líneas de trabajo que proponemos a continuación.

En primera instancia, podemos mejorar la definición de la \faseuno{}, empleando una heurística de inicialización diferente. Proponemos el uso de más de un tipo de plantillas, pues hasta ahora tan solo se ha utilizado las de tipo $3\times1$, cuando existen otras que permiten emplear menos controladores para un mayor número de sectores. 

Por ejemplo, en el caso de tener que añadir 3 nuevos sectores a la planificación inicial debido a que no se han podido sustituir por otros afines que se cierran, podemos emplear una plantilla $8\times3$, que es de la forma de la \autoref{fig:6:plantilla8x3}, donde cada letra representa un sector (mayúsculas en puesto ejecutivo, minúsculas en planificador) y un guion un periodo de descanso. Esta plantilla permite emplear un controlador menos de los que empleamos con las plantillas $3\times1$ (véase la \autoref{fig:3:plantilla-3x1}), pues añade 3 controladores para cada sector, una totalidad de 9 controladores. Otras posibles plantillas son la $4\times1$, que al tener más descansos permite más movilidad de periodos de trabajo de cara a la \fasedos{}. %TODO: poner la otra plantilla también??

\begin{figure}
	\centering
	\includegraphics[width=\linewidth]{plantilla8x3}
	\caption{Aspecto de una plantilla $8\times3$.}
	\label{fig:6:plantilla8x3}
\end{figure}

En la definición de esta fase, la \autoref{sec:3:inicializacion-soluciones}, propusimos la implementación de un quinto paso para mejorar la calidad de la solución inicial. 
Además, se puede mejorar la eficiencia de la heurística de selección de sectores afines para su sustitución empleando un algoritmo más inteligente que simplemente uno de tipo voraz.

En segunda instancia, podemos mejorar la eficiencia de la \fasedos{}. Para ello se podrán emplear otras definiciones de entornos diferentes, y puesto que son la componente más relevante dentro de la metaheurística VNS, cambiarán notablemente el rendimiento de la misma. 

Otra línea por estudiar es el uso de otras funciones de distancia para el SVNS. Como hemos visto, en nuestra definición de SVNS y funciones de distancia no ha habido buenos resultados, pero esto no quiere decir que empleando otras funciones alternativas no se obtengan mejores resultados. 

Otra posibilidad para mejorar la eficiencia de la segunda fase consiste en emplear otra metaheurística diferente a VNS o SA, por ejemplo \textit{Tabu Search} (TS) (Búsqueda Tabú) o incluso otras técnicas de inteligencia artificial nombradas al inicio de la \autoref{sec:3:metaheurística} como \textit{Machine Learning} e incluso combinando diferentes técnicas.

Por otro lado, las soluciones alcanzadas por el sistema podrían aportar más valor a los expertos de CRIDA si se realizase un estudio en mayor profundidad de las necesidades de negocio de cara a ponderar de forma más precisa y personalizada las diferentes funciones objetivo, en lugar de utilizar el método ROC.

En tercera instancia, mejorar la eficiencia general del sistema, de la forma mencionada en la sección anterior: rediseñando las estructuras de datos básicas de representación de soluciones del sistema en su conjunto, lo que creemos aportará una mejora de eficiencia significativa. Otra opción es portar el sistema a otro lenguaje de programación más orientado a la eficiencia como puede ser C/C++. 

Por último, parece importante probar el sistema en más profundidad, pues los casos aportados por CRIDA para este trabajo no prueban todas las restricciones y casuísticas del problema entre manos. Un ejemplo de aspecto del dominio del problema que no ha sido probado son los sectores nocturnos. También sería adecuado realizar un estudio de casos de uso del sistema, que permita poder organizar y crear diferentes los casos de prueba de forma que se garantice que se prueban todas las funcionalidades requeridas por CRIDA.

    \newpage
    
    %\include{capitulos/CapituloXXX-YYY/CapituloXXX}
    %\newpage

	\begin{appendices}
%    \section*{ANEXOS}
%	\appendix
%	\graphicspath{{anexos/recursos/}}
	
    \graphicspath{{anexos/AnexoA-Formato-Planificacion-Inicial/recursos/}}

\section{Formatos de los Ficheros de Entrada} \label{Anexo:formato-planificacion-inicial}

El \autoref{capitulo:4}, concretamente en su \autoref{sec:4:req-io} recopila en forma de requisitos de software el formato que debe tomar los ficheros de entrada del sistema. En este anexo describimos en mayor profundidad dichos formatos, con ejemplos de cada uno de ellos.

Los ficheros de entrada se dividen en dos grupos, uno destinado a la información de la Dependencia o Unidad de Control concreta (por ejemplo, Palma, Barcelona, Madrid...) y otro destinado a la información específica del caso o instancia del problema.

Es importante destacar que todos los ficheros que describiremos a continuación emplean un formato CSV separado por punto y coma (;) en codificación UTF-8, que permite representarlo en forma de tablas.

\subsection{Ficheros de las Dependencias}

Encontramos un total de 4 ficheros, todos ellos obligatorios: en caso de no encontrarse uno de ellos, el sistema no podrá ejecutarse.

El primero de ellos, es la relación de sectores elementales, y deberá denominarse ``\texttt{ListaSectoresElementales\_Dependencia.csv}'' (donde \texttt{Dependencia} es el nombre de la Dependencia). Tiene la forma de la \autoref{fig:A:ejemplo-dependencias-sectores-elementales}. En caso de ser elemetal el propio sector, la columna correspondiente deberá estar vacía.

\begin{figure}[h]
	\centering
	\includegraphics[width=0.5\linewidth]{ejemplo-dependencias-sectores-elementales}
	\caption{Fragmento del fichero \texttt{ListaSectoresElementales\_Barcelona.csv} como ejemplo del formato del fichero}
	\label{fig:A:ejemplo-dependencias-sectores-elementales}
\end{figure}

En segundo fichero es la matriz de afinidad de los sectores de la Unidad de Control. Es una matriz cuadrada de tamaño considerable, cuya primera fila y columna son iguales: la lista de los sectores en el mismo orden. Debe llamarse ``\texttt{MatrizAfinidad\_Dependencia.csv}''

\begin{figure}[h]
	\centering
	\includegraphics[width=\linewidth]{ejemplo-dependencias-matriz-afinidad}
	\caption{Fragmento de la Matriz de afinidad de Barcelona, tomada del fichero \texttt{MatrizAfinidad\_Barcelona.csv}}
	\label{fig:A:ejemplo-dependencias-matriz-afinidad}
\end{figure}

En caso de tomar valor 1, significa que los sectores son afines, mientras que si es 0, no lo son. Como puede verse, la diagonal principal siempre toma el valor de 1. Recuérdese que es un CSV, por lo que el fichero desde un editor de texto plano se verás así:

\noindent\fbox{%
	\parbox{\textwidth}{%
		[];L1234DXE;L1234DXN;L1234DXW;LEB1234E;LEB1234W;LEB12FXW; . . . \\
		L1234DXE;1;1;1;1;1;1;1;1;1;1;1;1;1; . . . \\
		L1234DXN;1;1;1;1;1;0;1;1;1;1;1;1;1; . . . \\
		L1234DXW;1;1;1;1;1;1;1;1;1;1;1;1;1; . . . \\
		LEB1234E;1;1;1;1;1;1;1;1;1;1;1;1;1; . . . \\
		LEB1234W;1;1;1;0;1;1;1;1;1;1;1;1;1; . . . \\
		LEB12FXW;1;1;1;1;1;1;1;0;1;1;1;1;1; . . . \\
		 . . . \\
	}%
}

El tercer fichero conforma la información que relaciona los sectores que pertenece a la Unidad de Control con los núcleos a los que pertenece cada uno de ellos. Además, incluye el tipo de sector, que puede ser ``RUTA'' o ``APP'' (aproximación). Debe llamarse ``\texttt{SectoresNucleos\_Dependencia}''. Véase un ejemplo en la \autoref{fig:A:ejemplo-dependencias-sectores-nucleos}

\begin{figure}[h]
	\centering
	\includegraphics[width=0.5\linewidth]{ejemplo-dependencias-sectores-nucleos}
	\caption{Fragmento del fichero \texttt{SectoresNucleos\_Barcelona.csv}}
	\label{fig:A:ejemplo-dependencias-sectores-nucleos}
\end{figure}

El cuarto y último fichero relativo a las Dependencias tiene se denomina de la forma: \texttt{SectorizacionesSectoresVolumenes\_Dependencia.csv}'' y es aquel que recopila todas las configuraciones posibles de los sectores. Es aquí donde la cargamos al sistema la información que nos dice qué sectores pertenecen a la sectorización 3A, por ejemplo. Además, se ha de incluir más de una fila con la misma sectorización y el mismo sector, pero diferentes volúmenes para poder indicar cúales son los volúmenes que abarca un sector concreto. Por ejemplo, véase en la \autoref{fig:A:ejemplo-dependencias-sectorizaciones-sectores-volumenes} cómo la sectorización 1A y el sector LECBBKE aparecen 7 veces con un volumen distinto.

\begin{figure}[h]
	\centering
	\includegraphics[width=0.6\linewidth]{ejemplo-dependencias-sectorizaciones-sectores-volumenes-reducida}
	\caption{Fragmento del fichero \texttt{SectorizacionesSectoresVolumenes\_Barcelona.csv}}
	\label{fig:A:ejemplo-dependencias-sectorizaciones-sectores-volumenes}
\end{figure}

Es importante destacar que no es lo mismo una sectorización para un núcleo u otro. Nótese en la figura anterior que los sectores que conforman una sectorización 3A para el núcleo ``Barcelona Ruta Este'' no son los mismos que aquellos indicados en el núcleo ``Barcelona Ruta Oeste''.

\subsection{Ficheros de los casos concretos}

Estos ficheros conforman los diferentes casos prueba del sistema, definidos en el \autoref{capitulo:5} y detallados el \autoref{Anexo:tabla-casos}. Cada caso tiene un identificador, el del Caso 1 empleado en este TFM es ``Id1m-01-01-2019'' y el nombre de cada fichero de entrada llevará al final su identificador.

El primero de los ficheros define el turno del caso: a qué hora comienza y termina y el tipo de turno, como puede verse en la \autoref{fig:A:ejemplo-casos-turno}.

\begin{figure}[h]
	\centering
	\includegraphics[width=0.7\linewidth]{ejemplo-casos-turno}
	\caption{Fragmento del fichero \texttt{Turno\_Id1m-01-01-2019.csv}}
	\label{fig:A:ejemplo-casos-turno}
\end{figure}

Las siglas de los tipos de turnos son:

\begin{enumerate}[align=left]
	\item[M] Mañana
	\item[T] Tarde 
	\item[ML] Mañana Largo
	\item[MC] Mañana Corto
	\item[TL] Tarde Largo
	\item[TC] Tarde Corto
	\item[N] Noche
\end{enumerate}


El segundo de los ficheros define los intervalos en los que están abiertas cada sectorización. Como se muestra en la \autoref{fig:A:ejemplo-casos-sectorizaciones}, la aprtura de sectores puede indicarse con el nombre del sector o con una configuración, aunque lo más común es hacerlo mediante la configuración (CONF). Se requiere del núcleo en primera instancia, delimitando el espacio concreto que estamos definiendo. Tras el campo del tipo de información a leer del fichero, se indica si el o los sectores son de tipo nocturno (0 para false, 1 para verdadero), el nombre del sector o configuración según el tipo indicado y las hora de inicio y fin en las que dicha configuración se encuentra abierta.

\begin{figure}[h]
	\centering
	\includegraphics[width=\linewidth]{ejemplo-casos-sectorizaciones}
	\caption{Fragmento del fichero \texttt{AperturaSectorizaciones\_Id1m-01-01-2019.csv}}
	\label{fig:A:ejemplo-casos-sectorizaciones}
\end{figure}

El tercer fichero describe los recursos humanos disponibles (los controladores) para cubrir las sectorizaciones indicadas en el fichero anterior. Se definen de la forma de la \autoref{fig:A:ejemplo-casos-controladores}. Como podemos ver, cada uno tiene un identificador, que debe ser único, y tiene dos tipos de acreditación, aquella que limita el tipo de sectores que puede controlar, y aquella que limita la zona del espacio aéreo que puede controlar (mediante núcleos).

\begin{figure}[h]
	\centering
	\includegraphics[width=0.6\linewidth]{ejemplo-casos-controladores}
	\caption{Fragmento del fichero \texttt{RecursosDisponibles\_Id1m-01-01-2019.csv}}
	\label{fig:A:ejemplo-casos-controladores}
\end{figure}

Los dos siguientes ficheros permiten introducir en la entrada las incidencias, uno para las de tipo de cambio de sectorización respecto al inicial esperado y otro para altas y/o bajas de controladores.

El fichero destinado a la incidencia de tipo sectorización tiene el mismo formato que el segundo, representado en la \autoref{fig:A:ejemplo-casos-sectorizaciones}. Este fichero tendrá como nombre \texttt{ModificacionSectorizaciones\_Identificador.csv}.

El fichero destinado a la incidencia de altas y bajas de controladores tiene un formato similar al tercero, pero añadiendo una columna al principio, que permita definir si se trata de un alta o una baja, véase la \autoref{fig:A:ejemplo-casos-baja-alta}. El controlador de baja debe existir en el fichero de los recursos disponibles, mientras que el de alta debe ser un controlador nuevo. Por ello, en el caso de la baja, los datos del controlador (columnas ACREDITACIÓN, NÚCLEO y TURNO) son opcionales.

\begin{figure}[h]
	\centering
	\includegraphics[width=0.8\linewidth]{ejemplo-casos-baja-alta}
	\caption{Fragmento del fichero \texttt{ModificacionRecursos\_Id6t-19-10-2018.csv} (Caso 7)}
	\label{fig:A:ejemplo-casos-baja-alta}
\end{figure}

Por último tenemos el fichero más importante: la planificación inicial del sistema, aquella que deja de ser válida debido a la contingencia acontecida. Este fichero se creó específicamente para este TFM y en un futuro, de cara al uso real de este sistema, deberá ser reemplazado por otro formato que sea común al del sistema que crea las planificaciones y que aquí llamamos \legacy{}. Este fichero fue creado lo más parecido posible al que emplean los trabajadores de los centros de control, que son de la forma de las Figuras \ref{fig:A:ejemplo-distribucion-crida-1} y \ref{fig:A:ejemplo-distribucion-crida-2}

\begin{figure}[h]
	\centering
	\includegraphics[width=\linewidth]{anexos/AnexoA-Formato-Planificacion-Inicial/recursos/ejemplo-distribucion-crida-1}
	\caption{Ejemplo de una de las planificaciones reales de CRIDA para el sector LECBMNI del Núcleo ESTE.}
	\label{fig:A:ejemplo-distribucion-crida-1}
\end{figure}

\begin{figure}[h]
	\centering
	\includegraphics[width=\linewidth]{anexos/AnexoA-Formato-Planificacion-Inicial/recursos/ejemplo-distribucion-crida-2}
	\caption{Ejemplo de una de las planificaciones reales de CRIDA para los sectores LECBLVL, LECBLVS y LECBLVU del Núcleo OESTE.}
	\label{fig:A:ejemplo-distribucion-crida-2}
\end{figure}

Como puede verse se trata de, en el caso de la \autoref{fig:A:ejemplo-distribucion-crida-1}, una plantilla $3\times1$ (véase la \autoref{fig:3:plantilla-3x1} de la \autoref{sec:3:inicializacion-soluciones}); y en el caso de la \autoref{fig:A:ejemplo-distribucion-crida-2} una $8\times3$ (véase la \autoref{fig:6:plantilla8x3} de la \autoref{sec:6:trabajo-futuro}). Tiene la peculiaridad de que se indican los intervalos de hora en la cabecera de las tablas (en las figuras anteriores pueden verse 50m, 38m y 19m) en lugar de repetirse slot a slot como se venia haciendo en el formato de salida del sistema \legacy{}. 

Para pedir a CRIDA información de casos de prueba no podíamos emplear un formato slot a slot, pues sería costoso de crear de cara al personal de la organización. Lo mas sencillo fue replicar los estadillos anteriores en un solo fichero, de la forma de la figura adjunta al final del anexo. Dicha figura es autoexplicativa y fue la que se envió a CRIDA para que empleara como formato para enviarnos la información relativa a los casos de prueba. Podemos apreciar cómo las cabeceras (marcadas en color naranja pálido) tienen los números que en los estadillos de las figuras anteriores marcaban de forma equivalente, en ambos casos en minutos.

Por último se ha de notar que en este nuevo formato se ha de incluir todas las columnas de manera que la suma de los minutos de el total del tiempo del turno. Una limitación de la escalarización del tiempo en slots de 5 minutos (véase la \autoref{sec:3:representacion-soluciones}) es que los minutos que indiquemos en este fichero deben ser múltiplos de 5, por lo que la precisión del tiempo de la \autoref{fig:A:ejemplo-distribucion-crida-2} no es posible de codificar. En su lugar empleamos el múltiplo de 5 más cercano, procurando que la suma total siga siendo la misma, los 450 minutos que dura el turno. La figura adjunta no está completa, por eso aparecen unos puntos suspensivos al final del último estadillo, de lo contrario la imagen no cabría en este documento. El fichero con esta información se ha llamado \texttt{DistribucionInicial\_Identificador.csv}, siendo el del documento adjunto aquel empleado para el Caso 1.

\begin{landscape}
	\includepdf[landscape=True]{DistribucionInicial}
\end{landscape}


%\blankpage

 	\graphicspath{{anexos/AnexoB-Diagrama-Clases-Completo/recursos/}}

\section{Diagrama de clases completo} \label{Anexo:diagrama-clases}

En este anexo incluimos el diagrama de clases completo, que se encuentra explicado detalladamente en la \autoref{sec:4:diseño} relativa al diseño del software del presente proyecto.

Este esquema puede ser visto en 4 partes: azul, amarillo/naranja, verde y rosa/magenta. Todas ellas emplean el principio de diseño software SOLID de segregación de la interfaz, que favorece el desacoplamiento del sistema; y el patrón de diseño \textit{Template Method}, que permite reutilizar código común reescribiendo únicamente aquellas partes diferenciadores de cada elemento concreto.

Por ejemplo, en la parte verde, relativa al paquete que representa el proceso de búsqueda VNS, se emplea el método \texttt{startExecution} como plantilla que emplea los demás métodos. En especial tenemos el método abstracto \texttt{vnsImplementation}, que se define unicamente en cada VNS concreto. Por ejemplo del VNS Descent emplea su interfaz de la estructura de entornos \texttt{NeighborhoodSet} para comunicar al sistema que desea realizar una búsqueda local. La estrucura de entornos enviará el mensaje a la interfaz \texttt{NeighborhoodStructure}, encargada de implementar los movimientos en sí, y empleará el tipo de entornos que la anterior interfaz le ha facilitado, por ejemplo, los deterministas. De esta forma, el paso de mensajes típicamente se está realizando de derecha a izquierda. Un ejemplo típico de éste paso de mensajes está representado en la \autoref{fig:B:diagrama-interraccion}, donde el VNS trata de obtener una solución $x'$ para comparar con la inicial $x$. Para ello necesita da uso al entorno actual, que se obtiene a partir de la clase interfaz \texttt{NeighborhoodSet} (amarillo). Una vez obtenido el entorno, se emplea para buscar una solución mediante el envío del mentaje \texttt{bestImprovement()}, que se transmite hasta llegar al movimiento concreto, en este caso, \texttt{MovMaxCarga}. El resultado se transmite de nuevo de vuelta.

\begin{figure}[h]
	\centering
	\includegraphics[width=\linewidth]{diagrama-interraccion}
	\caption{Diagrama de interacción de ejemplo.}
	\label{fig:B:diagrama-interraccion}
\end{figure}

\begin{landscape}
	\includepdf[landscape=True]{diagrama-clases-completo}
\end{landscape}

	\blankpage

 	\include{anexos/AnexoC-Flame-Diagram/AnexoC}
	\blankpage
	
 	\graphicspath{{anexos/AnexoD-Tabla-Casos/recursos/}}

\section{Tabla resumen de los casos de prueba} \label{Anexo:tabla-casos}

A continuación se incluye la tabla mencionada en el \autoref{capitulo:5} que resume la mayor parte del contenido descrito en la \autoref{sec:5:def-casos} de dicho capítulo. 

\begin{landscape}
\includepdf[landscape=True]{tabla-casos}
\end{landscape}
	\blankpage
	
 	\graphicspath{{anexos/AnexoE-Ejemplo-Ajuste-Parametrico/recursos/}}

\section{Ejemplo de Ajuste Paramétrico: Caso 1} \label{Anexo:ejemplo-ajuste-parametrico}

	\blankpage
	
 	\graphicspath{{anexos/AnexoF-Recopilacion-Soluciones-Por-Fases/recursos/}}

\section{Recopilación de las Soluciones de cada Caso} \label{Anexo:recopilacion-soluciones-por-fases}

En este anexo recopilamos para cada caso las figuras que representan visualmente:
\begin{enumerate}
	\item La planificación inicial, previa a los cambios debido a la incidencia
	\item La solución inicial, resultado de la \faseuno{}.
	\item La solución final, resultado de la \fasedos{} empleando VNS. 
\end{enumerate}


\newpage
\subsection{Caso 1}

\begin{figure}[!h]
	\centering
	\includegraphics[width=\linewidth]{caso1/caso1-fase0}
	\caption{Planificación inicial del Caso 1}
	\label{fig:caso1-fase0}
\end{figure}

\begin{figure}[!h]
	\centering
	\includegraphics[width=\linewidth]{caso1/caso1-fase1}
	\caption{Solución inicial (\faseuno{}) para el Caso 1}
	\label{fig:caso1-fase1}
\end{figure}

\begin{figure}[!h]
	\centering
	\includegraphics[width=\linewidth]{caso1/caso1-fase2-vns}
	\caption{Solución final (\fasedos{}) para el Caso 1}
	\label{fig:caso1-fase2}
\end{figure}

\FloatBarrier
\newpage
\subsection{Caso 3}

La planificación inicial es la misma que la del Caso 1, cambiando el slot del momento del cambio.

\begin{figure}[!h]
	\centering
	\includegraphics[width=\linewidth]{caso3/caso3-fase0}
	\caption{Planificación inicial del Caso 3}
	\label{fig:caso3-fase0}
\end{figure}

\begin{figure}[!h]
	\centering
	\includegraphics[width=\linewidth]{caso3/caso3-fase1}
	\caption{Solución inicial (\faseuno{}) para el Caso 3}
	\label{fig:caso3-fase1}
\end{figure}

\begin{figure}[!h]
	\centering
	\includegraphics[width=\linewidth]{caso3/caso3-fase2}
	\caption{Solución final (\fasedos{}) para el Caso 3}
	\label{fig:caso3-fase2}
\end{figure}

\FloatBarrier
\newpage
\subsection{Caso 4}

\begin{figure}[!h]
	\centering
	\includegraphics[width=\linewidth]{caso4/caso4-fase0}
	\caption{Planificación inicial del Caso 4}
	\label{fig:caso4-fase0}
\end{figure}

\begin{figure}[!h]
	\centering
	\includegraphics[width=\linewidth]{caso4/caso4-fase1}
	\caption{Solución inicial (\faseuno{}) para el Caso 4}
	\label{fig:caso4-fase1}
\end{figure}

\begin{figure}[!h]
	\centering
	\includegraphics[width=\linewidth]{caso4/caso4-fase2}
	\caption{Solución final (\fasedos{}) para el Caso 4}
	\label{fig:caso4-fase2}
\end{figure}

\FloatBarrier
\newpage
\subsection{Caso 5}

La planificación inicial es la misma que la del Caso 4, cambiando el slot del momento del cambio.

\begin{figure}[!h]
	\centering
	\includegraphics[width=\linewidth]{caso5/caso5-fase0}
	\caption{Planificación inicial del Caso 5}
	\label{fig:caso5-fase0}
\end{figure}

\begin{figure}[!h]
	\centering
	\includegraphics[width=\linewidth]{caso5/caso5-fase1}
	\caption{Solución inicial (\faseuno{}) para el Caso 5}
	\label{fig:caso5-fase1}
\end{figure}

\begin{figure}[!h]
	\centering
	\includegraphics[width=\linewidth]{caso5/caso5-fase2}
	\caption{Solución final (\fasedos{}) para el Caso 5}
	\label{fig:caso5-fase2}
\end{figure}

\FloatBarrier
\newpage
\subsection{Caso 6}

La planificación inicial es la misma que la del Caso 4, cambiando el slot del momento del cambio.

\begin{figure}[!h]
	\centering
	\includegraphics[width=\linewidth]{caso6/caso6-fase0}
	\caption{Planificación inicial del Caso 6}
	\label{fig:caso6-fase0}
\end{figure}

\begin{figure}[!h]
	\centering
	\includegraphics[width=\linewidth]{caso6/caso6-fase1}
	\caption{Solución inicial (\faseuno{}) para el Caso 6}
	\label{fig:caso6-fase1}
\end{figure}

\begin{figure}[!h]
	\centering
	\includegraphics[width=\linewidth]{caso6/caso6-fase2}
	\caption{Solución final (\fasedos{}) para el Caso 6}
	\label{fig:caso6-fase2}
\end{figure}

\FloatBarrier
\newpage
\subsection{Caso 7}

\begin{figure}[!h]
	\centering
	\includegraphics[width=\linewidth]{caso7/caso7-fase0}
	\caption{Planificación inicial del Caso 7}
	\label{fig:caso7-fase0}
\end{figure}

\begin{figure}[!h]
	\centering
	\includegraphics[width=\linewidth]{caso7/caso7-fase1}
	\caption{Solución inicial (\faseuno{}) para el Caso 7}
	\label{fig:caso7-fase1}
\end{figure}

\begin{figure}[!h]
	\centering
	\includegraphics[width=\linewidth]{caso7/caso7-fase2-vns}
	\caption{Solución final (\fasedos{}) para el Caso 7}
	\label{fig:caso7-fase2}
\end{figure}

\FloatBarrier
\newpage
\subsection{Caso 8}

\begin{figure}[!h]
	\centering
	\includegraphics[width=\linewidth]{caso8/caso8-fase0}
	\caption{Planificación inicial del Caso 8}
	\label{fig:caso8-fase0}
\end{figure}

\begin{figure}[!h]
	\centering
	\includegraphics[width=\linewidth]{caso8/caso8-fase1}
	\caption{Solución inicial (\faseuno{}) para el Caso 8}
	\label{fig:caso8-fase1}
\end{figure}

\begin{figure}[!h]
	\centering
	\includegraphics[width=\linewidth]{caso8/caso8-fase2}
	\caption{Solución final (\fasedos{}) para el Caso 8}
	\label{fig:caso8-fase2}
\end{figure}

\FloatBarrier
\newpage
\subsection{Caso 9}

La planificación inicial es la misma que la del Caso 8, cambiando el slot del momento del cambio.

\begin{figure}[!h]
	\centering
	\includegraphics[width=\linewidth]{caso9/caso9-fase0}
	\caption{Planificación inicial del Caso 9}
	\label{fig:caso9-fase0}
\end{figure}

\begin{figure}[!h]
	\centering
	\includegraphics[width=\linewidth]{caso9/caso9-fase1}
	\caption{Solución inicial (\faseuno{}) para el Caso 9}
	\label{fig:caso9-fase1}
\end{figure}

\begin{figure}[!h]
	\centering
	\includegraphics[width=\linewidth]{caso9/caso9-fase2}
	\caption{Solución final (\fasedos{}) para el Caso 9}
	\label{fig:caso9-fase2}
\end{figure}


	\end{appendices}

    %	REFERENCIAS
    \newpage

	%
	%  BIBLIOGRAPHY
	%
    \normalem
    \bibliographystyle{ieeetr}
    \bibliography{master}

\end{document}

