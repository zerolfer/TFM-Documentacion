% !TeX spellcheck = es_ES

\documentclass[spanish,12pt, a4paper,twoside]{paper}

\let\oldsection\section
\def\section{\cleardoublepage\oldsection}

\usepackage{afterpage}

\usepackage[dvipsnames]{xcolor}

\newcommand\blankpage{%
\null
\thispagestyle{empty}%
\addtocounter{page}{-1}%
\newpage}

%\renewcommand{\listoftables}{Índice de tablas}
\newcommand{\refcruzada}[2]{\hyperref[#2]{#1~\ref{#2}}}

\newcommand*{\NOTE}[1]{\textcolor{ForestGreen}{#1}}

\renewcommand{\labelitemi}{$\square$}
\renewcommand{\labelitemii}{$-$}
\renewcommand{\labelitemiii}{$\bullet$}
\renewcommand{\labelitemiv}{$\ast$}


\makeatletter
\def\namedlabel#1{\begingroup
\def\@currentlabel{#1}%
\@currentlabel
\phantomsection\label{item:\@currentlabel}\endgroup
}
\makeatother


\usepackage[textwidth=15cm, textheight=22.5cm, top=3.5cm, bottom=3.5cm,left= 4cm,right=2cm]{geometry}


\usepackage[spanish]{babel}
\usepackage[utf8]{inputenc}

\usepackage{graphicx}
\usepackage{graphics}
\usepackage{subcaption}
\usepackage{amsmath,amssymb}
\usepackage{float}
\usepackage{changepage}
%\usepackage{subcaption} This package can't be used in cooperation with the subfigure package

\usepackage{enumitem}

\usepackage{algorithm}
\usepackage{multirow}

\usepackage[breaklinks=true]{hyperref}
\usepackage[T1]{fontenc}

\hypersetup{
	pdftitle={Metaheurística alternativa para el problema},
	pdfauthor={Sergio Flórez Vallina}
	backref,
	unicode=true,
	bookmarksnumbered=true,
	bookmarksopen=true,
	bookmarksopenlevel=2,
%	pdfborder={1 1 1},
	colorlinks=true,
%	hidelinks=true,
	%pdfpagemode=UseOutlines,    % this is the option you were lookin for
	%pdfpagelayout=TwoPageRight
%	colorlinks=false,% hyperlinks will be black
%	linkbordercolor=black,% hyperlink borders will be red
	linkcolor=MidnightBlue
%	pdfborderstyle={/S/U/W 1}% border style will be underline of width 1pt
}

\begin{document}
    %\maketitle
    %\thispagestyle{empty}
    \begin{titlepage}

        \newcommand{\HRule}{\rule{\linewidth}{0.5mm}} % Defines a new command for the horizontal lines, change thickness here

        \center % Center everything on the page

        %	HEADING SECTIONS
        \includegraphics[width=2.25cm]{recursos/logoFi.png}
        \hspace{8cm}
        \includegraphics[width=2cm]{recursos/logoupm.png}
        \\[1cm]

        \textsc{\Large Escuela Técnica Superior de Ingenieros Informáticos}\\[0.5cm]
        \textsc{\large Universidad Politécnica de Madrid}
        \\[3cm]


        %	TITLE SECTION
        \HRule \\[0.4cm]
        { \huge \bfseries serdfgvesrdfserdzfesrdgf}\\[0.4cm] % Title of your document
        \HRule \\[2.5cm]

        \textsc{\LARGE yudtryutyu6ydrujfghdtyju}\\[0.5cm]
        \textsc{\Large dtyjhgftyghujbft yrf ghjdry tjghf }\\[2.5cm]

        %	AUTHOR SECTION
        \begin{flushright}
            \large
            AUTOR: Sergio Flórez Vallina\\
            TUTORES: Alfonso Mateos Caballero y \linebreak
            Antonio Jiménez Martín
        \end{flushright}

        \vspace{1.3cm}

        %	DATE SECTION
        { {2019}}\\[3cm]
        %	LOGO SECTION

        \vfill
        % Fill the rest of the page with whitespace

    \end{titlepage}

    \afterpage{\blankpage}
    \pagenumbering{roman}


    %	AGRADECIMIENTOS
    \section*{AGRADECIMIENTOS}
    Agradezco especialmente a Tino Tello Caballo por toda su paciencia y tiempo para ayudarme
    a dar vida a éste proyecto.

    %	RESUMEN
    \section*{RESUMEN}
    Extensin mxima de una pgina


    %	SUMMARY
    \section*{SUMMARY}
    Extensión máxima de una página


    %	ÍNDICE
    \tableofcontents % indice de contenidos



    %	INDICE DE FIGURAS Y TABLAS
    \listoffigures

    \listoftables

    %	CAPTULOS DEL TRABAJO FIN DE MSTER
    \newpage
    \pagenumbering{arabic}

    %
    % EN ESTE DOCUMENTO ESTÁ EL RESTO DE LA PLANTILLA
    % \section{INFORMACIÓN SOBRE EL TFM}

\subsection{Asignación de Trabajo Fin de Máster}
\noindent El proceso de asignacin de Trabajo Fin de Máster, aprobado por la CAMIA en su novena reunión ordinaria de 15/12/2011, es el siguiente:
\begin{enumerate}
	\item Los alumnos pueden contactar con los profesores del MUIA y acordar el tema de su Trabajo Fin de Mster.
	\item A travs de una aplicacin informtica desarrollada por el DIA (manual de usuario), los alumnos pueden introducir sus preferencias sobre las propuestas de TFM que anualmente realizan por los profesores del MUIA (entre Diciembre y Enero). Identifican, si as lo desean, en orden hasta un mximo de 5 propuestas que ms le atraigan.
	\item En el caso de que no les atraiga ninguna oferta, o no se le haya asignado ninguna de las seleccionadas (varios alumnos pueden seleccionar la misma propuesta), el alumno deber realizar una propuesta, encuadrndola en una de las materias del MUIA e indicando hasta tres profesores de la misma que puedan ejercer de directores.
\end{enumerate}

En el siguiente enlace (http://www.dia.fi.upm.es/grupos-investigacion) se dispone de un listado de los grupos de investigacin, con una descripcin breve de los mismos y enlaces a sus correspondientes pginas web.

Los alumnos pueden identificar a partir de la informacin proporcionada por los grupos de investigacn la lnea en que basar el desarrollo de su TFM e incluso de una posterior Tesis Doctoral.

Se permitide un TFM por dos profesores, previa solicitud y justificac de la misma a la CAMIA, siendo obligatorio que al menos uno de los dos profesores forme parte del profesorado del ter.

La  asigne Trabajos Fin de Master se encuentra disponible en la web.



\subsection{Tribunal evaluador}
\noindent Se constituiun tribunal para cada defensa de TFM. El director del TFM formarte del tribunal y eleg miembros restantes, debiendo ser:
\begin{itemize}
	\item Uno de ellos, un profesor del MUIA de la materia del TFM.
	\item  El otro, un profesor del MUIA de la materia del TFM o de una mater.
\end{itemize}

En caso de codireccionesl visto bueno.


\subsection{Proceso administrativo de defensa del TFM}
\noindent La Figura \label{fig:proceso} muestra el proceso completo desde la asignacer (TFM) hasta su defensa.

\begin{figure}[h]
	\centering
	\includegraphics[width=0.65\textwidth]{recursos/Proceso}
	\caption{Proceso desde la asignahasta la defensa del TFM}
	\label{fig:proceso}
\end{figure}


El TFM {\bf puede matricularse en cualquier momento a lo largo del curmico} en la Secreta alumnos (ETSIInf), donde se generarcarta de pago.

El tiempo que puede transcurrir entre la matriculla defensa del TFM no  limitado (salvo los 7  naturales de antelaccircunscrito al mismo curso .

Es necesario tener en cuenta que al hacer el pago, el importe de la  que llegue a la Universidad tiene que coincidir exactamente con el de la carta de pago. Si no hay coincidencia no se  defender hasta que esa cantidad coincidiera, por lo que las comisiones bancarias o cargos correspondientes transferencias desde el extranjero, cambios de divisas, etc. los tiene que asumir el alumno. Una vez realizado el pago se debe entregar en el Centro de Postgrado (o bien enviarlos mediante un mail a centro.postgrado@fi.upm.es) lo siguiente:

\begin{itemize}
	\item el resguardo de la transferencia.
	\item y los datos siguientes:     
	\begin{itemize}
		\item Nombre y apellidos de la persona matriculada.
		\item Nombre del Master.
		\item Fecha de pago.
		\item Cantidad transferida.
		\item Cuenta desde la que se transfiere la cantidad.
	\end{itemize}
\end{itemize}

{\bf Nota:} El alumno debe tener en cuenta que si no  matriculado de ninguna asignatura en el MUIA pierde su  oficial con la UPM y no puede optar a becas oficiales y no oficiales,  Por ello, recomendamos a los alumnos que  tengan pendiente el TFM la matriculen al principio del semestre correspondiente para mantener la  con la universidad.

Las {\bf defensas} de los TFM se  realizar a lo largo de todo el curso

Una vez matriculada el TFM, el alumno  con una {\bf  de 7  naturales} la fecha y tribunal de la defensa, mediante la instancia correspondiente, en la {\bf  que se debe entregar es la siguiente:
	
	\begin{itemize}
		\item Instancia con tribunal y fecha de la defensa del TFM y  del director/es.
		\item Copia de la carta de pago de  del TFM.
		\item Instancia de  de cara a que el TFM pueda ser publicada en el archivo digital de la UPM.
	\end{itemize}
	a de la misma un ejemplar del TFM en el formato prescrito en formato  (pdf).
	
	El secretario del tribunal l encargado de {\bf reservar hemiciclo} para la  del acto de defensa del TFM y de {\bf recoger y entregar las actas} de la defensa en la  de Postgrado de la ETSIInf.
	
	\subsection{Acto de defensa del TFM}
	\noindent La {\bf lengua} tanto de la memoria del TFM, como de la defensa del mismo ante el tribunal,  ser el castellano o el .
	
	El secretario del tribunal s de Postgrado de la ETSIInf.
	
	La {\bf defensa del TFM}  oral sobre el misma por parte del alumno durante un {\bf tiempo  de 20 minutos y  de 20 minutos.
		
		El tribunal  los siguientes aspectos a la hora de evaluar el TFM:
		\begin{itemize}
			\item El alumno {\bf conoce}  de Inteligencia Artificial que le permiten abordar y solucionar problemas de .
			\item El alumno {\bf aplica}  existentes de la Inteligencia Artificial para la  de un problema.
			\item El alumno {\bf crea} alguna  de  de la Inteligencia Artificial.
			\item El alumno {\bf crea y difunde}  aceptados) los resultados de la TFM en una revista o congreso (nacional o internacional) con  por pares.
		\end{itemize}
		
		
		
		\subsection{Confidencialidad}
		\noindent En el caso de que el alumno desee la confidencialidad sobre su TFM,  solicitarlo mediante la correspondiente instancia disponible en la web que se  con una copia impresa del TFM y se  por Registro en  de Alumnos.
		
		
		\subsection{Concesión de Matriculas de Honor}
		\noindent Para proponer la  de Matrícula de Honor, se  en cuenta los criterios ya aprobados en la CAMIA de 15/12/2012: El alumno {\bf crea y difunde}  o  (nacional o internacional) con  por pares.
		
		formada por 3 profesores del Master Universitario en Inteligencia Artificial (MUIA) para la  aquellos profesores que hayan tutorizado alguna de los TFM propuesta para MH.
		
		Una vez finalizada la defensa de todos los trabajos de fin de  lugar en el mes de Julio. 
		
		La  solamente  en cuenta los TFM que hayan sido propuestas para MH por los respectivos tribunales.
		
		El tribunal otra convocatoria posterior.
		
		Si hubiese un  de alumnos matriculados (de conformidad con lo dispuesto en el Real Decreto 1125/2003, de 5 de septiembre), se  en cuenta las siguientes recomendaciones:
		
		\begin{itemize}
			\item Se  de honor obtenidas por el alumno en asignaturas del master.
		\end{itemize}
		
		
		\section{TABLAS, FIGURAS, EXPRESIONES MATEMÁTICAS Y ALGORITMOS}
		
		\subsection{Figuras}
		
		Las Figuras \ref{fig:Bernoulli1} y \ref{fig:violin_besa_escenario4} muestran ejemplos de  insertar figuras en el TFM.
		\begin{figure*}[htb]
			\centering
			\begin{subfigure}{0.5\textwidth}
				\includegraphics[width=\textwidth]{recursos/Figure1a}
				\caption{Mean cumulative regret along trials}
				\label{fig:Bernoulli1_semilog}
			\end{subfigure}
			\begin{subfigure}{0.5\textwidth}
				\includegraphics[width=\textwidth]{recursos/Figure1b}
				\caption{Multiple violinplot}
				\label{fig:Bernoulli1_boxplot}
			\end{subfigure}
			\caption{Comparative of the policies for scenario 1}
			\label{fig:Bernoulli1}
		\end{figure*}
		
		\begin{figure*}
			\centering
			\includegraphics[width=0.5\textwidth]{recursos/Figure2}
			\caption{Violinplot fot BESA in scenario 4}
			\label{fig:violin_besa_escenario4}
		\end{figure*}
		
		
		
		\subsection{Expresiones matemáticas}
		A continuación, se muestran algunos ejemplos de expresiones matemáticas:
		\begin{equation}
		\mu^*\times 25000-\frac{1}{1000}\sum_{r=1}^{1000}\sum_{i=1}^{K}\sum_{j=1}^{25000}\mu_i\times X_{i,j}^r.
		\end{equation}
		
		\begin{equation}
		\mu_{\widetilde{A}}(x)=\left\{ \begin{array}{cc}
		\frac{x-a_{1}}{a_{2}-a_{1}} & if\; a_{1}\leq x\leq a_{2}\\
		1 & if\; a_{2}\leq x\leq a_{3}\\
		\frac{x-a_{4}}{a_{3}-a_{4}} & if\; a_{3}\leq x\leq a_{4}\\
		0 & otherwise
		\end{array}\right. .
		\end{equation}
		
		
		\begin{equation}
		\begin{tabular}{ll}
		$\widetilde{DD}(A_{1},A_{4})$ & $=\widetilde{DD}(A_{1},A_{4}|P_{1})\oplus 
		\widetilde{DD}(A_{1},A_{4}|P_{2})$ \\ 
		$=[\widetilde{dd}(A_{1},A_{2})\otimes \widetilde{dd}(A_{2},A_{4})]\oplus \lbrack \widetilde{dd}(A_{1},A_{3})\otimes \widetilde{%
			dd}(A_{3},A_{4})].$%
		\end{tabular}%
		\end{equation}
		
		
		
		\begin{itemize}
			\item Si$\;{max} \{(a_{4}-a_{1}),(b_{4}-b_{1})\}\neq 0$, entonces
			\begin{eqnarray*}
				{\small S(}\widetilde{A}{\small ,}\widetilde{B}{\small )} &{\small =}&%
				\left. {\small 1-(1-\alpha -\beta })\times \left ( {\small 1-}\frac{\int_{0}^{1}%
					{\small \mu }_{\widetilde{A}\cap \widetilde{B}}{\small (x)dx}}{\int_{0}^{1}%
					{\small \mu }_{\widetilde{A}\cup \widetilde{B}}{\small (x)dx}}\right)
				\right.  \\
				&&\left. -{\small \alpha } \frac{\sum {\small \mid a}_{i}{\small -b}_{i}%
					{\small \mid }}{{\small 4}}-{\small \beta }\frac{{\small d[(X}_{\widetilde{A}%
					}{\small ,Y}_{\widetilde{A}}{\small ),(X}_{\widetilde{B}}{\small ,Y}_{%
						\widetilde{B}}{\small )]}}{{\small M}}\right., 
			\end{eqnarray*}
			
			\item En caso contrario,%
			\begin{eqnarray*}
				{\small S(}\widetilde{A}{\small ,}\widetilde{B}{\small )} &{\small =}&%
				\left. {\small 1-}%
				\left( \frac{{\small 1-\alpha -\beta }}{{\small 2}}{\small +\alpha } \right) \times
				\frac{\sum {\small \mid a}_{i}{\small -b}_{i}{\small \mid }}{{\small 4}}%
				{\small -}\right.  \\
				&&\left. {\small -}\left( \frac{{\small 1-\alpha -\beta }}{{\small 2}}%
				{\small +\beta }\right)\times \frac{{\small d[(X}_{\widetilde{A}}{\small ,Y}_{%
						\widetilde{A}}{\small ),(X}_{\widetilde{B}}{\small ,Y}_{\widetilde{B}}%
					{\small )]}}{{\small M}}\right., 
			\end{eqnarray*}
		\end{itemize}
		donde $\alpha +\beta <1$, $\mu _{\widetilde{\chi }}$ es la funcion de pertenencia de $\widetilde{\chi}$, 
		\begin{equation}
		M=\underset{[0,1]\times[0,\frac{1}{2}]}{max}\{d((x,y),(x^{\prime },y^{\prime }))\}\text{,} 
		\end{equation}%
		\begin{equation*}
		\mu _{\widetilde{A}\cap \widetilde{B}}(x)=\underset{0\leq x\leq 1}{min}%
		\{\mu _{\widetilde{A}}(x),\mu _{\widetilde{B}}(x)\} ,
		\;\;\; \mu _{\widetilde{A}\cup \widetilde{B}}(x)=\underset{0\leq x\leq 1}{max}%
		\{\mu _{\widetilde{A}}(x),\mu _{\widetilde{B}}(x)\}.
		\end{equation*}%
		
		\subsection{Algoritmos}
		
		El Algoritmo \ref{getDelay} ilustra la forma que debe adoptarse. 
		\begin{algorithm}[h]
			%\begin{algorithmic}
			{\bf  Data:} ($t_0$ = instante en el que se genera el retardo)
			\medskip
			
			\hspace{0.5em} {\bf if} $(update\_architecture==1)$ {\bf then} 
			
			\hspace{1.5em} {\bf if} $(delay\_scenario==1)$ {\bf then} delay$=C$
			
			\hspace{1.5em} {\bf else} 
			
			\hspace{2.5em} {\bf if} $(reward\_scenario==1)$ {\bf then} 
			
			\hspace{3.5em} delay $\leftarrow [0,300]$-trunc\_Exp($\lambda=1/80$)
			
			\hspace{2.5em} {\bf else} 
			
			\hspace{3.5em} delay $\leftarrow [0,480]$-trunc\_Exp($\lambda=1/150$)
			
			\hspace{2.5em} {\bf end if}
			
			\hspace{1.5em} {\bf end if}
			
			\hspace{0.5em} {\bf else} (arquitectura en modo batch)
			
			\hspace{1.5em} delay= difference(24:00, $t_0$)
			
			\hspace{0.5em} {\bf end if}
			
			\hspace{0.5em}  {\bf return} delay
			
			{\bf end} 
			\caption{$getDelay(t_0)$}
			\label{getDelay}
		\end{algorithm}
		
		
		\subsection{Tablas}
		Las Tablas \ref{table:results45} y \ref{table:risk} muestran el formato de tabla a utilizar.
		
		\begin{table}[htb]
			\centering
			\caption{Mean cumulative regrets and standard deviations}
			\label{table:results45}
			\begin{tabular}{llllll}
				\hline
				& \multicolumn{2}{c}{\small Truncated Poisson} &  & \multicolumn{2}{c}{\small Truncated Exponential} \\ 
				\cline{2-3}\cline{5-6}\cline{5-6}
				& {\small Mean} & ${\small \sigma}$ &  &  {\small Mean} & ${\small \sigma}$\\ \hline
				{\small UCB}      & {\small 2632.65} & {\small 246.03}  &  & {\small 1295.79} & {\small 514.03}   \\
				{\small DMED+}            & {\small 978.56} & {\small 225.24}  &  & {\bf \small645.70} & {\small 493.8}   \\
				{\small KL-UCB}   & {\small 1817.4} & {\small 236.57}  &  & {\small 1219.98} & {\small 510.69}   \\ 
				{\small KL-UCB poisson}    & {\bf \small314.99*} & {\small 201.79}  &  & {\small -} & {\small -}   \\
				{\small KL-UCB exp}    & {\small -} & {\small -}  &  & {\small 786.30} & {\small 498.16}   \\
				{\small KL-UCB+}    & {\small 1190.64} & {\small 225.82}  &  & {\small 813.45} & {\small 494.59}   \\
				{\small BESA}      & {\small 2015.73} & {\small 3561.5}  &  & {\small 755.87} & {\small 2323.22}   \\
				{\small PR-1}            & {\small 1314.9} & {\small 234.25}  &  & {\small 660.64} & {\small 492.37}   \\ 
				{\small PR-2 (TS)}  & ${\bf 917.67}$ & {\small222.79}  &  & {\bf \small630.38} & {\small487.01} \\
				{\small PR-3}  & ${\bf 736.6}$ & {\small210.96}  &  & {\bf \small565.79*} & {\small480.99} \\
				\hline
			\end{tabular}
		\end{table}
		
		\begin{table}[htb]
			\centering
			\caption{Risks to $A_5$ after the implementation of the selected safeguards }
			\label{table:risk}
			\begin{tabular}{cccc}
				\hline
				\noalign{\smallskip} 
				{\scriptsize{Threat}}& {\scriptsize{Confidentiality}} & {\scriptsize{Integrity}} & {\scriptsize{Authenticity}}\tabularnewline
				\hline  
				{\scriptsize{$T_{1}^{1}$}} & \scriptsize{(16.9, 161.72, 936.2, 3681.5)} & \scriptsize{(32.70, 239.7, 1295.6, 5197.4)} & \scriptsize{(25.1, 198.6, 1576.7, 5777.1)}\\
				{\scriptsize{$T_{1}^{2}$}} & \scriptsize{(0, 49.6, 458.1, 1791.2)} & \scriptsize{(0, 29.7, 289.7, 1397.1)} & \scriptsize{(0, 24.6, 352.6, 1552.9)}\\
				{\scriptsize{$T_{2}^{2}$}} & \scriptsize{(0, 49.6, 458.1, 1791.2)} & \scriptsize{(0, 29.7, 289.7, 1397.1)} & \scriptsize{(76, 379.3, 2074.3, 5588.4)}\\
				{\scriptsize{$T_{1}^{3}$}} & \scriptsize{(12.2, 110.5, 647.2, 2465.6)} & \scriptsize{(21.9, 147.3, 744.3, 2958.7)} & \scriptsize{(6.8, 58.5, 487.1, 1923.2)}\\ 
				{\scriptsize{$T_{1}^{4}$}} & \scriptsize{(34.8, 245.5, 1176.8, 3793.2)} & \scriptsize{(62.7, 327.4, 1353.3, 4551.9)} & \scriptsize{(19.5, 129.9, 885.7, 2958.7)}\\
				\hline 
			\end{tabular}
		\end{table}
		
		
		\section{CONTENIDOS DEL TFM}
		Durante la resultante de la tesis satisface los deseos o necesidades del cliente (real, potencial o ficticio).
		
		{\bf Conclusiones:}
		Establecer las conclusiones del trabajo  actuales aplicadas al problema, planteando leneas de I+D+i realistas y capaces de superarlos.
		
		\section{CONCLUSIONES Y LENEAS FUTURAS DE TRABAJO}
		Establecer las conclusiones del trabajo apoyándose fundamentalmente en los datos y observaciones obtenidas durante su desarrollo. Discutir que medios, cauces, etapas y tecnologías hartan falta (si procede) para llevar a cabo una implantación real de los resultados.
		Discutir los limites de las tecnologías actuales aplicadas al problema, planteando leneas de I+D+i realistas y capaces de superarlos.
		
		\section{SOBRE LAS REFERENCIAS}
		
		La bibliográfica o referencias deben aparecer siempre al final de la tesis, incluso en aquellos casos donde se hayan utilizado notas finales. La bibliográfica debe incluir los materiales utilizados, incluida la edición, para que la cita pueda ser fácilmente verificada. 
		
		\bigskip
		{\bf Citar dentro del texto:}
		
		Las fuentes consultadas se describen brevemente dentro del texto y estas citas cortas se amplían en una lista de referencias final, en la que se ofrece la información bibliográfica completa. 
		
		La cita dentro del texto es una referencia corta que permite identificar la publicación de donde se ha extraído una frase o parafraseado una idea, e indica la localización precisa dentro de la publicación fuente. Esta cita informa del apellido del autor, la fecha de publicación y la pagina (o paginas) y se redacta de la forma que puede verse a través de los siguientes ejemplos:
		
		Cuando se citan las palabras exactas del autor deben presentarse entre comillas e indicarse, tras el apellido del autor y, entre paréntesis, la fecha de publicación de la obra citada, seguida de la/s pagina/s.
		
		Si lo que se reproduce es la idea de un autor (no sus palabras exactas) no se ponrse; debe indicarse siempre con puntos suspensivos entre corchetes [...]
		
		Ejemplos de como citar una referencia en el texto son los siguientes \cite{Ashtiani2014} o \cite{Ashtiani2014,Mateos2009,Vicente2016}.
		
		
		\bigskip
		{\bf Como ordenar las referencias:}
		\begin{enumerate}
			\item Las referencias bibliográficas deben presentarse ordenadas alfabéticamente por el apellido del autor, o del primer autor en caso de que sean varios.
			\item Si un autor tiene varias obras se orde
			\item Si son trabajos de un autor en colabora de publicación. Las publicaciones individuales se colocan antes de las obras en colaboración.
		\end{enumerate}
		
		\bigskip
		{\bf Como citar un articulo de revista}
		
		Un articulo de revista, siguiendo las normas de la APA, se cita de acuerdo con el siguiente esquema general:
		Apellido(s), Iniciales del nombre o nombres. (Aulo.
		
		\bigskip
		{\bf Cmo citar una monografista/libro}
		
		Las monografistas, siguiendo las normas de la APA, se citan de acuerdo con el siguiente esquema general:
		Apellido(s), n cursiva.
		
		\bigskip
		{\bf Como citar un capitulo de un libro}
		
		Los c Editorial.
		
		\bigskip
		{\bf Cmo citar un acta de un congreso}
		
		Apellido(s), Iniciales del nombre o nombres. (A). Ttulo del trabajo. En A. A. Apellido(s) Editor A, B. B. Apellido(s) Editor B, y C. Apellido(s) Editor C (Eds. o Comps. et.), Nombre de los proceedings en cursiva (pp. xxx-xxx). Lugar de publicaci: Editorial.
		
		\bigskip
		{\bf Como citar tesis doctorales, trabajos fin de míster o proyectos fin de carrera}
		
		Apellido(s), Nombre. (Aro). Titulo de la obra en cursiva. (Tesis doctoral). Institución a académica en la que se presenta. Lugar.
		
		\bigskip
		{\bf Como citar un recurso de Internet}
		
		Los recursos disponibles en Internet pueden presentar una tipografía muy variada: revistas, , portales, bases de datos... Por ello, es muy difícil dar una pauta general que sirva para cualquier tipo de recurso.
		Como mínimo una referencia de Internet debe tener los siguientes datos:
		\begin{enumerate}
			\item Titulo y autores del documento.
			\item Fecha en que se )
		\end{enumerate}
		
		Veamos, a .
		
		Monografistas:
		Se emplea la misma forma de cita que para las monografistas en versión impresa. Debe agregar la URL y la fecha en que se consulta el documento
		
		de revistas:
		Se emplea la misma forma de cita que para los artículos de revista en  impresa. Debe agregar la URL y la fecha en que se  el documento.
		
		de revistas  que se encuentran en una base de datos:
		Se emplea la misma forma de cita que para los  de revista en  impresa, pero debe  el nombre de la base datos, la fecha en que se  el documento.
    %
    %

    \graphicspath{{capitulos/Capitulo1-Introduccion/recursos/}}

\section{Introducción y objetivos}

En éste capítulo describiremos el contexto principal y las hipótesis iniciales de las que parte el proyecto, así como una ligera introducción al problema bajo estudio en el presente proyecto de fin de máster.

\begin{figure}
	\centering
	\includegraphics[width=0.7\linewidth]{Figure1a}
	\caption{ figura de test}
	\label{fig:figure1a}
\end{figure}

    \newpage

	\graphicspath{{capitulos/Capitulo2-Definicion-del-problema/recursos/}}

\section{Definición del problema} \label{apartado:2}

Tal y como se ha introducido antes, el proyecto ABACO pretende automatizar el proceso de creación de un horario de trabajo para
los distintos controladores del espacio aéreo de forma que dada una sectorización del espacio aéreo, todos los sectores puedan ser
controlados.

\subsection{Sectores y sectorización}
En primera lugar, explicaremos brevemente cómo se divide el espacio aéreo del territorio español, cuyo organismo encargado de su gestión es AENA. Si bien la realidad es muy compleja, aquí unicamente describiremos una simplificación de la misma, omitiéndose detalles técnicos de aviación que no son necesarios para la implementación del sistema.
\\

El espacio aéreo mundial se encuentra dividido en \textit{FIR}s (\textit{Flight Information Region}), áreas del territorio sobre las que se mueven los diferentes aviones de cada compañía aérea de cada país, en la \refcruzada{Figura}{fig:fireuropa} puede verse graficamente los límites de cada región. En el caso de España, podemos ver que tiene control sobre 3 \textit{FIR}s: el de Barcelona, el de Madrid y el de Canarias, sin embargo, a nivel nacional, existen algunas subdivisiones denominadas \textit{Dependencias} (ya que dependen del \textit{FIR} en el que se encuentren), que permiten una mejor gestión del territorio:
\begin{itemize}
	\item Barcelona RutaE
	\item Barcelona RutaW
	\item Barcelona TMA ESTE
	\item Barcelona TMA NORTE
	\item Barcelona TMA OESTE
	\item Canarias ACC App
	\item Canarias ACC Ruta
	\item Madrid Ruta 1
	\item Madrid Ruta 2
	\item Madrid TMA NORTE
	\item Madrid TMA SUR
	\item Malaga App
	\item Palma TACC
	\item Sevilla TACC
	\item  Valencia TACC TMA
\end{itemize}

Algunos de ellos aparecerán en los casos de prueba del sistema del \refcruzada{Apartado}{apartado:5}
\begin{figure}
	\centering
	\includegraphics[width=1\linewidth]{capitulos/Capitulo2-Definicion-del-problema/recursos/FIR_europa}
	\caption{FIRs del la zona europea. Fuente: EUROCONTROL}
	\label{fig:fireuropa}
\end{figure}



	\newpage

%	%! Suppress = LineBreak
%! Suppress = LabelConvention
\graphicspath{{capitulos/Capitulo3-Metodologia-propuesta/recursos/}}

\section{Metodología propuesta} \label{capitulo:3}
En este capítulo se describe en detalle la metodología propuesta para resolver el problema planteado en la \hyperref[capitulo:2]{sección anterior}, entendiendo por metodología el conjunto de decisiones previas a la implementación tomadas con el fin de plantear una forma de resolver dicho problema.

La hipótesis de partida (\hyperref[H2]{H2}) plantea como punto de comienzo del proyecto el uso de una metaheurística con el fin de optimizar todos los parámetros del sistema, pero hemos de definir dichos parámetros antes de poder definir la metaheurística.

Para comenzar con el planteamiento de la metodología, podemos comenzar desde el punto de vista de la ingeniería: verlo como un sistema de caja negra que recibe una entrada y una salida.
El sistema debe poder corregir la planificación de controladores de ese día, por lo tanto, es claro que la entrada será esa planificación. Recordemos que el sistema \legacy{} será empleado por el personal del aeropuerto para realizar la planificación de forma automatizada, por lo que la entrada del sistema deberá tener un formato común con la salida del sistema \legacy{}.
Respecto a la salida, deberá ser de un formato comprensible por el personal gerente de los puestos de control.

Por último, resta detallar la parte más importante: la \textit{caja negra} propiamente dicha. Pues bien, como hemos dicho antes, en primer lugar el sistema recibirá una solución inicial, que deberá ser tratada de acuerdo con las contingencias habidas. Por ejemplo, ocurre un cambio de sectorización en mitad del turno que no estaba planificado. 
A partir del momento en el que se cierra debemos eliminar todas las apariciones de los sectores que se cierran y añadir los que se abren, pero no antes de dicho momento (más detalles en la \autoref{sec:3:inicializacion-soluciones}). 
Llamaremos al momento en el que sucede la incidencia como momento actual para simplificar, aunque lo habitual es que el sistema sea ejecutado varios minutos antes de que suceda la incidencia.

Una vez tratada la solución inicial, la metaheurística ya podrá dar comienzo sobre ella, buscando diferentes 
planificaciones alternativas (soluciones) y dando como resultado la mejor. 

%Con todo, el sistema tiene cuatro módulos:
%\begin{itemize}
%	\item Módulo de lectura de datos: lleva a cabo las tareas de lectura e inicialización de estructuras de datos.
%	\item Módulo de inicialización: inicializa la solución inicial de acuerdo con la(s) contingencias recibidas del módulo anterior
%	\item Módulo de búsqueda: lleva a cabo la búsqueda de una solución factible al problema.
%	\item Módulo de entrega de datos: lleva a cabo las tareas de escritura y trazabilidad de las soluciones.
%\end{itemize}

Sin entrar en detalles de implementación (para ello véase el \autoref{capitulo:4}), el sistema tiene, claramente, dos 
\textit{Fases}:
\begin{enumerate}[label={}]
	\item \label{Fase 1} Fase 1: Tratamiento de la solución
	\item \label{Fase 2} Fase 2: Resolución del problema
\end{enumerate}

En adelante, aludiremos a la fase del sistema que comprende la inicialización de la solución de entrada en función de las necesidades del caso concreto de incidencia que se produzca como \faseuno{} o \textit{Fase de Inicialización}. 
Mientras que la \fasedos{} o simplemente \textit{Metaheurística}, será la fase del sistema en la que se resolverá el problema propiamente dicho de acuerdo con las pautas establecidas en forma de restricciones y puntuaciones sobre la metaheurística.
En las próximas secciones definiremos cada una de estas Fases en detalle.

\subsection{Representación de las soluciones}
\label{sec:3:representacion-soluciones}

Antes de entrar en detalle, es importante explicar cómo se han representado las soluciones. Recordemos que estamos representado planificaciones, es decir una relación matricial de sectores con trabajadores a lo largo del tiempo que dura un turno. De forma que, dada una lista de controladores, en cada instante de tiempo tendremos un sector asignado, así como el tipo de puesto (planificador o ejecutivo).

Las filas de la matriz serán por tanto los controladores que tengamos a nuestra disposición así como los que añadamos dinámicamente para facilitar la inicialización y que serán eliminados en la \fasedos{}, mientras que las columnas de la matriz representarán el tiempo (véase \autoref{fig:3:ejemplo-distribucion-inicial}) desde el inicio del turno hasta el final del turno. Por lo que el tamaño de cada una dependerá de la instancia concreta del problema que estemos 
resolviendo.

El tiempo es una variable continua, que nos permitiría conocer el sector que controla un trabajador para un momento exacto del tiempo como por ejemplo las 8:29:17 (horas, minutos, segundos), una precisión del todo innecesaria en este problema, pero también difícil de representarlo en este tipo de problemas de \textit{timetabling}. Para poder 
representar el tiempo, debemos convertir la variable continua en discreta mediante el proceso denominado 
discretización: renunciamos a la precisión del tiempo fragmentándolo en intervalos de tiempo uniformes que llamaremos \textit{slots}, por ejemplo de 5 minutos cada uno.

\begin{figure}[htbp]
	\begin{subfigure}{\linewidth}
		\centering
		\includegraphics[width=\linewidth]{tiempo-continuo}
		\caption{Tiempo continuo}
		\label{fig:timepo-continuo}
	\end{subfigure}
	
	\begin{subfigure}{\linewidth}
		\centering
		\includegraphics[width=\linewidth]{tiempo-disccreto}
		\caption{Tiempo discreto con \textit{slots} de 5 minutos}
		\label{fig:timepo-disccreto}
	\end{subfigure}
	
	\caption{Ilustración de la discretización del tiempo}
\end{figure}

Esta discretización nos hace perder precisión, por lo que el tamaño del slot deberá ser el adecuado para no perder capacidad de representación en nuestras soluciones y por ende limitar espacio de búsqueda en exceso, lo cual podría desembocar en que una buena solución no pueda ser representada y por lo tanto no será contemplada por el sistema de búsqueda así que nunca se dará como solución.
En nuestro caso, hemos elegido un tamaño de slot de 5 minutos debido a que se trata del máximo común divisor de todas las restricciones numéricas del dominio del problema (véase la \autoref{sec:4:RD}). Los expertos de \gls{CRIDA} están satisfechos con este nivel de detalle.

En las representaciones realizadas (véase como ejemplo la \autoref{fig:3:ejemplo-distribucion-inicial}) se utilizan identificadores de tres letras en lugar del nombre completo del sector para abreviar y mantener el número de caracteres constante.
Se han añadido también colores para una mejor visualización.
Las letras en mayúscula (AAA-ZZZ) representan un trabajo en puesto de ejecutivo, mientras que las letras minúsculas (aaa-zzz) indican un trabajo en puesto planificador. Los descansos se representan mediante unos (111).
Hemos agrupado slots contiguos tanto de trabajo como de descanso idénticos, de manera que visualmente sea más cómodo de entender. Para que las soluciones aquí presentadas tengan validez final, deberíamos añadir indicadores de las horas de los cambios de puesto, sin embargo, para este documento esto no es realmente importante por lo que podemos omitirlo.


%\NOTE{ AÑADIR ESTRUCTURA PIRAMIDAL DE LAS SOLUCIONES (de cara a referenciarlo en el fitness \autoref{apartado:adaptacion-fitness})} % PUESTO EN EL APARTADO DE FITNESS

\begin{figure}[htbp]
	\centering
	\includegraphics[width=\linewidth]{Ejemplo-distribucion-inicial}
	\caption[Ejemplo de una solución inicial]{Ejemplo de una posible solución inicial. Constituida mediante el uso 
		de plantillas.}
	\label{fig:3:ejemplo-distribucion-inicial}
\end{figure}

\subsection{Fase 1: Inicialización de Soluciones} \label{sec:3:inicializacion-soluciones}

La fase de inicialización toma como entrada la planificación inicial: aquella planificada para el día y que ya no tiene validez debido a la incidencia; junto a los datos relativos a la incidencia, que son:

\begin{itemize}
	\item Hora a la que se produce la incidencia.
	\item Tipo de incidencia.
	\item Si la incidencia es por un cambio imprevisto de sectorización, la nueva sectorización.
	\item Si la incidencia es por una baja de un trabajador, hora de la baja y los datos del trabajador y, opcionalmente, hora del alta y datos del trabajador (puede ser el mismo u otro que no forme parte del turno inicial).
\end{itemize}

\begin{figure}[htbp]
	\centering
	\includegraphics[width=\linewidth]{Esquema-Fase-1-extendido}
	\caption{Diagrama de flujo del funcionamiento de la Fase 1}
	\label{fig:3:esquema-fase-1}
\end{figure}

Con esos datos, la \faseuno{} deberá convertir la planificación inválida en una \textit{solución inicial}, que 
emplearemos como punto de partida para el sistema de búsqueda inteligente que es la \fasedos{}. Para ello distinguimos dos tipos de tareas, las relativas a la incidencia por cambio de sector (pasos 1 y 2) y las relativas a las bajas y altas de los trabajadores (pasos 3 y 4). Los pasos pueden verse esquemáticamente en la \autoref{fig:3:esquema-fase-1}.
Adicionalmente, un quinto paso fue planteado para poder facilitar la tarea de la \fasedos{}, que consistía en reducir el número de controladores añadidos artificialmente en los pasos anteriores moviendo, heurísticamente, carga de trabajo a otros controladores que la soporten. Finalmente no fue implementada y fue añadida como trabajo futuro.

En la figura \autoref{fig:3:ejemplo-distribucion-inicial} se muestra cómo sería una posible planificación inicial. En este caso ha sido creada en base a \textit{plantillas} o \textit{estadillos}, que es el método empleado habitualmente por el personal para crear la planificación. Consiste en la repetición de un patrón conformado por tres 
controladores para un sector, en el que se suceden trabajo en puesto planificador, trabajo en puesto ejecutivo y descanso con un desfase en incremento para cada controlador, de forma que en cada instante de tiempo (imagínese una línea transversal) habrá un controlador en puesto ejecutivo, otro en planificador y otro descansado (véase  la \autoref{fig:3:plantilla-3x1}). \gls{CRIDA} sabe que el uso de estas plantillas si bien no es lo más óptimo es lo más cómodo tanto para la creación manual de la planificación como de cara a no incumplir las restricciones de cada trabajador (ver requisito). %TODO: referencias!!).

\begin{figure}
	\centering
	\includegraphics[width=0.9\linewidth]{capitulos/Capitulo3-Metodologia-propuesta/recursos/Plantilla-3x1}
	\caption[Aspecto de una plantilla 3x1]{Aspecto de una plantilla 3x1. Las letras mayúsculas representan trabajo en 
		puesto ejecutivo y las minúsculas en planificador.}
	\label{fig:3:plantilla-3x1}
\end{figure}

En las representaciones realizadas, se utilizan identificadores de tres letras en lugar del nombre completo del sector para abreviar y mantener el número de caracteres contante. Se han añadido también colores para una mejor visualización.
\textbf{Las letras en mayúscula (AAA-ZZZ) representan un trabajo en puesto de ejecutivo, mientras que las letras minúsculas (aaa-zzz) indican un trabajo en puesto planificador. Los descansos se representan mediante unos (111)}.
Hemos agrupado slots contiguos tanto de trabajo como de descanso idénticos, de manera que visualmente sea más cómodo de entender. Para que las soluciones aquí presentadas tengan validez final, deberíamos añadir indicadores de las horas de los cambios de puesto, sin embargo para este documento esto no es realmente importante por lo que podemos omitirlo.

El tipo de plantilla descrito se le denomina $3\times1$ (3 controladores para 1 sector) pero existen otros tipos como $8\times3$ o $4\times1$, no obstante, la más importante para el sistema es la $3\times1$, que será empleada durante esta Fase.

\subsubsection{Paso 1: Eliminar sectores que se cierran}
El primer paso es el encargado de eliminar los sectores que se cierran. Pongamos por ejemplo que tenemos una 
sectorización 5A que pasa a ser una 6C en un momento dado, tal y como se ilustra en la 
\autoref{fig:3:ejemplo-cambio-sectorizacion}. Como ya hemos dicho previamente, nosotros partimos de una planificación inicial como la representada en la \autoref{fig:3:ejemplo-distribucion-inicial}, que con la nueva sectorización queda totalmente inutilizada, pues podemos ver sectores que ya no se encuentran abiertos a partir del punto de cambio.

\begin{figure}[htbp]
	\centering
	\includegraphics[width=\linewidth]{Ejemplo-cambio-sectorizacion}
	\caption[Ejemplo de cambio de sectorización]{Ejemplo de un posible cambio de sectorización en la Unidad de Control 
	de Barcelona. En color aquellos sectores comunes a ambas sectorizaciones}
	\label{fig:3:ejemplo-cambio-sectorizacion}
\end{figure}


Identificamos pues, el momento de la incidencia a las 10:30, sin embargo el \textit{momento actual} viene dado a como parte de la entrada. En este caso, han decidido que sea media hora antes de la incidencia, a las 10:00 horas, que equivale al slot número 30:
\[ 
	10 \,h-7 \,h \,30 \,min = 2 \,h \,30 \,min = \left(2 \, \cancel{h} \times \frac{60 \,min}{1 \,\cancel{h}}\right) 
	\,min + 30\,min = 150 
	\,min 
\]

\[
	150 \,\cancel{min} \times \frac{1\,slot}{5\,\cancel{min}} = 30\,slots
\]

Antes de dicha hora, la planificación no debe ser alterada en ningún caso, pues representa el pasado. En la 
\autoref{fig:3:ejemplo-distribucion-inicial} la hemos representado con una línea roja. 

En el resto de la planificación, debemos eliminar todos los sectores que no aparecen. Para ello eliminamos aquellos que se cierran: los que dentro de la sectorización antigua, no pertenezcan a la nueva (una resta de conjuntos) (es decir, los que siguen en color negro en el cuadro azul de la \autoref{fig:3:ejemplo-cambio-sectorizacion}). 
Adicionalmente, para obtener una mejor solución inicial y favorecer así a la búsqueda, en el momento de eliminar un sector de la sectorización inicial, tratamos de sustituirlo por uno de los sectores nuevos que se abren (los de la nueva sectorización, los del cuadro naranja en color negro) \textbf{de forma que sean afines de entre sí}, pues el controlador seguirá pudiendo controlarlo sin problemas de acreditación. %FIXME es esto cierto?

Para hacer más eficiente el recorrido del algoritmo, en lugar de ir slot a slot, podemos agruparlos mientras la sectorización sea la misma. Buscamos un sector afín a cada sector que se cierra y lo sustituimos en todas las apariciones dentro de ese conjunto de slots. Así sucesivamente para cada tramo de slots con la misma sectorización. 
La búsqueda del sector afín es un algoritmo voraz que obtiene el primer sector de entre los que se abren que sea afín al que se cierra, evitando repeticiones.

\begin{algorithm}[H]
%	\SetAlgoLined
	\DontPrintSemicolon
	\KwData{
		
		$Sectorizacion_{inicial} = $ conjunto de sectores de la sectorización inicial para cada slot.
			
		$Sectorizacion_{modificada} = $ conjunto de sectores de la sectorización modificada para cada slot.
	}
	\medskip
	
	\ForEach{conjunto de slots con la misma sectorización}{
		$cerrados \leftarrow { Sectorizacion_{modificada}[slot] \setminus Sectorizacion_{inicial}[slot] }$\;
		$abiertos \leftarrow { Sectorizacion_{inicial}[slot] \setminus Sectorizacion_{modificada}[slot] }$\;
		
		\ForEach{$sector_c \in cerrados$}{
			$afin \leftarrow$ buscarPrimerAfin($Sectorizacion_{modificada}[slot]$)\;
			
			\If{$\nexists{afin}$}{
				$\forall$ aparición de $sector_c$, sustituir por descansos $(111)$\;
			} \Else{
				$\forall$ aparición de $sector_c$, sustituir por $afin$\;
			}
		}
			
	}
	
	\caption{Heurística de inicialización: AFINIDADES}
\end{algorithm}


En nuestro ejemplo, solo tenemos un único sector que se cierra, LECBGOI, que sustituiremos, mediante el algoritmo, por el sector LECBG12, que es el primero afín de entre los nuevos abiertos. Mediante esta heurística, evitaremos tener que añadir plantillas (ver \autoref{apartado:3:paso-2}) de todos los sectores nuevos reutilizando los controladores ya existentes.

\subsubsection{Paso 2: Introducir plantillas de los nuevos sectores} \label{apartado:3:paso-2}

Partimos de la lista de sectores nuevos que se abren y que no han sido ya empleados en el paso anterior como sustituto de alguno de los que se cierran. Lo que haremos será añadir a la planificación una plantilla $3\times1$ como la de la 
\autoref{fig:3:plantilla-3x1} donde alternamos trabajo y descanso a tamaños iguales: el doble de trabajo (uno en cada puesto) por cada uno de descanso. Las plantillas pueden emplearse con cualquier escala de tiempo, manteniendo las proporciones, por ejemplo 2 horas de trabajo y una de descanso.
Para este proyecto se ha utilizado un tamaño de 6 slots de descanso (12 de trabajo) debido a que los resultados eran mejores empleando esta proporción en proyectos previos a la presente tesis, por lo que se ha mantenido dicha proporción.

Para cada uno de los sectores mencionados añadiremos una de estas plantillas, con 3 controladores adicionales inexistentes que emplearemos auxiliarmente y que trataremos de eliminar en la \fasedos{}. En caso de no ser posible eliminar todos los auxiliares, diremos que para resolver el problema actual necesitamos obligatoriamente de un controlador extra.

De esta forma, obtendremos una planificación en la que se han tenido en cuenta las contingencias relativas a los cambios de sectorización. Si la instancia concreta del problema incluye únicamente esta incidencia, la planificación de la \autoref{fig:3:ejemplo-distribucion-pasos-1-y-2} sería una \textit{solución inicial} preparada para emplear como entrada a la \fasedos{}.

\begin{figure} 
	\centering
	\includegraphics[width=\linewidth]{Ejemplo-distribucion-pasos-1-y-2}
	\caption[Planificación tras los pasos 1 y 2 de la Fase 1]{Planificación tras los pasos 1 y 2 de la \faseuno{} siguiendo el ejemplo de la \autoref{fig:3:ejemplo-distribucion-inicial}. Los controladores con identificador C0 son imaginarios, es decir no existen pero son necesarios en la inicialización y deberán ser eliminador en la \fasedos{}} 
	\label{fig:3:ejemplo-distribucion-pasos-1-y-2}
\end{figure}

\subsubsection{Paso 3 y 4: Dar de baja/alta a los controladores}
Estos pasos son ejecutados únicamente en caso de que haya una incidencia relacionada con el personal. Podría suceder que tras ponerse de baja repentinamente, otro controlador cubra ese puesto a lo largo de la jornada, por lo que tendremos que hacer dos modificaciones a la planificación:

El controlador que se da de baja dejará de trabajar ese día, sin embargo no podemos eliminarlo de la planificación puesto que en momentos previos al \textit{momento actual} si ha trabajado, y hemos de contabilizarlo como carga de trabajo. Emplearemos el carácter especial ``000'' para indicar que se trata de un slot en el que el controlador no está trabajando pero tampoco descansado, al que se deberá tratar de forma especial durante la ejecución de la \fasedos{}, pues no se podrá mover a otro controlador ni se le podrá asignar nuevo trabajo. 

Si ningún trabajador se reincorpora a su puesto de trabajo, emplearemos un controlador imaginario al que se le asignará toda la carga de trabajo que tenía el controlador de baja a partir del momento de la incidencia.

La \autoref{fig:3:ejemplo-distribucion-pasos-3-y-4} muestra un ejemplo donde el controlador 23 se ha puesto de baja a las 9:30 (slot 30) y no hay reincorporación (en caso de haberla el controlador $C_0$ tendría su identificador correspondiente). Nótese que se emplean los caracteres de fuera de turno (``000'') en todos los slots a partir de la incidencia en el caso del controlador de baja mientras que para el controlador imaginario (o el reincorporado según se aplique) sucede lo contrario: todos los slots previos a la incidencia tienen el mencionado carácter. 
%TODO Se tratan de slots inalterables en todos los casos, pero a la hora de contabilizarse el trabajo (ver Fitness) %TODO referencia no se han de computar como tiempo de descanso

\begin{figure}
	\centering
	\includegraphics[width=\linewidth]{Ejemplo-distribucion-pasos-3-y-4}
	\caption[Ejemplo de planificación tras los pasos 3 y 4 de la \faseuno{}]{Ejemplo de planificación tras los pasos 3 y 4 de la \faseuno{}. En este caso no ha habido reincorporaciones}
	\label{fig:3:ejemplo-distribucion-pasos-3-y-4}
\end{figure}
%
%
%


\subsection{Fase 2: Metaheurística de optimización multiobjetivo} \label{sec:3:metaheurística}
Se trata del núcleo principal del presente proyecto, pues trata de dar solución al problema en sí mediante un enfoque de metaheurísticas.
\\

Como ya se ha introducido previamente, estamos ante un problema de \textit{timetabling/scheduling} que son generalmente problemas complejos debido a su naturaleza combinatoria. 
Mateméticamente se dice que pertenecen al conjunto de los problemas llamados \textit{NP-Duros}, pues los algoritmos clásicos empleados para resolverlos tienen una complejidad al menos de tipo exponencial. Clásicamente se han empleado por ejemplo los algoritmos 





se han tratado de resolver empleando diversas técnicas: inicialmente se emplearon heurísticas

\subsubsection{Búsqueda en Entornos Variables (VNS)}
test
\subsubsection{Adaptación del VNS al problema}
test
\paragraph{Función Fitness} 
test

\paragraph{Definiciones de entornos}
test

\paragraph{Búsqueda diversificada/intensificada}
test

\paragraph{Condiciones de Parada}
test












%	\newpage

	%\include{capitulos/CapituloXXX-YYY/CapituloXXX}
    %\newpage

    \section*{ANEXOS}


    %	REFERENCIAS
    \newpage

%    \begin{thebibliography}{00}
        \bibliographystyle{splncs04}
        \bibliography{master}

%    \end{thebibliography}
\end{document}

