\graphicspath{{capitulos/Capitulo1-Introduccion/recursos/}}


\section{Introducción}

Este Trabajo de Fin de Máster nace como continuación del proyecto ABACO, iniciado en 2013 por la Universidad Politécnica de Madrid en
colaboración con el Centro de Referencia de Investigación, Desarrollo e Innovación ATM (CRIDA \url{https://crida.es}), que pretende la
automatización del proceso de creación de la planificación de los turnos de los trabajadores que controlan el espacio 
aéreo en sus puestos de control de los aeropuertos españoles.

En este capítulo se describe el contexto principal y los objetivos e hipótesis iniciales de las que parte el presente Trabajo de Fin de Máster, así como una breve introducción al problema bajo estudio.
En el \autoref{capitulo:2}, se describirá el problema en profundidad. En el \autoref{capitulo:3} se expondrá la metodología propuesta para su resolución. En el \autoref{capitulo:4} se tratará la implementación del sistema.
El \autoref{capitulo:5} se centrará en los resultados experimentales obtenidos; y, finalmente, el \autoref{capitulo:6} mostrará las conclusiones y líneas futuras de trabajo.

\subsection{Objetivos del proyecto}
\label{sec:Objectivos}
El proyecto \gls{ABACO} es realmente grande, y continua en constante evolución, pasando por las manos de 
diferentes alumnos tanto del máster como del doctorado en Inteligencia Artificial impartidos por la Universidad Politécnica de Madrid, algunos de esos trabajos se encuentran citados a lo largo de esta memoria. 

Este trabajo pretende continuar el proyecto llevándolo un nivel superior: hasta ahora el sistema resolvía únicamente el problema de conformar por completo la distribución del personal. 
Sin embargo, la empresa necesita en algunas ocasiones reescribir parte de la planificación debido a una incidencia, por ejemplo, la baja repentina de uno de los trabajadores. Por lo tanto, este nuevo problema consiste en resolver parte de la solución anterior, reescribiendo únicamente aquella parte de la planificación que pertenezca al futuro, manteniendo lo anterior como constante. 
Por supuesto, para conformar la nueva solución, se ha de considerar en todo momento la parte fija. El problema se describe en detalle en el \autoref{capitulo:2}.

\NEW{Además, debido a los requisitos del nuevo sistema, trataremos también de mejorar el rendimiento general del mismo, modificando ciertas partes del software anterior.}

\subsection{Hipótesis iniciales del proyecto}
\label{sec:Hipotesis}
\begin{enumerate}[label={H\arabic*}]
	\item \label{H1} Es posible implementar las alteraciones, modificaciones y extensiones al sistema en un tiempo máximo de 7 meses, de forma que cumpla todos los requisitos del mismo y resuelva el problema dado (véase la \autoref{sec:4:RD}).
	\item \label{H2} El empleo de la metaheurística \vns{} (VNS) mejora el rendimiento%
	\footnote{\NEW{Entiéndase como el coste computacional en unidades de tiempo del sistema en su totalidad, y no de la metaheurística de forma aislada.}}
	del sistema en comparación con \sa{} (SA).
	\item \label{H3} \NEW{Es posible mejorar la eficiencia del sistema, cuantificada en tiempo de ejecución y consumo de memoria RAM, mediante la modificación de algoritmos y estructuras de datos reemplazándolos por otros más eficientes.}
\end{enumerate}

%
%   GLOSSARY
%
\glsaddall
\printglossary[title={Definiciones, acrónimos y abreviaturas}, nonumberlist]
%
%
%