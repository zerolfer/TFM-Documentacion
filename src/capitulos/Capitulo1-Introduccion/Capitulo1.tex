\graphicspath{{capitulos/Capitulo1-Introduccion/recursos/}}


\section{Introducción}

Este proyecto nace como continuación del proyecto ABACO, iniciado en 2013 por la Universidad Politécnica de Madrid en
colaboración con la empresa de innovación en el tráfico aéreo CRIDA\footnote{\url{https://crida.es}}, que pretende la
automatización del proceso de creación de la planificación de los turnos de los trabajadores que controlan el espacio 
aéreo en sus puestos de control de los aeropuertos españoles.
\\

En ésta sección se describe el contexto principal y los objetivos e hipótesis iniciales de las que parte el proyecto, 
así como una ligera introducción al problema bajo estudio en el presente Proyecto de Fin de Máster. Por su parte, en el 
\autoref{apartado:2}, se describirá el problema en profundidad; en el \autoref{apartado:3}
la metodología propuesta para su resolución; en el \autoref{apartado:4} los detalles de implementación;
en el \autoref{apartado:5} los resultados experimentales obtenidos; y, finalmente, en el \autoref{apartado:6} las 
conclusiones y trabajo a futuro. También \NOTE{completar con los anexos} % TODO: Complemtar con los anexos

\subsection{Objetivos del proyecto}
\label{sec:Objectivos}
El proyecto \gls{ABACO} es realmente grande, y continua en constante evolución, pasando por las manos de 
diferentes alumnos
tanto del máster como del doctorado en Inteligencia Artificial impartidos por la Universidad, algunos de esos trabajos 
se encuentran citados a lo largo de este documento. 
Por ello, este trabajo pretende continuar el proyecto llevándolo un nivel más allá: hasta ahora el sistema resolvía 
unicamente el problema de conformar por completo la distribución del personal, sin embargo, la empresa necesita en 
algunas ocasiones reescribir parte de la planificación debido a una incidencia, por ejemplo la baja repentina de uno de 
los trabajadores, por lo tanto éste nuevo problema consiste en resolver parte del problema, reescribiendo unicamente 
aquella parte de la planificación que pertenezca al futuro, manteniendo lo anterior como constante. Por supuesto, para 
conformar la nueva solución, se ha de considerar en todo momento la parte fija. El problema se describe en detalle en 
la \autoref{apartado:2}.
\\

Además, debido a los requisitos del nuevo sistema, trataremos también de mejorar el rendimiento general del sistema,
modificando ciertas partes del software anterior para lograr mejor rendimiento.

\subsection{Hipótesis iniciales del proyecto}
\label{sec:Hipotesis}
\begin{enumerate}[label={H\arabic*}]
    \item[\namedlabel{H1}]  Es posible implementar las modificaciones y extensiones al sistema en un tiempo máximo
    \item de 7 meses, de forma que cumpla todos los requisitos del mismo y resuelva el problema dado %TODO referenciar el apartado de requisitos
    \item[\namedlabel{H2}]  El empleo de la metaheurística VNS mejora el rendimiento neto
    \footnote{Entiéndase como el coste computacional en unidades de tiempo aislado de la metaheurística, no del sistema en su totalidad}
    del sistema en comparación con el SA
    %\item[\namedlabel{H3}]  Custom item label for entry three
\end{enumerate}


%\ref{item:H2}

%\begin{figure}
%	\centering
%	\includegraphics[width=0.7\linewidth]{Figure1a}
%	\caption{figura de test}
%	\label{fig:figure1a}
%\end{figure}

%\subsection{Glosario}
%\label{sec:Definiciones}
%\begin{description}
%    \item[ACACO] \label{ACABO} Nombre del proyecto
%    
%    
%    \item[ACC (Centro de Control)] \label{ACC}
%    <<Centro de control de tránsito aéreo responsable de la circulación aérea segura a lo largo de las rutas ATS 
%(servicio de tránsito aéreo). Un ACC se divide en varios sectores, cada uno de los cuales tiene claramente definidas 
%sus responsabilidades. Los procedimientos para transferir una aeronave de un sector a otro entre estados limítrofes 
%están perfectamente definidos por acuerdos internacionales, así como bilaterales>>.~\cite{ENAIRE-web}
%
%    \item [ATC (Control de Tránsito Aéreo)] \label{ATC} <<Término común que designa todos los servicios proporcionados 
%para asegurar y acelerar el flujo de tráfico aéreo a través del espacio aéreo controlado>>.~\cite{ENAIRE-web}
%
%    \item [CRIDA] \label{CRIDA} \textit{acrón.} Centro de Referencia de Investigación, Desarrollo e Innovación 
%    en \gls{ATC}
%    %Gestión del Tráfico Aéreo}
%    . Agrupación de interés económico (A.I.E.) sin ánimo de lucro 
%    establecida por ENAIRE, la Universidad Politécnica de Madrid (UPM) e Ingeniería y Economía del Transporte, S.A. 
%    (INECO).~\cite{CRIDA-web}.
%    
%    
%
%    \item [Núcleo] \label{Nucleo} Conjunto de Sectores. Un sector puede pertenecer a más de un núcleo (relación N a 
%N). Esta agrupación se lleva a cabo para poder gestionar los posibles sectores que cada controlador puede controlar, 
%de 
%esta forma, un controlador tiene acreditación para un único núcleo.
%
%    \item[\namedlabel{Matriz de Afinidad}] Tabla o matriz booleana cuyas filas y columnas son los diferentes sectores 
%de una Unidad de Control dada, de forma que la intersección de dos sectores tendrá el valor de Cierto si y solo si los 
%sectores son afines entre sí (relación bidireccional)
%
%    \item[Proyecto Airport 2050+] \label{AIRPORT} Proyecto europeo con colaboración española por parte de UPM, CRIDA e 
%    INECO
%   
%    \item[TMA] \label{TMA} Área de control terminal. <<Espacio aéreo controlado en torno a uno o varios aeropuertos 
%    donde se realizan las maniobras de aproximación (aterrizajes y despegues)>>~\cite{ENAIRE-web}.

   
    \glsaddall
    \printglossary[title={Definiciones, acrónimos y abreviaturas}]
%    \glossarysection[Glosarios]
%    
%    
%\end{description}