\subsection{Fase 2: Metaheurística de optimización multiobjetivo} \label{sec:3:metaheurística}
Se trata del núcleo principal del presente proyecto, pues trata de dar solución al problema en sí mediante un enfoque de metaheurísticas.
\\

Como ya se ha introducido previamente, estamos ante un problema de \textit{timetabling/scheduling} que son generalmente problemas complejos debido a su naturaleza combinatoria. 
Mateméticamente se dice que pertenecen al conjunto de los problemas llamados \textit{NP-Duros}, pues los algoritmos clásicos empleados para resolverlos tienen una complejidad al menos de tipo exponencial. Clásicamente se han empleado por ejemplo los algoritmos 





se han tratado de resolver empleando diversas técnicas: inicialmente se emplearon heurísticas

\subsubsection{Búsqueda en Entornos Variables (VNS)}
test
\subsubsection{Adaptación del VNS al problema}
test
\paragraph{Función Fitness} 
test

\paragraph{Definiciones de entornos}
test

\paragraph{Búsqueda diversificada/intensificada}
test

\paragraph{Condiciones de Parada}
test










