\graphicspath{{capitulos/Capitulo2-Definicion-del-problema/recursos/}}

\section{Definición del problema} \label{apartado:2}

Tal y como se ha introducido antes, el proyecto ABACO pretende automatizar el proceso de creación de un horario de trabajo para
los distintos controladores del espacio aéreo de forma que, dada una sectorización de este, todos los sectores puedan ser
controlados siguiendo las pautas establecidas por el dominio del problema.

El control del espacio aéreo (también conocido como ATC, \textit{Air Trafic Control}) es una tarea que se lleva a cabo
en todos los aeropuertos con el fin de monitorizar los diferentes aviones que sobrevuelan una determinada zona del cielo,
de cara a garantizar la seguridad de sus rutas (lo que se denomina control de ruta), así como de sus aterrizajes
(que se llama control de aproximación o de área terminal), encargándose también de las comunicaciones de voz tanto
tierra-aire con los pilotos de las aeronaves (vía radio), como tierra-tierra con otros controladores u otro personal
de gestión (vía telefónica)~\cite{ENAIRE-web}.
La zona de trabajo de los controladores aéreos se denomina Centros de Control de Tráfico Aéreo, cuyos puestos de 
trabajo tienen un aspecto similar al de la \autoref{fig:2:enaire-atc}.

\begin{figure}[htbp]
    \centering
    \includegraphics[width=0.7\linewidth]{ENAIRE-ATC}
    \caption{Fotografía de uno de los puestos de control, como puede verse está conformado por dos personas. Fuente: \url{https://muia.ml/wzfw8}}
    \label{fig:2:enaire-atc}
\end{figure}


\subsection{Organización del espacio aéreo}
\label{section:2:sectores-y-sectorizacion}
Cada controlador tiene asignado durante un cierto intervalo de tiempo de su turno de trabajo, un sector que debe controlar.
Por ello, en primer lugar, explicaremos brevemente cómo se divide el espacio aéreo del territorio español, cuyo organismo
encargado de su gestión es precisamente ENAIRE
\footnote{\url{https://www.enaire.es/sobre_enaire/conoce_enaire/quienes_somos_ENAIRE}}.
Si bien la realidad es muy compleja, aquí únicamente describiremos una simplificación de la misma, omitiéndose
detalles técnicos de aviación que no son necesarios para la implementación del sistema.
\\

El espacio aéreo mundial se encuentra dividido en \textit{FIR}s (\textit{Flight Information Region}), áreas del
territorio sobre las que se mueven los diferentes aviones de cada compañía aérea de cada país, en la
\autoref{fig:2:fireuropa} puede verse gráficamente los límites de cada región. En el caso de España,
podemos ver que tiene control sobre 3 \textit{FIR}s: el de Barcelona, el de Madrid y el de Canarias, sin embargo,
a nivel nacional, existen algunas subdivisiones denominadas \textit{Dependencias} (ya que dependen del \textit{FIR}
en el que se encuentren), que permiten una mejor gestión del territorio. Las dependencias están constituidas por Centros de Control (\hyperref[ACC]{ACC}) (uno por cada Dependencia) y Torres de Control. Algunos de ellos aparecerán en los casos reales de prueba del sistema del \autoref{apartado:5}, así que los enumeramos a continuación:

\begin{figure}[htbp]
    \centering
    \includegraphics[width=0.7\linewidth]{Division-Espacio-Nacional}
    \caption{Simplificación de la división del espacio aéreo nacional}
    \label{fig:2:regiones}
\end{figure}

\begin{itemize}
    \item Barcelona
    \begin{itemize}
        \item Barcelona RutaE
        \item Barcelona RutaW
        \item Barcelona TMA\footnote{Ver definición en la \hyperref[TMA]{\autoref{sec:Definiciones}}} ESTE
        \item Barcelona TMA NORTE
        \item Barcelona TMA OESTE
    \end{itemize}
    \item Canarias
    \begin{itemize}
        \item Canarias ACC App
        \item Canarias ACC Ruta
    \end{itemize}
    \item Madrid
    \begin{itemize}
        \item Madrid Ruta 1
        \item Madrid Ruta 2
        \item Madrid TMA NORTE
        \item Madrid TMA SUR
    \end{itemize}
    \item Malaga App
    \item Palma TACC
    \item Sevilla TACC
    \item  Valencia TACC TMA
\end{itemize}


\begin{figure}[htbp]
    \centering
    \includegraphics[width=\linewidth]{FIR_europa}
    \caption{FIRs de la zona europea. Fuente: EUROCONTROL}
    \label{fig:2:fireuropa}
\end{figure}

Cada una de estas Dependencias, está constituida por un cierto número de sectores, que se agrupan de forma que se cubra
todo el espacio de la zona, conformando lo que llamaremos una \textit{configuración o sectorización} concreta.
Existen varias configuraciones estandarizadas, de manera que dado un número de sectores, el espacio aéreo sea cubierto en su totalidad.
Por eso, las sectorizaciones se nombran utilizando dos caracteres: un número y una letra; el número indica precisamente
la cantidad de sectores de la configuración, mientras que la letra permite diferenciar entre dos posibles configuraciones
que emplean el mismo número de sectores.
\\

Por ejemplo, la configuración 3B de la dependencia MADRID Ruta 1 consta de los sectores LECMSAI, LECMBLI y LECMDPI,
mientras que una 3D consta de LECMSAN, LECMASI y LECMBDP. En ambos casos se utilizan tres sectores y el espacio cubierto
total es equivalente, como puede apreciarse en la \autoref{fig:2:comparativa3B-3D}.
De la misma forma, si utilizamos una configuración 5A, se utilizarían 5 sectores diferentes, pero el espacio aéreo a
controlar sería nuevamente equivalente, como se aprecia en la \autoref{fig:2:sectorizacion-5a}.
Nótese además, que los sectores LECMSAN (amarillo) y LECMASI (verde) son los mismo que los de la sectorización 3D (\autoref{fig:2:sectorizacion-3d})
pero el sector LECMBDP (azul) se ha dividido o \textit{abierto} en los sectores LECMBLI, LECMDGI y LECMPAI.
\\

\begin{figure}[htbp]
    \centering
    \begin{subfigure}{\linewidth}
        \centering
        \includegraphics[width=0.6\linewidth]{sectorizacion-3B}
        \caption{sectorizacion 3B\linebreak}
        %	\label{fig:sectorizacion-3b}
    \end{subfigure}

    \begin{subfigure}{\linewidth}
        \centering
        \includegraphics[width=0.6\linewidth]{sectorizacion-3D}
        \caption{sectorizacion 3D\linebreak}
        \label{fig:2:sectorizacion-3d}
    \end{subfigure}

    \begin{subfigure}{\linewidth}
        \centering
        \includegraphics[width=0.6\linewidth]{sectorizacion-5A}
        \caption{sectorizacion 5A}
        \label{fig:2:sectorizacion-5a}
    \end{subfigure}

    \caption{Ejemplo de comparativa de una sectorización 3B y 3D de la dependencia MADRID Ruta 1}
    \label{fig:2:comparativa3B-3D}
\end{figure}


Cuando cambiamos de una configuración de partida a otra, nos encontramos 2 posibles situaciones: que pasemos de una sectorización
con un menor número de sectores a otra con mayor (como en el ejemplo anterior) donde se \textit{abrirán} sectores; o por el
contrario que pasemos de una sectorización con un mayor número de sectores a otra menor (por ejemplo, el caso contrario: de una 5A a una 3D)
donde se \textit{cerrarán} sectores, pero como hemos visto, algunos serán comunes y no sufrirán cambio.
\\

Existen dos tipos de sectores: Ruta y Aproximación (APP), que tienen que ver con el tipo de tareas que los controladores llevan a cabo.
Los sectores de tipo Ruta precisan de tareas de control de ruta: garantizar la seguridad en las rutas de las naves aéreas; mientras que los sectores de tipo Aproximación precisan de tareas de aproximación o de área terminal: principalmente la gestión de los aterrizajes en el aeropuerto.
\\

El siguiente concepto importante es el de afinidad entre sectores. Sin entrar en demasiado detalle, la unidad mínima de
división del espacio aéreo son los volúmenes, de esta forma los sectores subirán un determinado número de volúmenes.
Pues diremos que dos sectores \textit{son afines entre sí} si comparten al menos un volumen entre sí. Para saber si dos
sectores son afines, utilizaremos la llamada \textit{matriz de afinidad} que, si bien puede ser construida manualmente,
en nuestro caso será tratada como un input del sistema para ahorrar tiempo innecesario de cómputo, pues es constante 
para cada dependencia (en el Apartado XXX se muestra un ejemplo). %TODO referencia!!
\\

Por último, resta introducir el concepto de \textit{núcleo}. Un núcleo no es más que una agrupación de alto nivel de
varios sectores, pudiendo un sector pertenecer a varios núcleos (relación N a N). El uso de los núcleos facilita la
gestión de la acreditación los controladores aéreos, pudiendo estos controlar únicamente un determinado núcleo (ver Requisito XXX.) %TODO referencia


\NOTE{Hasta aquí gramática revisada} %TODO eliminar nota  
%_                                                                                                                      
% 

\subsection{Gestión de tráfico aéreo}
La gestión del tráfico aéreo (\hyperref[ATC]{ATC}) se desempeña en las salas de control por un conjunto de trabajadores 
llamados controladores aéreos. Los controladores trabajan a lo largo de un turno de una duración determinada, que puede 
ser de Mañana, Tarde o Noche, con la peculiaridad de que los de Mañana y Tarde pueden ser de tipo Largo o Corto. Cuando 
el turno es Largo quiere decir que se toma parte del turno de noche, extendiendo así el periodo laboral (se ilustra un 
ejemplo en la \autoref{fig:2:tipos-turnos}). Cada \hyperref[ACC]{Centro de Control} tiene sus propios horarios 
predefinidos para cada uno de los turnos.
\\

\begin{figure}
    \centering
    \includegraphics[width=\linewidth]{tipos-turnos}
    \caption{Esquema que representa los tipos de turnos. Fuente:~\cite{articulo1}}
    \label{fig:2:tipos-turnos}
\end{figure}

Por otro lado, en cada puesto de control debe haber dos controladores, cada uno en una posición diferente: 
\textit{ejecutivo} y \textit{planificador}. El controlador en posición planificador es responsable de anticiparse a 
posibles problemas y avisar al otro controlador (en posición ejecutivo) antes de que éstos lleguen a producirse. Por su 
parte, el ejecutivo es quien debe llevar a cabo la tarea principal así como solucionar los conflictos que el 
planificador le comunique. La fotografía de la \autoref{fig:2:enaire-atc} muestra cómo es uno de estos puestos de 
trabajo, con un controlador de cada tipo.
\\

Como ya se comentó en el apartado anterior, cada controlador tiene una acreditación que le permite controlar un 
\hyperref[Nucleo]{núcleo} específico, y por ende, únicamente podrá trabajar en puesto que impliquen el control del 
tráfico aéreo de un determinado conjunto de sectores, lo que equivaldría a una sección específica del espacio aéreo 
total.
\\

Adicionalmente, en vista de las diferentes tareas a realizar por los controladores en función del tipo de sectores 
acreditados, parece lógico pensar que exista acreditación adicional, que aumenta el nivel de restricción de los 
sectores que puede controlar en función del tipo que estos sean. Por ello, hay dos tipos de acreditaciones, que tienen 
el nombre de PTD y CON. Los controladores con acreditación CON, únicamente podrán controlar los sectores de tipo Ruta, 
mientras que los de tipo PTD pueden controlar tanto Ruta como Aproximación (ver 
\autoref{table:2:acreditaciones}).

\begin{table}[h]
    \centering
    \caption{Tipos de sectores que se pueden controlar según el tipo de acreditación de un controlador}
    \begin{tabular}{lcc}
    	\hline
    	\textbf{Tipo acreditación} & \textbf{Sectores tipo Ruta} & \textbf{Sectores tipo CON} \\ \hline
    	PTD                        &        $\checkmark$         &        $\checkmark$        \\
    	CON                        &        $\checkmark$         &                            \\ \hline
    \end{tabular}
    \label{table:2:acreditaciones}
\end{table}

La configuración de sectores en cada instante de tiempo puede variar en función del flujo del tráfico aéreo zonal, por 
lo que no puede conocerse con exactitud previamente. No obstante, puede ser predicho con antelación, de manera que 
se pueda realizar la asignación de controladores a puestos previamente.
Es en este punto donde diverge el sistema existente con el propuesto en la presente tesis. Más detalladamente en la 
\autoref{apartado:2:detalles-sistema}.
\\

Supongamos el ejemplo de la \autoref{fig:2:ejemplo-apertura-sectorizaciones}, en la que se muestra una posible 
sectorización predicha 
con antelación por el personal del aeropuerto. Podemos apreciar los conceptos definidos anteriormente: el turno

\subsection{Apreciación de la diferencia del sistema frente al inicial}
\label{apartado:2:detalles-sistema}
Más detalladamente: el sistema previo (en adelante \textit{legacy}) resolvía el problema de asignación de controladores 
con antelación previa, por lo que los requerimientos de software no eran estrictos, esto es, el tiempo no era un factor 
crítico (si bien importante en todo sistema de optimización inteligente). El sistema podía permanecer en segundo plano 
durante horas para lograr alcanzar una solución válida (ya sea una óptima o simplemente factible) 


\begin{figure}[htbp]
    \centering
    \includegraphics[width=\linewidth]{ejemplo-apertura-sectorizaciones}
    \caption{}
    \label{fig:2:ejemplo-apertura-sectorizaciones}
\end{figure}













