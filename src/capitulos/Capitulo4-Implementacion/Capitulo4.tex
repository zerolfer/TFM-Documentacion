
\section{Implementación} \label{capitulo:4}

\NOTE{CONTAR AQUI LAS ETAPAS DE SW. Y DESPUES DESCRIBIR UNA A UNA EL INPUT Y EL OUPUT. PARA LA FASE DE REQUISITOS AQUÍ HAY ALGUNO:}



Con todo, el sistema tiene cuatro módulos:
\begin{itemize}
	\item Módulo de lectura de datos: lleva a cabo las tareas de lectura e inicialización de estructuras de datos.
	\item Módulo de inicialización: inicializa la solución inicial de acuerdo a la(s) contingencias recibidas del módulo anterior
	\item Módulo de búsqueda: lleva a cabo la búsqueda de una solución factible al problema.
	\item Módulo de entrega de datos: lleva a cabo las tareas de escritura y trazabilidad de las soluciones.
\end{itemize}



\subsection{Requisitos del sistema}
Una vez definidos los conceptos básicos, procedemos a recopilar las características y restricciones del sistema.

\subsubsection{Requisitos de entrada/salida}

\begin{enumerate}[label={\textbf{RIO\arabic*}}]
	\item  Una entrada al sistema se compondrá de dos partes: la información de la dependencia y la información del caso,
	de esta forma, la información común a varios casos será independiente de cada caso concreto.
	\item La información de la dependencia será un subdirectorio con el nombre de la dependencia, contendrá 4 ficheros:
	\begin{enumerate}[label*={\textbf{.\arabic*}}]
		\item  Lista de todos los sectores pertenecientes la unidad de control y los sectores elementales\footnote{
			Sector que comprende una zona del espacio aéreo que no es subdivisible empleando otros sectores. Recuérdese el sector LECMBDP (azul) de la \autoref{fig:2:sectorizacion-3d} se podía sustituir por otros más pequeños, por lo tanto no es elemental
		} por los que están formados cada uno de los sectores.
		
		\item  Matriz de Afinidad de los sectores de la dependencia (definida en la 	\autoref{section:2:sectores-y-sectorizacion})
		\item Lista de los sectores pertenecientes a la unidad de control, en la que nos indica el tipo de sector (véase~\ref{RD-tipos-sector}) y los núcleos a los que pertenece (ver~\ref{RD-sector-nucleo}).
	\end{enumerate}
	
\end{enumerate}


\subsubsection{Restricciones de dominio}
Las restricciones de dominio son aquellos requisitos del sistema que son impuestos únicamente por el dominio del problema, no por la propia naturaleza del sistema ni de forma externa.

\begin{enumerate}[label={\textbf{RD\arabic*}}]
	\item \label{RD-tipos-sector}  Cada sector tendrá un tipo de sector, que podrá ser Aproximación o Ruta
	\item  \label{RD-sector-nucleo} Cada sector tendrá uno o varios núcleos asociados, así como cada núcleo tendrá un conjunto de sectores (relación N a N)
	
\end{enumerate}