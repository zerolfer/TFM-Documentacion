
\section{Implementación} \label{capitulo:4}

Como ingeniero informático, este proyecto se ha desarrollado de forma orientada al proceso de ingeniería de software que consta de las siguientes etapas o procesos: 
\begin{enumerate}
    \item Planificación
    \item Análisis
    \item Diseño
    \item Implementación propiamente dicha
    \item Pruebas
    \item Despliegue y explotación    
\end{enumerate}
%planificación, análisis, diseño, implementación propiamente dicha, pruebas y despliegue/explotación. 
Adicionalmente existe la etapa de monitorización y seguimiento que se desarrollan posteriormente de la puesta en marcha del sistema en un entorno real, y esto no forma parte del presente TFM debido a su naturaleza de investigación (al igual que el propio máster), así que no se han llevado a cabo dichos procesos.

En las sucesivas secciones de este capítulo iremos describiendo los procesos llevados a cabo en cada una de estas etapas y las conclusiones alcanzadas en las mismas.

\subsection{Planificación}
\label{sec:4:planificacion}
En esta primera etapa presente en todo proyecto tratamos de fijar principalmente el concepto de alcance, que podríamos definir como el trabajo realizado para entregar un producto, servicio o resultado con
las funciones y características especificadas~\cite{PMBOK}.

Los proyectos de naturaleza de investigación e innovación como éste requieren de una cierta flexibilidad a la hora de definir el alcance, puesto que planteamos hipótesis iniciales que en función de ser válidas o no, plantearemos otras nuevas. Como éste TFM forma parte de un proyecto mayor, es más sencillo definir el alcance, aunque este sufrió alguna modificación.

El proyecto parte con la definición de las hipótesis iniciales recogidas en la \autoref{sec:Hipotesis}. 
Pretendía abarcar en primera instancia la comprensión del dominio del problema y la lógica de negocio sobre la que poder comenzar a trabajar, y tras ello, podría dar comienzo el desarrollo de la metaheurística VNS descrita en el \autoref{capitulo:3} y analizar los resultados comparándolos con el SA (experimentación, recopilado en el \autoref{capitulo:5}). Al comienzo del proyecto se implementarían los 5 tipos de VNS descritos previamente:
\begin{itemize}
	\item Variable Neighborhood Descent (VND)
	\item Reduced Variable Neighborhood Search (RVNS)
	\item Basic Variable Neighborhood Search (BVNS)
	\item General Variable Neighborhood Search (GVNS)
	\item Skewed Variable Neighborhood Search (SVNS)
\end{itemize}

Y posteriormente fue ampliado con la variación de la naturaleza de los entornos probabilista.

Por otro lado, el sistema organizativo acordado con los tutores para este TFM consistió en un desarrollo iterativo en ciclos marcados por metas en una jornada de trabajo en un despacho ofrecido por el grupo de investigación de 8 horas diarias de Febrero de 2019 a Junio de 2019. Finalmente las fechas fueron ampliadas. Se trata de una variación de los sprints de SCRUM combinada con un tablero KANBAN creado en la aplicación web Tello. 

A continuación describiremos cada uno de esos ciclos.

El primer ciclo incluía el proceso de planificación y de análisis junto al estudio del sistema y los documentos previos facilitados por los tutores (\textit{literature review}). Tuvo por meta la comprensión del dominio del problema (primera parte) y de la lógica de negocio utilizada en los antecedentes del TFM (véase la \autoref{capitulo:2:detalles-sistema})

Los demás ciclos únicamente requerirán de las fases de diseño, implementación y pruebas. La meta para el segundo era la implementación de la \faseuno{} (diseño, implementación y pruebas). El tercer ciclo como meta fue la implementación de una primera versión funcional del sistema implementando unicamente el VND, definiendo previamente los diferentes entornos a emplear. Para el cuarto ciclo se debían implementar el resto de variaciones del VNS así como realizar un despliegue del sistema empaquetandolo y enviandolo en forma de entrgable a los tutores (véase la \autoref{sec:4:despliegue}). El cuarto ciclo fue añadido posteriormente para subsanar errores e implementar los entornos probabilísticos que ya fueron descritos en la \autoref{capitulo:3:busqueda-divers-intens}. Finalmente se encuentra el ciclo de experimentación, donde comparamos los resultados y analizamos la eficiencia del sistema.

En la \autoref{sec:4:implementacion} se describen las tecnologías software utilizadas. %TODO: hacerlo efectivamente

% TODO: ¿cronograma?

\subsection{Análisis}
El objetivo de esta fase consiste en la educción de los requisitos que precisa el software a partir de los documentos e información facilitada por los tutores y demás stakeholders.

Los \textit{stakeholders} del proyecto son: CRIDA, los tutores, y los estudiantes de doctorado o máster implicados actual o anteriormente (Tino Tello, Jónatan Lara y Pablo Lozano) y será a quienes se deberá recurrir para responder dudas. El \textit{stakeholder} principal por supuesto es CRIDA, que guiará los objetivos principales y responderá a las preguntas de mayor complejidad y que competan a dominio del problema. Los tutores tendrán las mismas competencias pero a nivel del TFM y no del proyecto en su totalidad, y serán quienes faciliten la documentación inicial relativa al dominio del problema. El resto de \textit{stakeholders} y especialmente Tino Tello, proporcionarán la información relativa al software y aclararán conceptos de esta índole.

La documentación inicial facilitada por los tutores se basa en dos papers uno publicado y otro sin publicar a fecha de inicio del TFM~\cite{articulo1, articulo2}. Una vez leída la documentación inicial y comprendido el dominio del problema, se realizó una entrevista con Tino Tello para comprender la implementación del sistema \legacy{} para poder comenzar a trabajar. En dicha entrevista se obtuvo como conclusión la creación de un repositorio con control de versiones Git disponible para todos los \textit{stakeholders} y el inicio del primer ciclo de la planificación (definidos en \autoref{sec:4:planificacion}).

En este punto se identifican los requisitos del sistema tanto del dominio (obtenidos del software \legacy{} y de la documentación inicial) como del sistema (entrada/salida, interfaces, etc.) que se encuentran recopilados en las próximas secciones.

\subsubsection{Requisitos del dominio}
\label{sec:4:RD}

Los requisitos de dominio son aquellas restricciones del sistema que son impuestas únicamente por el dominio del problema, y no por la propia naturaleza del sistema ni de forma externa.

Este tipo de requisitos son necesarios para la constitución de una solución factible y permite cuantificar lo mejor o peor que son una solución de otra mediante el conteo (ponderado) del numero de restricciones incumplidas (véase el objetivo \ref{O2}). 

En el sistema \legacy{} existían dos tipos de restricciones: obligatorias y deseables, pues una vez alcanzada la factibilidad se intentaba mejorar. Sin embargo, en este sistema nos moveremos por soluciones infactibles buscando la que sea mejor, a ser posible factible pero no como condición necesaria. Estas restricciones son impuestas por la legislación española, recogidas en el BOE, fundamentalmente en:

\begin{itemize}
	\item Real Decreto 1001/2010, de 5 de agosto, por el que se establecen normas de seguridad aeronáutica en relación con los tiempos de actividad y los requisitos de descanso de los controladores civiles de tránsito aéreo. \\
	(\url{https://www.boe.es/eli/es/rd/2010/08/05/1001/con})
	
	\item Ley 9/2010, de 14 de abril, por la que se regula la prestación de servicios de tránsito aéreo, se establecen las obligaciones de los proveedores civiles de dichos servicios y se fijan determinadas condiciones laborales para los controladores civiles de tránsito aéreo. \\
	(\url{https://www.boe.es/eli/es/l/2010/04/14/9})
\end{itemize} 



En primer lugar tenemos las fundamentales:

%\begin{enumerate}[label={\textbf{RDF\arabic*}}]
\begin{enumerate}[label={\textbf{RD\arabic*}}, ref={RD\arabic*},  align=left]
	
	\item Todas las posiciones de control deben estar cubiertas por controladores de manera exclusiva, exhaustiva y bajo las restricciones definidas.
	\begin{enumerate}[label*={\textbf{.\arabic*}}]
		\item Cada sector y posición, deben ser cubiertos en los intervalos donde estén abiertos (exhaustividad).
		\item Cada sector y posición, deben ser cubiertos por un único controlador (exclusividad).
	\end{enumerate}
	
	\item Un controlador no puede tener dos asignaciones diferentes en el mismo instante.
	\begin{enumerate}[label*={\textbf{.\arabic*}}]
		\item Entendemos por asignación la combinación de sector y posición.
		\item un controlador no puede estar cubriendo en el mismo
		instante la posición de ejecutivo y planificador de un sector y tampoco puede estar
		asignado a dos sectores diferentes en el mismo instante, sea cual sea la posición.
	\end{enumerate}

	\item \label{RD:tipos-sector}  Cada sector tendrá un tipo de sector, que podrá ser Aproximación o Ruta

	\item  \label{RD:sector-nucleo} Cada sector tendrá uno o varios núcleos asociados, así como cada núcleo tendrá un conjunto de sectores (relación N a N)
	
	\item \label{RD:nucleo-controlador} Cada controlador estará acreditado para controlar un único núcleo.

\end{enumerate}



A continuación las relativas a la posición y asignación de los controladores:

\begin{enumerate}[resume*]
	\item \label{RD:acreditacion-valida} Una determinada posición podrá ser asignada a un controlador si el controlador está habilitado en el núcleo al que pertenece el sector que le corresponde, o bien en uno de los núcleos a los que pertenece el sector
	(si este es un sector común) independientemente de la sectorización por la que el sector se encuentre abierto.
	
	\item A un controlador tipo CON solo podrá asignársele una posición cuyo sector sea Ruta (véase la \autoref{table:2:acreditaciones})
	
	\item Los sectores o agrupaciones de dos sectores que se indique en la entrada del problema, se deberán cubrir con 4 controladores en el turno	de noche.

	\item Un controlador solo puede operar en su turno correspondiente: si pertenece al turno largo, en el turno largo y si pertenece al turno corto en el turno corto.

	\item Un controlador no puede cambiar de posición de control de una posición ejecutiva de un determinado sector a una posición ejecutiva de otro sector diferente, sin que exista un descanso entre medias, a no ser que ambos sectores sean afines (cambio de configuración). Nota: si no hay cambio de configuración de sectores no es posible que dos sectores diferentes posean volúmenes comunes.
	
	\item El número máximo de sectores por los que rota un controlador es 3.
	
	
	
	
\end{enumerate}



Y por último las relativas a los tiempos de trabajo y descanso de los controladores:

\begin{enumerate}[resume*]
	
	\item \label{RD:porcentaje-min-descanso} El porcentaje de tiempo de descanso mínimo en turno diurno (mañana o tarde), incluyendo turnos largos es del 25\% como mínimo.
	\begin{enumerate}[label*={\textbf{.\arabic*}}]
		\item En caso de los turnos de noche, el porcentaje de tiempo de descanso mínimo en el turno de noche será
	como mínimo del 33 \%.
	\end{enumerate}
	
	\item No es posible un periodo de trabajo continuo mayor de dos horas en los que el controlador no realice ningún periodo de descanso.
	
	\item No puede existir ningún periodo de dos horas y media en los que un controlador realice un periodo total de descanso menor de media hora. Es decir, dentro de una ventana de tiempo de dos horas y media un controlador debe tener mínimo 30 minutos de descanso, sin ser necesario que estos se realicen de forma continua.
	
	\item El tiempo mínimo de trabajo continuado es de 15 minutos.
	
	\item El tiempo mínimo de descanso continuado es de 15 minutos.
	
	\item El tiempo mínimo en posición de un controlador es de 15 minutos.
	
	\item Todos los controladores deben trabajar como mínimo 15 minutos.
		
\end{enumerate}


\subsubsection{Requisitos de entrada/salida}
Los requisitos de entrada y salida (I/O) son aquellos relativos a la los ficheros y formatos definidos tanto para la entrada al sistema como en la salida.

\begin{enumerate}[label={\textbf{RIO\arabic*}}, ref={RC\arabic*},  align=left]
	
	\item  Una entrada al sistema se compondrá de dos partes: la información de la dependencia y la información del caso, de esta forma, la información común a varios casos será independiente de cada caso concreto.
	
	\item La información de la dependencia será un subdirectorio con el nombre de la dependencia, contendrá 4 ficheros:
	\begin{enumerate}[label*={\textbf{.\arabic*}}]
		\item  Lista de todos los sectores pertenecientes la unidad de control y los sectores elementales\footnote{
			Sector que comprende una zona del espacio aéreo que no es subdivisible empleando otros sectores. Recuérdese el sector LECMBDP (azul) de la \autoref{fig:2:sectorizacion-3d} se podía sustituir por otros más pequeños, por lo tanto no es elemental
		} por los que están formados cada uno de los sectores.
		
		\item  Matriz de Afinidad de los sectores de la dependencia (definida en la \autoref{section:2:sectores-y-sectorizacion}).
		
		\item Lista de los sectores pertenecientes a la unidad de control, en la que nos indica el tipo de sector (véase~\ref{RD:tipos-sector}) y los núcleos a los que pertenece (ver~\ref{RD:sector-nucleo}).
	\end{enumerate}
	
\end{enumerate}

\subsection{Diseño}
Lorem ipsum

\subsection{Implementación}
\label{sec:4:implementacion}

Las herramientas software a utilizar serán:
\begin{itemize}
	\item Tello (planificación)
	\item IntelliJ IDEA (implementación)
	\item Jupyther Notebook (experimentación)
	\item Microsoft Office (diagramas)
	\item \TeX{} Studio (memoria)
\end{itemize}


\subsection{Pruebas}
Lorem ipsum

\subsection{Despliegue y explotación} 
\label{sec:4:despliegue}

Lorem ipsum



















\NOTE{CONTAR AQUI LAS ETAPAS DE SW. Y DESPUES DESCRIBIR UNA A UNA EL INPUT Y EL OUPUT. PARA LA FASE DE REQUISITOS AQUÍ HAY ALGUNO:}



Con todo, el sistema tiene cuatro módulos:
\begin{itemize}
	\item Módulo de lectura de datos: lleva a cabo las tareas de lectura e inicialización de estructuras de datos.
	\item Módulo de inicialización: inicializa la solución inicial de acuerdo a la(s) contingencias recibidas del módulo anterior
	\item Módulo de búsqueda: lleva a cabo la búsqueda de una solución factible al problema.
	\item Módulo de entrega de datos: lleva a cabo las tareas de escritura y trazabilidad de las soluciones.
\end{itemize}









\subsubsection{Detalles de implementación del sistema}
\label{sec:detalles-impl-sistema}
