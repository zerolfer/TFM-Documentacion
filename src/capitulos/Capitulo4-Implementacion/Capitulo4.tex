
\section{Implementación} \label{capitulo:4}

Como ingeniero informático, este proyecto se ha desarrollado de forma orientada al proceso de ingeniería de software que consta de las siguientes etapas o procesos: 
\begin{enumerate}
    \item Planificación
    \item Análisis
    \item Diseño
    \item Implementación propiamente dicha
    \item Pruebas
    \item Despliegue y explotación    
\end{enumerate}
%planificación, análisis, diseño, implementación propiamente dicha, pruebas y despliegue/explotación. 
Adicionalmente existe la etapa de monitorización y seguimiento que se desarrollan posteriormente de la puesta en marcha del sistema en un entorno real, y esto no forma parte del presente TFM debido a su naturaleza de investigación (al igual que el propio máster), así que no se han llevado a cabo dichos procesos.

En las sucesivas secciones de este capítulo iremos describiendo los procesos llevados a cabo en cada una de estas etapas y las conclusiones alcanzadas en las mismas.

\subsection{Planificación}
En esta primera etapa presente en todo proyecto tratamos de fijar principalmente el concepto de alcance, que podríamos definir como el trabajo realizado para entregar un producto, servicio o resultado con
las funciones y características especificadas~\cite{PMBOK}.

Los proyectos de naturaleza de investigación e innovación como éste requieren de una cierta flexibilidad a la hora de definir el alcance, puesto que planteamos hipótesis iniciales que en función de ser válidas o no, plantearemos otras nuevas. Como éste TFM forma parte de un proyecto mayor, es más sencillo definir el alcance, aunque este sufrió alguna modificación.

El proyecto parte con la definición de las hipótesis iniciales recogidas en la \autoref{sec:Hipotesis}. 
Pretendía abarcar en primera instancia la comprensión del dominio del problema y la lógica de negocio sobre la que poder comenzar a trabajar, y tras ello, podría dar comienzo el desarrollo de la metaheurística VNS descrita en el \autoref{capitulo:3} y analizar los resultados comparándolos con el SA (experimentación, recopilado en el \autoref{capitulo:5}). Al comienzo del proyecto se implementarían los 5 tipos de VNS descritos previamente:
\begin{itemize}
	\item Variable Neighborhood Descent (VND)
	\item Reduced Variable Neighborhood Search (RVNS)
	\item Basic Variable Neighborhood Search (BVNS)
	\item General Variable Neighborhood Search (GVNS)
	\item Skewed Variable Neighborhood Search (SVNS)
\end{itemize}

Y posteriormente fue ampliado con la variación de la naturaleza de los entornos probabilista.

Por otro lado, el sistema organizativo acordado con los tutores para este TFM consistió en un desarrollo iterativo en ciclos marcados por metas en una jornada de trabajo en un despacho ofrecido por el grupo de investigación de 8 horas diarias de Febrero de 2019 a Junio de 2019. Finalmente las fechas fueron ampliadas. Se trata de una variación de los sprints de SCRUM combinada con un tablero KANBAN creado en la aplicación web Tello. 

A continuación describiremos cada uno de esos ciclos.

El primer ciclo incluía el proceso de planificación y de análisis junto al estudio del sistema y los documentos previos facilitados por los tutores (\textit{literature review}). Tuvo por meta la comprensión del dominio del problema (primera parte) y de la lógica de negocio utilizada en los antecedentes del TFM (véase la \autoref{capitulo:2:detalles-sistema})

Los demás ciclos únicamente requerirán de las fases de diseño, implementación y pruebas. La meta para el segundo era la implementación de la \faseuno{} (diseño, implementación y pruebas). El tercer ciclo como meta fue la implementación de una primera versión funcional del sistema implementando unicamente el VND, definiendo previamente los diferentes entornos a emplear. Para el cuarto ciclo se debían implementar el resto de variaciones del VNS así como realizar un despliegue del sistema empaquetandolo y enviandolo en forma de entrgable a los tutores (véase la \autoref{sec:4:despliegue}). El cuarto ciclo fue añadido posteriormente para subsanar errores e implementar los entornos probabilísticos que ya fueron descritos en la \autoref{capitulo:3:busqueda-divers-intens}. Finalmente se encuentra el ciclo de experimentación, donde comparamos los resultados y analizamos la eficiencia del sistema.

En la \autoref{sec:4:implementacion} se describen las tecnologías software utilizadas. %TODO: hacerlo efectivamente

% TODO: ¿cronograma?

\subsection{Análisis}
Lorem ipsum

\subsection{Planificación}
Lorem ipsum

\subsection{Diseño}
Lorem ipsum

\subsection{Implementación}
\label{sec:4:implementacion}

Las herramientas software a utilizar serán:
\begin{itemize}
	\item Tello (planificación)
	\item IntelliJ IDEA (implementación)
	\item Jupyther Notebook (experimentación)
	\item Microsoft Office (diagramas)
	\item \TeX{} Studio (memoria)
\end{itemize}


\subsection{Pruebas}
Lorem ipsum

\subsection{Despliegue y explotación} 
\label{sec:4:despliegue}

Lorem ipsum



















\NOTE{CONTAR AQUI LAS ETAPAS DE SW. Y DESPUES DESCRIBIR UNA A UNA EL INPUT Y EL OUPUT. PARA LA FASE DE REQUISITOS AQUÍ HAY ALGUNO:}



Con todo, el sistema tiene cuatro módulos:
\begin{itemize}
	\item Módulo de lectura de datos: lleva a cabo las tareas de lectura e inicialización de estructuras de datos.
	\item Módulo de inicialización: inicializa la solución inicial de acuerdo a la(s) contingencias recibidas del módulo anterior
	\item Módulo de búsqueda: lleva a cabo la búsqueda de una solución factible al problema.
	\item Módulo de entrega de datos: lleva a cabo las tareas de escritura y trazabilidad de las soluciones.
\end{itemize}



\subsection{Requisitos del sistema}
Una vez definidos los conceptos básicos, procedemos a recopilar las características y restricciones del sistema.

\subsubsection{Requisitos de entrada/salida}

\begin{enumerate}[label={\textbf{RIO\arabic*}}]
	\item  Una entrada al sistema se compondrá de dos partes: la información de la dependencia y la información del caso,
	de esta forma, la información común a varios casos será independiente de cada caso concreto.
	\item La información de la dependencia será un subdirectorio con el nombre de la dependencia, contendrá 4 ficheros:
	\begin{enumerate}[label*={\textbf{.\arabic*}}]
		\item  Lista de todos los sectores pertenecientes la unidad de control y los sectores elementales\footnote{
			Sector que comprende una zona del espacio aéreo que no es subdivisible empleando otros sectores. Recuérdese el sector LECMBDP (azul) de la \autoref{fig:2:sectorizacion-3d} se podía sustituir por otros más pequeños, por lo tanto no es elemental
		} por los que están formados cada uno de los sectores.
		
		\item  Matriz de Afinidad de los sectores de la dependencia (definida en la 	\autoref{section:2:sectores-y-sectorizacion})
		\item Lista de los sectores pertenecientes a la unidad de control, en la que nos indica el tipo de sector (véase~\ref{RD-tipos-sector}) y los núcleos a los que pertenece (ver~\ref{RD-sector-nucleo}).
	\end{enumerate}
	
\end{enumerate}


\subsubsection{Restricciones de dominio}
Las restricciones de dominio son aquellos requisitos del sistema que son impuestos únicamente por el dominio del problema, no por la propia naturaleza del sistema ni de forma externa.

\begin{enumerate}[label={\textbf{RD\arabic*}}]
	\item \label{RD:tipos-sector}  Cada sector tendrá un tipo de sector, que podrá ser Aproximación o Ruta
	\item  \label{RD:sector-nucleo} Cada sector tendrá uno o varios núcleos asociados, así como cada núcleo tendrá un conjunto de sectores (relación N a N)
	
\end{enumerate}

\subsubsection{Detalles de implementación del sistema}
\label{sec:detalles-impl-sistema}
